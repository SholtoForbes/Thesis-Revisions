% numerical.tex

\cleardoublepage
\chapter{Optimised Trajectory Including Fly-Back}\label{chapter:Flyback}
	
This chapter presents the maximum payload-to-orbit trajectory of the Representative Launch System, with the fly-back of the scramjet accelerator included within the optimal trajectory calculation performed by LODESTAR. 
Flying back the scramjet accelerator for landing at the initial launch site is one of the primary enabling factors in the cost efficient operation of a launch system of this type. If the scramjet accelerator is launched onto a trajectory from which it is not able to fly-back, it must perform a downrange landing, necessitating transportation of the scramjet accelerator back to its launch site, a costly and time consuming process, and requiring international landing facilities to be available. 
Flying back the scramjet accelerator during the launch process removes the need for costly transportation from a downrange launch site, and allows for rapid refurbishment and re-use with short turn-around times.
In addition, if a launch site is used from which there is no downrange landing site, the scramjet accelerator must necessarily fly-back to the initial launch site. 


The fly-back of the scramjet accelerator requires changing the heading of the scramjet accelerator after third stage separation, covering the necessary ground distance for return, and decelerating to reduce the speed of the scramjet accelerator to landing approach speed, while maintaining a suitable descent angle to allow for a controlled approach. 
The return of the scramjet accelerator to the initial launch site is included in the optimisation process to assess whether it is possible for the fly-back of the scramjet accelerator to be achieved as a part of the launch process, and to maximise the overall payload-to-orbit efficiency of the launch system. This is compared to the optimal efficiency, maximum payload-to-orbit trajectory without fly-back in Case 2 (detailed in Chapter \ref{chapter:Ascent}) to assess the detrimental effects of the fly-back on the overall performance of the launch system. 
A sensitivity analysis is conducted, in a similar fashion to Chapter \ref{chapter:Ascent}. 
This sensitivity analysis allows the influence of the fly-back of the scramjet accelerator on the design sensitivities of the launch system to be analysed.


\section{Case 11: Maximum Efficiency Ascent Trajectory with Scramjet Accelerator Return}\label{sec:case11}
\begin{figure}[ht]% updated 15/8/19
	\centering
	\includegraphics[width=1\linewidth]{H:/github-home/LODESTAR-revisions/Results/mode11/GroundTrackStandard}
	\caption{Maximum efficiency, optimal payload-to-orbit trajectory path with the inclusion of scramjet accelerator fly-back (Case 11). } % note the heading angle at 10s is used
	\label{fig:GroundTrackStandard}
\end{figure}
\noindent
LODESTAR is used to optimise the trajectory of the rocket-scramjet-rocket launch system, including the return of the scramjet accelerator to its initial launch site. The optimised trajectory is shown in Figure \ref{fig:GroundTrackStandard}. 
The rocket-scramjet-rocket launch system is shown to be able to launch a small satellite to sun synchronous orbit, 
while flying-back the scramjet accelerator to the initial launch site location, and approaching the landing site at appropriately low altitude ($<$1km) and speed ($<$20m/s) to allow for landing. 
The maximum efficiency trajectory with fly-back attains a payload mass to SSO of \PayloadToOrbitStandard kg, a \textcolor{black}{-24.3}kg (\textcolor{black}{-15.5}\%) reduction in payload mass compared to the optimised ascent-only trajectory in Case 2, detailed in Chapter \ref{chapter:Ascent}. 
The benefits of flying back the scramjet accelerator to its initial launch site, compared to the alternative of transporting the scramjet accelerator back to the launch site from a remote landing, are likely to far outweigh this associated reduction in payload. 
\begin{table}[ht] % upodated 14/8/19
	\centering
	\begin{tabular}{l c } 
		\hline \textbf{Trajectory Condition}
		&Value 
		\\
		\hline \textbf{Payload to Orbit (kg)}
		& \textbf{\PayloadToOrbitStandard}
		\\
		\textbf{Total $\eta_{exergy}$ (\%)}
		& \textbf{\totalExergyEffStandard}
		\\
		\hline 
		\textbf{1$^{st}$ Stage $\eta_{exergy}$ (\%)}
		& \textbf{\firstExergyEffStandard}
		\\
		
		\textbf{Separation Alt, 1$\rightarrow$2 (km)}
		& \firstsecondSeparationAltStandard
		\\
		\textbf{Separation v, 1$\rightarrow$2 (m/s)}
		& \firstsecondSeparationvStandard
		\\
		\textbf{Separation $\gamma$, 1$\rightarrow$2 (deg)}
		& \firstsecondSeparationgammaStandard
		\\
		\hline 
		\textbf{2$^{nd}$ Stage $\eta_{exergy}$ (\%)}
		& \textbf{\secondExergyEffStandard}
		\\
		
		\textbf{Separation Alt, 2$\rightarrow$3 (km)}
		& \secondthirdSeparationAltStandard
		\\
		\textbf{Separation $v$, 2$\rightarrow$3 (m/s)}
		& \secondthirdSeparationvStandard
		\\
		\textbf{Separation $\gamma$, 2$\rightarrow$3 (deg)}
		& \secondthirdSeparationgammaStandard
		\\
		
		\textbf{2$^{nd}$ Stage Distance Flown (km)}
		& \SecondDistStandard
		\\
		\textbf{2$^{nd}$ Stage Return Fuel (kg)}
		& \returnFuelStandard
		\\
		\textbf{2$^{nd}$ Stage Return Distance (km)}
		& \returnDistStandard
		\\
		\hline 
		\textbf{3$^{rd}$ Stage $\eta_{exergy}$ (\%)}
		& \textbf{\thirddExergyEffStandard}
		\\
		
		\textbf{3$^{rd}$ Stage $t$, $q >$ 5kpa (s)}
		& \thirdqOverFiveStandard
		\\
		\textbf{3$^{rd}$ Stage Fuel Mass (kg)}
		& \thirdmFuelStandard
		\\
		\hline 
	\end{tabular} 
	\caption{Selected trajectory conditions for a maximum efficiency, optimal payload-to-orbit trajectory including scramjet accelerator fly-back (Case 11).}
\end{table}


\section{Ascent Trajectory}

\begin{figure}[ht]% updated 14/8/19
	\centering
	\includegraphics[width=0.9\linewidth]{H:/github-home/LODESTAR-revisions/Results/mode11/FirstStageStandard}
	\caption{The first stage of the optimal efficiency maximum payload-to-orbit trajectory with scramjet accelerator fly-back (Case 11). }
	\label{fig:FirstStageStandard}
\end{figure}
When the fly-back of the scramjet accelerator is included in the trajectory optimisation, the shape of the ascent trajectory of the launch system is altered significantly when compared to the ascent-only trajectory in Case 2, detailed in Chapter \ref{chapter:Ascent}.
 The first stage initially pitches towards the east.
 After pitchover, the \textcolor{black}{angle of attack is maintained at 0$^\circ$ for 35.6s, before it is reduced gradually to -3.9$^\circ$ at 71.9s} in order to adjust the altitude and trajectory angle, before increasing back to 0$^\circ$ at first-second stage separation. 
 The scramjet accelerator is released in an easterly direction, at a heading angle of \textcolor{black}{2.16}$^\circ$, an altitude of \firstsecondSeparationAltStandard km, and a trajectory angle of \firstsecondSeparationgammaStandard $^\circ$. 
 
  \begin{figure}[!ht] % updated 14/8/19
  	\centering
  	\includegraphics[width=1\linewidth]{H:/github-home/LODESTAR-revisions/Results/mode11/SecondStageStandard}
  	\caption{The acceleration of the scramjet accelerator flying a maximum efficiency, optimal payload-to-orbit trajectory with scramjet accelerator fly-back (Case 11). }
  	\label{fig:SecondStageStandard}
  \end{figure}
  These separation conditions cause an altitude increase to occur at the beginning of the scramjet accelerator trajectory, in the same manner to the optimised ascent-only trajectory in Case 2, detailed in Chapter \ref{chapter:Ascent}.
The altitude increase \textcolor{black}{after the first stage-scramjet accelerator separation} takes the scramjet accelerator to an altitude of \textcolor{black}{32.3}km at \textcolor{black}{47.1}s, and decreases the dynamic pressure of the scramjet accelerator to \textcolor{black}{14.9}kPa, \textcolor{black}{reaching close to the lower limit of operation of the scramjet engines,} \textcolor{black}{during which time} the bank angle of the scramjet accelerator is increased in order to reduce lift in the vertical direction, minimising the size of the initial altitude raising manoeuvre. 
 After the first-second stage separation, the bank angle is increased, at the maximum change rate, to \textcolor{black}{52.2}$^\circ$, further decreasing the lift of the scramjet accelerator and decreasing its altitude. As the altitude of the scramjet accelerator begins to reduce, the bank angle \textcolor{black}{reduces} and the angle of attack is raised to \textcolor{black}{3.9}$^\circ$ to increase lift, slowing the descent of the scramjet accelerator so that maximum dynamic pressure is not exceeded. 
 The bank angle then begins to increase once more in order to change the heading angle of the scramjet accelerator, reaching its maximum of \textcolor{black}{66.2}$^\circ$ at 135.0s flight time. 

After this point, the bank angle of the scramjet accelerator is maintained between \textcolor{black}{48.0}$^\circ$ and \textcolor{black}{52.5}$^\circ$, exhibiting higher bank angles towards the latter part of the ascent. At the end of the scramjet accelerator's ascent, the bank angle is reduced to under 2$^\circ$, so that lift is maximised during the pull-up. The third stage is released at exactly 0$^\circ$ bank angle, defined as a constraint on the end of the trajectory to ensure that the third stage rocket is released in the vertical plane, and is able to manoeuvre to orbit. 


The angle of attack of the scramjet accelerator is significantly higher over the trajectory in Case 11, where fly-back is included, compared to maximum payload-to-orbit trajectory with no fly-back in Case 2, detailed in Section \ref{sec:optimisednoreturn}. These significantly  higher angles of attack are a result of the high bank angle of the scramjet accelerator throughout its trajectory, which causes the lift vector of the scramjet accelerator to be inclined by the bank angle, so that a portion of the lift is used for changing the heading of the scramjet accelerator, rather than providing vertical force. 
 The higher angles of attack result in the optimal trajectory of the scramjet accelerator following a close to maximum dynamic pressure path for most of the duration of its trajectory, without the altitude raising manoeuvre observed in Section \ref{sec:optimisednoreturn}.
 The increase in angle of attack means that the scramjet accelerator no longer flies within the homogeneous region of the specific impulse of the C-REST engines. Instead, as shown in Figure \ref{fig:NetIspStandard}, the flight conditions are close to a region where an increase in angle of attack causes a sharp decrease in specific impulse. 
This indicates that the angle of attack, and consequently the allowable bank angle, of the scramjet accelerator is being limited by the performance of the C-REST engines. 
 The scramjet accelerator stays close to its maximum dynamic pressure until a pull-up is performed at \textcolor{black}{386.6}s flight time. 
 \begin{figure}[!ht]% updated 14/8/19
 	\centering
 	\includegraphics[width=\linewidth]{H:/github-home/LODESTAR-revisions/Results/mode11/NetIspStandard}
 	\caption{Net $I_{SP}$ contours for the scramjet accelerator at Mach numbers between 5.5 and 8.5, showing the optimised trajectory path (Case 11). }
 	\label{fig:NetIspStandard}
 \end{figure}

The higher angles of attack flown by the scramjet accelerator also have the consequence of decreasing the net specific impulse of the scramjet accelerator during its acceleration, with the maximum specific impulse being decreased by -\textcolor{black}{3.4}\%.
The overall exergy efficiency of the scramjet accelerator is decreased, to \secondExergyEffStandard\%$\eta$, a decrease of \textcolor{black}{-0.643}\%$\eta$ (\textcolor{black}{-14.6}\%) compared to the maximum payload-to-orbit trajectory with no fly-back. This exergy efficiency decrease is due partially to the decrease in the specific impulse of the scramjet engines, but more significantly is attributed to the fuel necessary for the return flight resulting in less fuel being available for the ascent of the scramjet accelerator, and thus less `useful' work being attained from the fuel.
A total fuel mass of \textcolor{black}{1304.2}kg is used during the scramjet accelerator's acceleration phase, out of a total of 1562kg of available fuel. This reduction in fuel mass used, along with the reduction in net specific impulse due to the higher angle of attack values, reduces the speed at second-third stage separation by \textcolor{black}{-134.0}m/s (\textcolor{black}{-5.1}\%) compared to the maximum payload-to-orbit case with no scramjet accelerator fly-back in Case 2. The scramjet accelerator pulls up to \secondthirdSeparationAltStandard km altitude and \secondthirdSeparationgammaStandard $^\circ$ trajectory angle before the second-third stage separation, a difference of only \textcolor{black}{-0.94}km (\textcolor{black}{-2.1}\%) and \textcolor{black}{+0.4}$^\circ$ (\textcolor{black}{+3.3}\%) compared to the maximum payload-to-orbit trajectory without fly-back, indicating that the inclusion of fly-back does not have a large effect on the magnitude of the pull-up manoeuvre. 

The third stage trajectory, shown in Figure \ref{fig:ThirdStageStandard}, is very similar to the third stage trajectory without fly-back in Case 2, shown in Section \ref{sec:optimisednoreturn}, although it is released at a slightly lower altitude and speed and thus spends a slightly longer time in-atmosphere. 
The exergy efficiency of the third stage is decreased by \textcolor{black}{-2.386}\%$\eta$ (\textcolor{black}{-15.4}\%) when compared to the maximum payload-to-orbit trajectory with no scramjet accelerator fly-back in Case 2. This lowered efficiency is primarily due to the lower speed of the third stage release, which increases the availability losses of the third stage due to propulsive inefficiencies, as well as the losses due to the additional energy needed to accelerate the increased third stage fuel and structural mass to orbit. 


\begin{figure}[ht!]% updated 14/8/19
\centering
\includegraphics[width=1\linewidth]{H:/github-home/LODESTAR-revisions/Results/mode11/ThirdStageStandard}
\caption{The third stage trajectory of a maximum efficiency, optimal payload-to-orbit trajectory with scramjet accelerator fly-back (Case 11). }
\label{fig:ThirdStageStandard}
\end{figure}


\section{Fly-Back Trajectory}
\noindent
The optimised fly-back trajectory for Case 11 is shown in Figure \ref{fig:ReturnStandard}.
The scramjet accelerator is shown to be capable of fly-back, using \returnFuelStandard kg of fuel, \textcolor{black}{16.5}\% of the total scramjet accelerator fuel. It is assumed that the aerodynamics of the scramjet accelerator are unchanged during return, ie. The bay doors close immediately after third stage release. 
Throughout its fly-back the scramjet accelerator performs distinct skipping manoeuvres, and ignites the scramjet engines a total of three times. 
These skips are consistent with previous research on hypersonic vehicles, which has shown that a periodic skipping trajectory increases the downrange distance achievable during both powered and unpowered flight\cite{Moshman2014,Darby2011,Toso2015,Tetlow1992,Eggers1957,Kanda2007,Chai2015}, and serve to reduce the fuel necessary for the return flight. 
\begin{figure}[ht!] % updated 14/8/19
	\centering
	\includegraphics[width=1\linewidth]{H:/github-home/LODESTAR-revisions/Results/mode11/ReturnStandard}
	\caption{The fly-back trajectory of the scramjet accelerator flying a maximum efficiency, optimal payload-to-orbit trajectory (Case 11). }
	\label{fig:ReturnStandard}
\end{figure}

It is observed that the optimised trajectory exhibits characteristics which can be separated into three distinct segments; 1. initial turn, 2. boost-skip, and 3. approach, as indicated in Figure \ref{fig:ReturnStandard}. 
 
\subsubsection{ Initial Turn}
The scramjet accelerator separates from the third stage rocket at a bank angle of 0$^\circ$, and then increases its bank angle at close to the maximum change rate until \textcolor{black}{94.5}s return flight time, at which point \textcolor{black}{ the maximum of 90.0}$^\circ$ bank angle is reached. This high bank angle initially serves to decrease the lift of the scramjet accelerator, minimising the height of the first skip, with the angle of attack of the scramjet accelerator kept low, indicating that the fly-back of the scramjet accelerator is a key component of the performance trade-off at second-third stage separation. As the scramjet accelerator reaches the zenith of its initial skip, at \textcolor{black}{73.6}s flight time and \textcolor{black}{66.5}km altitude, the angle of attack is rapidly increased, up to the maximum of \textcolor{black}{10.0}$^\circ$, and shortly after the bank angle begins to decrease below its maximum. This increase in angle of attack and decrease in bank angle generates additional lift to slow the descent of the scramjet accelerator into the trough of the first skip, ensuring that the dynamic pressure limit is not exceeded. 
Although the bank angle is reduced from its maximum, a high bank angle is maintained as the scramjet accelerator descends out of its skip, beginning a sharp change in heading angle as the dynamic pressure increases. 
This early heading angle change serves to minimise the down-range distance flown, and reduce the fuel necessary for fly-back. 


At \textcolor{black}{194.1}s flight time, as the vehicle descends and dynamic pressure increases rapidly, the scramjet engines are ignited. The C-REST engines are powered-on in the trough between the first and second skips, at a point of high potential specific impulse, and initially burn for \textcolor{black}{33.8}s. During the initial burn, the L/D of the scramjet accelerator increases significantly, due to the scramjet engine flow paths of the scramjet accelerator generating thrust, rather than drag. 
This increase in L/D raises the altitude of the scramjet accelerator and, along with heavy banking, changes the heading of the scramjet accelerator significantly. 
The burn is limited by the lower inlet dynamic pressure limit of the C-REST engines, of 20kPa, and terminates at \textcolor{black}{227.9}s flight time. 

\subsubsection{ Boost-Skip}\label{sec:boost}

After the initial burn ends, the angle of attack of the scramjet accelerator is decreased to 3.2$^\circ$, and the scramjet accelerator executes its second skip. Once the scramjet accelerator is descending again, and as soon as the dynamic pressure is high enough for C-REST engine operation at \textcolor{black}{369.0}s return flight time, the scramjet engines are once again ignited.
During the second burn, the angle of attack of the scramjet accelerator is increased, to modify the temperature and Mach number at the inlet of the C-REST engines so that the maximum specific impulse is obtained from the C-REST engines during the burn. 
The angle of attack varies between \textcolor{black}{4.7}$^\circ$ to \textcolor{black}{3.7}$^\circ$ during the second burn, and the L/D is once again raised significantly, initiating the third skip. 
This skip raises the altitude of the scramjet accelerator to \textcolor{black}{53.3}km, before it decreases once again. 
The third and last burn is initiated at \textcolor{black}{581.7}s and lasts until \textcolor{black}{615.9}s, when the remaining fuel has been depleted. Before the third burn, the angle of attack is \textcolor{black}{first increased up to 7.5$^\circ$, and then decreased at the beginning of the burn, to a minimum of 4.1}$^\circ$ during the burn. The angle of attack values observed are similar to those observed during the second burn, indicating that these angle of attack values obtain a high specific impulse from the C-REST engines. This can be observed in Figure \ref{fig:returnIspStandard}, which shows the specific impulse profile of the return flight during the boost-skip phase. 

After the third burn phase, the angle of attack is initially controlled so that the skipping trajectory of the scramjet accelerator is dampened.
Immediately after the third burn phase, the angle of attack is reduced, to \textcolor{black}{2.84}$^\circ$. This reduction coincides with the ascent portion of the fourth skip, reducing the lift, and the amount of altitude gained. 
As the zenith of the forth skip is reached, the angle of attack is increased to \textcolor{black}{7.5}$^\circ$, increasing the lift, and once again slowing the descent. 
This high angle of attack is \textcolor{black}{increased until 984.9s, in stages corresponding with the descent of the last two small skips}.
It is notable that the \textcolor{black}{angle of attack is controlled in this way during the unpowered portion of the trajectory, as it indicates that the skips are being damped.} This implies that some degree of skipping is desirable after the final scramjet burn, but that the angle of attack is being controlled to produce optimally sized skips, which reduce in size as the vehicle decelerates. 

\subsubsection{Approach}

After the final small skip, at \textcolor{black}{1046.0}s flight time, the angle of attack is adjusted, so that a gradual, controlled descent is initiated. 
After the skip phase, as the vehicle is approaching Mach 1, the angle of attack is reduced gradually to bring the scramjet accelerator down below 1km altitude at termination, in a controlled manner. At \textcolor{black}{1455.0}s, the bank angle is increased, in order to perform a final adjustment of the heading angle, to bring the scramjet accelerator back to its initial launch site. 
The scramjet accelerator drops below 1km altitude at \textcolor{black}{-17.0}$^\circ$ trajectory angle and a speed of \textcolor{black}{95.2}m/s (Mach 0.28). It is assumed that the scramjet accelerator is able to perform a landing manoeuvre after this point. 


\textcolor{black}{
	\section{Modelling of Uncertainties}}\label{sec:uncreturn}
\noindent
A Monte-Carlo study is conducted, using the same uncertainty magnitudes as in Section \ref{sec:unc}, with more detail available in Appendix \ref{sec:uncquant}. A performance-based confidence interval of 6.8-233.6kg in payload-to-orbit is calculated from this Monte-Carlo study using 97.5\% percentile limits. This confidence interval is similar in magnitude to the confidence interval with no fly-back presented in Section \ref{sec:unc}, and similarly illustrates the relatively large uncertainties present due to possible variations in launch system performance. Once again, however, this interval is entirely positive, indicating that it is likely that some payload-to-orbit can be achieved. 

%\textcolor{black}{
%	\section{Case 11: Aerothermal Heating Analysis}}\label{sec:therm2}
%\noindent
%The aerothermal heating on the scramjet accelerator and third stage rocket during flight has been investigated in Appendix \ref{sec:therm}. On the scramjet accelerator, the temperature on the windward surfaces of the tungsten nose and the Carbon-Carbon wing leading edge reach 2015.1K and 1665.8K respectively. 
%These locations are assumed to have no direct thermal connection to the internal insulation, and the temperatures that have been observed are within the operational regimes of the external heat shielding materials, which for the purposes of this study are considered sufficient for shielding during flight.
%Various locations on the body of the scramjet accelerator are investigated, with the interior heat shield surfaces of the nose cone reaching a maximum of 1363.1K, the cowl 1384.9K, the wing 1326.4K, and the tail 1142.3K.
%These temperatures are well within the operational regime of the C-C external heat shielding material, although the internal temperatures are at the limits of the operational regime of the internal multilayer insulation, which is nominally designed for operation in the 900-1100$^\circ$C (1173-1373K) range.
%Temperatures on the nose and body of the third stage rocket are investigated and found to be significantly lower than those experienced by the scramjet accelerator. The nose tip of the third stage reaches a maximum inner surface temperature of 953.5K, while the nose cone and body reach maximum temperatures of 475.9K and 779.8K respectively. These relatively low temperatures indicate that the current heat shielding design is likely appropriate for resisting the thermal loading experienced by the third stage rocket.

\section{Energy Usage Analysis}

\begin{table}[!ht]% updated 29/12/19
	\centering
	\begin{tabular}{l c c} 
		& \textbf{Case 2} & \textbf{Case 11} \\
		\hline \textbf{Trajectory Condition}
		& No Fly-Back
		& With Fly-Back
		\\
		\textcolor{blue}{Total KE + PE of Payload}
		& 5.94 GJ
		& 5.02 GJ
		\\
		
		\textbf{First Stage }  
		&
		&
		\\
		\textbf{Ascent Fuel Exergy} 
		&\textbf{\firstEnergyStandardNoReturn} GJ
		&\textbf{\firstEnergyStandard} GJ
		\\
		
		\textcolor{blue}{KE + PE of Payload}
		& \firstWpayloadStandardNoReturn \% (0.22 GJ)
		& \firstWpayloadStandard \% (0.19 GJ)
		\\
		\textcolor{red}{KE + PE of  2$^{nd}$ \& 3$^{rd}$ Stage}
		& \firstWnextStageStandardNoReturn \% (13.66 GJ)  & \firstWnextStageStandard \% (13.79 GJ)
		\\
		\textcolor{red}{Overcoming Drag} 
		& \WDoneStandardNoReturn \% (5.03 GJ) & \WDoneStandard \% (4.87 GJ)
		\\
		\textcolor{red}{KE + PE of 1$^{st}$ Stage Structural Mass} 
		& \WoneStandardNoReturn \% (2.04 GJ) & \WoneStandard \% (2.06 GJ)
		\\
		\textcolor{red}{KE + PE of 1$^{st}$ Stage Fuel Mass} 
		& \WmFoneStandardNoReturn \% (5.39 GJ) & \WmFoneStandard \% (5.41 GJ)
		\\ 
		\textcolor{red}{Availability Losses}  
		& \PlossoneCombinedStandardNoReturn \% (180.95 GJ) & \PlossoneCombinedStandard \% (181.00 GJ)
		\\ 
		\textbf{Scramjet Accelerator}  
		&
		&
		\\
		\textbf{Ascent Fuel Exergy} 
		& \textbf{\secondEnergyStandardNoReturn} GJ & \textbf{\secondEnergyStandard} GJ
		\\
		\textcolor{blue}{KE + PE of Payload}
		& \secondWpayloadStandardNoReturn \% (0.39 GJ) & \secondWpayloadStandard \% (0.28J)
		\\
		\textcolor{red}{KE + PE of 3$^{rd}$ Stage}
		& \secondWnextStageStandardNoReturn \% (7.85 GJ) & \secondWnextStageStandard \% (6.76 GJ)
		\\
		\textcolor{red}{Overcoming Drag}
		& \WDsecondStandardNoReturn \% (38.26 GJ) & \WDsecondStandard \% (33.56 GJ)
		\\
		\textcolor{red}{KE + PE of Scramjet Accelerator Structural Mass}  
		& \WsecondStandardNoReturn \% (12.39 GJ) & \WsecondStandard \% (10.58 GJ)
		\\
		\textcolor{red}{KE + PE of scramjet accelerator Fuel Mass}  
		& \WmFsecondStandardNoReturn \% (2.49 GJ) & \WmFsecondStandard \% (2.27 GJ)
		\\
		\textcolor{red}{Propulsion Inefficiency}  
		& \PlosssecondStandardNoReturn \% (126.01 GJ) & \PlosssecondStandard \% (103.00 GJ)
		\\
		
		
		
		\textbf{Return Fuel Exergy} 
		& - & \textbf{\returnEnergyStandard} GJ
		\\
		KE + PE of Scramjet Accelerator Structural Mass
		& - & \WreturnStandard \% (-17.61 GJ)
		\\
		\textcolor{red}{Overcoming Drag}
		& - & \WDreturnStandard \% (29.47 GJ)
		\\
		\textcolor{red}{KE + PE of scramjet accelerator Fuel Mass}  
		& - & \WmFreturnStandard \% (-0.51 GJ)
		\\
		\textcolor{red}{Availability Losses}  
		& - & \PlossreturnCombinedStandard \% (19.56 GJ)
		\\
		\textbf{Third Stage}  
		&
		&
		\\
		
		\textbf{In-Atmosphere Fuel Exergy}  
		& \textbf{\thirdEnergyStandardNoReturn} GJ & \textbf{\thirdEnergyStandard} GJ
		\\
		\textcolor{blue}{KE + PE of Payload}  
		&\thirddExergyEffAtmStandardNoReturn \% (0.73 GJ) &\thirddExergyEffAtmStandard \% (0.72 GJ)
		\\
		\textcolor{red}{Overcoming Drag}  
		& \WDthreeStandardNoReturn \% (0.11 GJ) & \WDthreeStandard \% (0.12 GJ)
		\\
		\textcolor{red}{KE + PE  of 3$^{rd}$ Stage Structural Mass}  
		& \WthreeStandardNoReturn \% (1.34 GJ) & \WthreeStandard \% (1.57 GJ)
		\\
		
		\textcolor{red}{KE + PE  of 3$^{rd}$ Stage Fuel Mass}  
		& \WmFthreeStandardNoReturn \% (9.28 GJ) & \WmFthreeStandard \% (10.30 GJ)
		\\
		
		\textcolor{red}{KE + PE of Heat Shield}  
		
		& \WHSthreeStandardNoReturn \% (1.22 GJ) & \WHSthreeStandard \% (1.41 GJ)
		\\
		
		\textcolor{red}{Availability Losses}   
		& \PlossthreeStandardNoReturn \% (4.10 GJ) & \PlossthreeStandard \% (4.58 GJ)
		\\
		\textbf{Circularisation and Hohman Transfer Fuel Exergy}  
		& \textbf{\HTExergyStandardNoReturn}  GJ & \textbf{\HTExergyStandard}  GJ
		\\
		\textcolor{blue}{KE + PE of Payload}  
		& \HTeffStandardNoReturn \% (4.60 GJ) & \HTeffStandard \% (3.83 GJ)
		\\
		\textcolor{red}{All Other Energy Losses}  
		& \HTlossStandardNoReturn \% (14.46 GJ) & \HTlossStandard \% (13.62 GJ)
		\\
		\hline 
	
	\end{tabular} 
	\caption{An energy usage breakdown of the ascent trajectories, both with, and without, scramjet accelerator fly-back (Cases 11 \& 2). Blue indicates a 'productive' energy usage, whereas red indicates energy 'wastage'. Negative energy indicates energy being supplied.}
	\label{tab:effStandard}
\end{table}


An energy usage analysis is conducted for Case 11; a maximum payload-to-orbit trajectory, including the fly-back of the scramjet accelerator. This is compared to the energy usage breakdown of Case 2; the optimised trajectory without the fly-back of the scramjet accelerator in Table \ref{tab:effStandard}. Similarly to Section \ref{sec:exergy1}, the energy used to accelerate the payload is shown, along with the energy imparted to the successive stages; the energy used overcoming drag; the energy used imparting energy to the structural mass of each stage, which is separated; and the availability losses. 



The fly-back of the scramjet accelerator reduces the fuel, and thus the fuel exergy, available to the scramjet accelerator during ascent.
This lower exergy, along with the altered manoeuvrability needs of the scramjet accelerator when the fly-back is included, results in a raising of the altitude and trajectory angle at the first-second stage separation. The increased altitude and trajectory angle at separation increases the fuel mass that the first stage rocket is able to use efficiently, and also increases the exergy efficiency of the first stage, due to lower drag and higher propulsive efficiency, partly compensating for the decrease in the efficiency of the scramjet accelerator due to fly-back. Overall, when the fly-back is included in Case 11, more of the exergy of the first stage is utilised imparting energy upon the combination of the payload and the successive stages, at \textcolor{black}{6.92}\% (\textcolor{black}{13.97} GJ), compared to \textcolor{black}{6.87}\% (\textcolor{black}{13.87} GJ) without scramjet accelerator fly-back in Case 2.



When flying a trajectory that includes the scramjet accelerator's fly-back, the drag losses during the ascent of the scramjet accelerator consist of a larger percentage of the ascent fuel exergy usage  (\WDsecondStandard \%, compared to \WDsecondStandardNoReturn \% without fly-back). This is despite the lower speeds over which the scramjet accelerator is accelerating when fly-back is included, and is due to the less favourable first-second stage separation conditions, as well as the high banking throughout the acceleration requiring a higher angle of attack to maintain lift. 
The energy necessary to return the scramjet accelerator to its initial launch location is provided by both the fuel used during the return (\textcolor{black}{30.92} GJ), as well as the kinetic and potential energy imparted upon the scramjet accelerator during its ascent. Significantly more energy is required to overcome drag during the return (\textcolor{black}{29.47} GJ) than is available from the kinetic and potential energy of the scramjet accelerator (\textcolor{black}{17.61} GJ), illustrating the necessity for igniting the scramjet engines during the return flight. 


When the fly-back is included, the second-third stage separation occurs at a lower altitude and speed, and the lower fuel exergy of the scramjet accelerator during its ascent results in less energy being imparted upon the payload and third stage by the scramjet accelerator (\textcolor{black}{7.04} GJ), compared to the trajectory without fly-back (\textcolor{black}{8.24} GJ). 
The lower, slower separation point when fly-back is included causes the losses of the third stage to increase from all sources, and for more energy to be used exiting the atmosphere. 



\section{Design Sensitivity Analysis}\label{sec:sensitivity}

It has been shown that the fly-back of the scramjet accelerator has a significant effect on the performance of the rocket-scramjet-rocket launch system, and that the maximum payload-to-orbit optimised trajectory changes significantly to compensate for the additional requirement of successfully returning the scramjet accelerator stage. This section investigates the sensitivity of the launch system to changes in the vehicle design in the same manner as Section \ref{sec:sensitivityNoReturn}, with the fly-back of the scramjet accelerator included, \textcolor{black}{in order to investigate the response of the optimised trajectory to variations in the design of the launch system, and to inform future design decisions.} This sensitivity study varies the following parameters by $\pm$10\%:
\begin{itemize}
	\item Case 12: Dynamic Pressure, 
	\item Case 13: Specific Impulse,
	\item Case 14: Scramjet Accelerator Drag,
	\item Case 15: Scramjet Accelerator Mass,
	\item Case 16: Scramjet Accelerator Fuel Mass,
	\item Case 17: Third Stage Mass,
	\item Case 18: Third Stage Thrust.
\end{itemize}
As in Section \ref{sec:sensitivityNoReturn}, the effect of third stage drag is negligible. For this reason, variation in the third stage drag is omitted in this section. 

The launch system is able to successfully place a small satellite in orbit for every performance condition that has been tested, while returning the scramjet accelerator to its initial launch location for landing. 
Every maximum payload-to-orbit optimised trajectory exhibits considerable banking during the scramjet accelerator's ascent trajectory, as well as a pull-up of the scramjet accelerator before third stage release. 
In every case the optimised return flight path exhibits initial turn, boost-skip and approach phases. 




Across the trajectories studied, there are two distinct shapes for the return trajectory of the scramjet accelerator. The more common return trajectory shape has been shown in the preceding section, and consists of three or more large skips to begin the return trajectory. The second trajectory shape exhibits a small second skip, with the first two burns very closely spaced, or combined into one longer burn. An example of this second type of return trajectory is shown in Figure \ref{fig:ReturnComparison}. During the first two burns, a higher bank angle is maintained when compared to the large skip trajectory shape, however, after the first two burns are completed, the bank angle is reduced more rapidly. 
During simulations, it was observed that on occasion, the optimal return trajectory type would change as the initial guess or problem setup was altered, with no significant change in the payload-to-orbit capabilities of the launch system. This variability suggests that there is minimal difference between the two shapes of return trajectory, and that both can potentially lead to efficient return flights. 





\subsection{Case 12: Maximum Dynamic Pressure Sensitivity with Fly-Back}

\begin{table}[ht]% updated 11/1/20
	\centering
	\begin{tabular}{l c c c c c c} 
		\hline \textbf{Trajectory Condition}   \qquad  $q_{max}$:
		&45kPa
		&47.5kPa
		&50kPa
		&52.5kPa
		&55kPa
		& $\Delta/\Delta$/\%$q_{max}$
			\\
			\hline \textbf{Payload to Orbit (kg)}
			& \textbf{\PayloadToOrbitqForty}
			& \textbf{\PayloadToOrbitqFortyFive}
			& \textbf{\PayloadToOrbitqStandard}
			& \textbf{\PayloadToOrbitqFiftyFive}
			& \textbf{\PayloadToOrbitqSixty}
			&\textbf{1.4}
			\\
			\textbf{Total $\eta_{exergy}$ (\%)}
			& \textbf{\totalExergyEffqForty}
			& \textbf{\totalExergyEffqFortyFive}
			& \textbf{\totalExergyEffqStandard}
			& \textbf{\totalExergyEffqFiftyFive}
			& \textbf{\totalExergyEffqSixty}
			& \textbf{0.00014}
			\\
			\hline 
			\textbf{1$^{st}$ Stage $\eta_{exergy}$ (\%)}
			& \textbf{\firstExergyEffqForty}
			& \textbf{\firstExergyEffqFortyFive}
			& \textbf{\firstExergyEffqStandard}
			& \textbf{\firstExergyEffqFiftyFive}
			& \textbf{\firstExergyEffqSixty}
			& -
			\\
			\textbf{Separation Alt, 1$\rightarrow$2 (km)}
			& \firstsecondSeparationAltqForty
			& \firstsecondSeparationAltqFortyFive
			& \firstsecondSeparationAltqStandard
			& \firstsecondSeparationAltqFiftyFive
			& \firstsecondSeparationAltqSixty
			&-0.09
			\\
			\textbf{Separation v, 1$\rightarrow$2 (m/s)}
			& \firstsecondSeparationvqForty
			& \firstsecondSeparationvqFortyFive
			& \firstsecondSeparationvqStandard
			& \firstsecondSeparationvqFiftyFive
			& \firstsecondSeparationvqSixty
			& -
			\\
			\textbf{Separation $\gamma$, 1$\rightarrow$2 (deg)}
			& \firstsecondSeparationgammaqForty
			& \firstsecondSeparationgammaqFortyFive
			& \firstsecondSeparationgammaqStandard
			& \firstsecondSeparationgammaqFiftyFive
			& \firstsecondSeparationgammaqSixty
			& -
			\\
			\hline 
			\textbf{2$^{nd}$ Stage $\eta_{exergy}$ (\%)}
			& \textbf{\secondExergyEffqForty}
			& \textbf{\secondExergyEffqFortyFive}
			& \textbf{\secondExergyEffqStandard}
			& \textbf{\secondExergyEffqFiftyFive}
			& \textbf{\secondExergyEffqSixty}
			& -
			\\
			\textbf{Separation Alt, 2$\rightarrow$3 (km)}
			& \secondthirdSeparationAltqForty
			& \secondthirdSeparationAltqFortyFive
			& \secondthirdSeparationAltqStandard
			& \secondthirdSeparationAltqFiftyFive
			& \secondthirdSeparationAltqSixty
			& -
			\\
			\textbf{Separation $v$, 2$\rightarrow$3 (m/s)}
			& \secondthirdSeparationvqForty
			& \secondthirdSeparationvqFortyFive
			& \secondthirdSeparationvqStandard
			& \secondthirdSeparationvqFiftyFive
			& \secondthirdSeparationvqSixty
			& -
			\\
			\textbf{Separation $\gamma$, 2$\rightarrow$3 (deg)}
			& \secondthirdSeparationgammaqForty
			& \secondthirdSeparationgammaqFortyFive
			& \secondthirdSeparationgammaqStandard
			& \secondthirdSeparationgammaqFiftyFive
			& \secondthirdSeparationgammaqSixty
			& -
			\\
			\textbf{2$^{nd}$ Stage Flight Time (s)}
			& \secondFlightTimeqForty
			& \secondFlightTimeqFortyFive
			& \secondFlightTimeqStandard
			& \secondFlightTimeqFiftyFive
			& \secondFlightTimeqSixty
			&-5.73
			\\
			\textbf{2$^{nd}$ Stage Distance Flown (km)}
			& \SecondDistqForty
			& \SecondDistqFortyFive
			& \SecondDistqStandard
			& \SecondDistqFiftyFive
			& \SecondDistqSixty
			&-8.33
			\\
			\textbf{2$^{nd}$ Stage Return Fuel (kg)}
			& \returnFuelqForty
			& \returnFuelqFortyFive
			& \returnFuelqStandard
			& \returnFuelqFiftyFive
			& \returnFuelqSixty
			& -
			\\
			\textbf{2$^{nd}$ Stage Return Distance (km)}
			& \returnDistqForty
			& \returnDistqFortyFive
			& \returnDistqStandard
			& \returnDistqFiftyFive
			& \returnDistqSixty
			& -
			\\
			\hline 
			\textbf{3$^{rd}$ Stage $\eta_{exergy}$ (\%)}
			& \textbf{\thirddExergyEffqForty}
			& \textbf{\thirddExergyEffqFortyFive}
			& \textbf{\thirddExergyEffqStandard}
			& \textbf{\thirddExergyEffqFiftyFive}
			& \textbf{\thirddExergyEffqSixty}
			& \textbf{0.146}
			\\
			\textbf{3$^{rd}$ Stage $t$, $q >$ 5kpa (s)}
			& \thirdqOverFiveqForty
			& \thirdqOverFiveqFortyFive
			& \thirdqOverFiveqStandard
			& \thirdqOverFiveqFiftyFive
			& \thirdqOverFiveqSixty
			& -
			\\
			\textbf{3$^{rd}$ Stage Fuel Mass (kg)}
			& \thirdmFuelqForty
			& \thirdmFuelqFortyFive
			& \thirdmFuelqStandard
			& \thirdmFuelqFiftyFive
			& \thirdmFuelqSixty
			&-1.44
			\\
			\hline 
		\end{tabular} 
	\caption{Comparison of key trajectory parameters with variation in the maximum dynamic pressure of the scramjet accelerator, with fly-back (Case 12).}
	\label{tab:qvarreturn}
\end{table}

\noindent
The maximum dynamic pressure allowable during flight is varied by $\pm$10\% in order to determine the sensitivity of the launch system to the structural and thermal limitations of the scramjet accelerator.  
Table \ref{tab:qvarreturn} shows a summary of the key parameters of each optimised trajectory, and trajectory comparison plots are shown in Appendix \ref{sec:app_comparison21}. The variation in each trajectory condition normalised to a one percent variation in the dynamic pressure is shown, if there is a clear trend. The payload-to-orbit of the launch system improves by \textcolor{black}{+5.5}kg (\textcolor{black}{+4.2}\%) at 55kPa, and decreases by \textcolor{black}{-22.2}kg (\textcolor{black}{-16.8}\%) at 45kPa.
The overall exergy efficiency of the system increases as the maximum dynamic pressure increases, by \textcolor{black}{+0.049}\%$\eta$ at 55kPa, and decreases as the maximum dynamic pressure decreases, by \textcolor{black}{-0.227}\%$\eta$ at 45kpa. 

\textcolor{black}{
When fly-back is included, no trends are observed in the exergy efficiencies of the first stage or scramjet accelerator due to maximum dynamic pressure variation. Compared to the sensitivity study with no fly-back in Section \ref{sec:sensitivityNoReturn}, the trade-offs between the efficiencies of the stages include the manoeuvrability of the scramjet accelerator, which dictates the fuel used during the return flight. This additional factor produces more complicated energy trade-offs, resulting in differing optimal trajectory shapes, with slightly different tradeoffs between the stages as the dynamic pressure is varied due to the differing nonlinearities in the engine efficiencies and vehicle performances at each dynamic pressure limit.
However, in all cases, increasing the maximum dynamic pressure improves the manoeuvring capabilities of the scramjet accelerator and increases the acceleration rate during ascent, which leads to a reduced flight time (-23.1s at 55kPa, and +90.6s at 45kPa), and less ground coverage (-46.1km at 55kPa, and +128.1km at 45kPa). Increasing the dynamic pressure limit generally allows for a higher scramjet accelerator-third stage release angle due to increased manoeuvrability. In all cases increasing the dynamic pressure limit increases the efficiency of the third stage rocket, also generally reducing the time that the third stage spends flying in greater than 5kPa dynamic pressure. }


\subsection{Case 13: Scramjet Accelerator Drag Sensitivity with Fly-Back}\label{sec:dragvar}


\begin{table}[ht]% updated 31/12/19
	\centering
	\begin{tabular}{l c c c c c c} 
		\hline \textbf{Trajectory Condition}   \qquad  $C_{d,2}$:
		&90\%
		&95\%
		&100\%
		&105\%
		&110\%
		& $\Delta/\Delta$/\%$C_{d,2}$
		\\
		\hline \textbf{Payload to Orbit (kg)}
		& \textbf{\PayloadToOrbitCdNinety}
		& \textbf{\PayloadToOrbitCdNinetyFive}
		& \textbf{\PayloadToOrbitCdStandard}
		& \textbf{\PayloadToOrbitCdOneHundredFive}
		& \textbf{\PayloadToOrbitCdOneHundredTen}
		&\textbf{-1.8}
		\\
		\textbf{Total $\eta_{exergy}$ (\%)}
		& \textbf{\totalExergyEffCdNinety}
		& \textbf{\totalExergyEffCdNinetyFive}
		& \textbf{\totalExergyEffCdStandard}
		& \textbf{\totalExergyEffCdOneHundredFive}
		& \textbf{\totalExergyEffCdOneHundredTen}
		& \textbf{-0.00018}
		\\
		\hline 
		\textbf{1$^{st}$ Stage $\eta_{exergy}$ (\%)}
		& \textbf{\firstExergyEffCdNinety}
		& \textbf{\firstExergyEffCdNinetyFive}
		& \textbf{\firstExergyEffCdStandard}
		& \textbf{\firstExergyEffCdOneHundredFive}
		& \textbf{\firstExergyEffCdOneHundredTen}
		& \textbf{-0.032}
		\\
		\textbf{Separation Alt, 1$\rightarrow$2 (km)}
		& \firstsecondSeparationAltCdNinety
		& \firstsecondSeparationAltCdNinetyFive
		& \firstsecondSeparationAltCdStandard
		& \firstsecondSeparationAltCdOneHundredFive
		& \firstsecondSeparationAltCdOneHundredTen
		& -
		\\
		\textbf{Separation v, 1$\rightarrow$2 (m/s)}
		& \firstsecondSeparationvCdNinety
		& \firstsecondSeparationvCdNinetyFive
		& \firstsecondSeparationvCdStandard
		& \firstsecondSeparationvCdOneHundredFive
		& \firstsecondSeparationvCdOneHundredTen
		&-4.1
		\\
		\textbf{Separation $\gamma$, 1$\rightarrow$2 (deg)}
		& \firstsecondSeparationgammaCdNinety
		& \firstsecondSeparationgammaCdNinetyFive
		& \firstsecondSeparationgammaCdStandard
		& \firstsecondSeparationgammaCdOneHundredFive
		& \firstsecondSeparationgammaCdOneHundredTen
		&0.23
		\\
		\hline 
		\textbf{2$^{nd}$ Stage $\eta_{exergy}$ (\%)}
		& \textbf{\secondExergyEffCdNinety}
		& \textbf{\secondExergyEffCdNinetyFive}
		& \textbf{\secondExergyEffCdStandard}
		& \textbf{\secondExergyEffCdOneHundredFive}
		& \textbf{\secondExergyEffCdOneHundredTen}
		& \textbf{-0.039}
		\\
		\textbf{Separation Alt, 2$\rightarrow$3 (km)}
		& \secondthirdSeparationAltCdNinety
		& \secondthirdSeparationAltCdNinetyFive
		& \secondthirdSeparationAltCdStandard
		& \secondthirdSeparationAltCdOneHundredFive
		& \secondthirdSeparationAltCdOneHundredTen
		&-0.06
		\\
		\textbf{Separation $v$, 2$\rightarrow$3 (m/s)}
		& \secondthirdSeparationvCdNinety
		& \secondthirdSeparationvCdNinetyFive
		& \secondthirdSeparationvCdStandard
		& \secondthirdSeparationvCdOneHundredFive
		& \secondthirdSeparationvCdOneHundredTen
		&-11.17
		\\
		\textbf{Separation $\gamma$, 2$\rightarrow$3 (deg)}
		& \secondthirdSeparationgammaCdNinety
		& \secondthirdSeparationgammaCdNinetyFive
		& \secondthirdSeparationgammaCdStandard
		& \secondthirdSeparationgammaCdOneHundredFive
		& \secondthirdSeparationgammaCdOneHundredTen
		&0.07
		\\
		\textbf{2$^{nd}$ Stage Flight Time (s)}
		& \secondFlightTimeCdNinety
		& \secondFlightTimeCdNinetyFive
		& \secondFlightTimeCdStandard
		& \secondFlightTimeCdOneHundredFive
		& \secondFlightTimeCdOneHundredTen
		&3.37
		\\
		\textbf{2$^{nd}$ Stage Distance Flown (km)}
		& \SecondDistCdNinety
		& \SecondDistCdNinetyFive
		& \SecondDistCdStandard
		& \SecondDistCdOneHundredFive
		& \SecondDistCdOneHundredTen
		& -
		\\
		\textbf{2$^{nd}$ Stage Return Fuel (kg)}
		& \returnFuelCdNinety
		& \returnFuelCdNinetyFive
		& \returnFuelCdStandard
		& \returnFuelCdOneHundredFive
		& \returnFuelCdOneHundredTen
		& -
		\\
		\textbf{2$^{nd}$ Stage Return Distance (km)}
		& \returnDistCdNinety
		& \returnDistCdNinetyFive
		& \returnDistCdStandard
		& \returnDistCdOneHundredFive
		& \returnDistCdOneHundredTen
		&-23.27
		\\
		\hline 
		\textbf{3$^{rd}$ Stage $\eta_{exergy}$ (\%)}
		& \textbf{\thirddExergyEffCdNinety}
		& \textbf{\thirddExergyEffCdNinetyFive}
		& \textbf{\thirddExergyEffCdStandard}
		& \textbf{\thirddExergyEffCdOneHundredFive}
		& \textbf{\thirddExergyEffCdOneHundredTen}
		& \textbf{-0.173}
		\\
		\textbf{3$^{rd}$ Stage $t$, $q >$ 5kpa (s)}
		& \thirdqOverFiveCdNinety
		& \thirdqOverFiveCdNinetyFive
		& \thirdqOverFiveCdStandard
		& \thirdqOverFiveCdOneHundredFive
		& \thirdqOverFiveCdOneHundredTen
		& -
		\\
		\textbf{3$^{rd}$ Stage Fuel Mass (kg)}
		& \thirdmFuelCdNinety
		& \thirdmFuelCdNinetyFive
		& \thirdmFuelCdStandard
		& \thirdmFuelCdOneHundredFive
		& \thirdmFuelCdOneHundredTen
		& 1.77
		\\
		\hline 
	\end{tabular} 
	\caption{Comparison of key trajectory parameters with variation in the drag of the scramjet accelerator, with fly-back (Case 13).}
	\label{tab:comparison41}
\end{table}

\noindent
The coefficient of drag is varied by $\pm$10\% to investigate the effect of variation in the aerodynamic design of the scramjet accelerator on the performance of the launch system, when the fly-back of the scramjet accelerator is included. Appendix \ref{sec:app_comparison41} presents trajectory comparison plots, and Table \ref{tab:comparison41} compares key parameters of each trajectory. 
Increasing the drag of the scramjet accelerator by 10\% decreases the payload-to-orbit by \textcolor{black}{-18.5}kg (\textcolor{black}{-14.0}\%), while decreasing the drag by 10\% increases the payload-to-orbit by \textcolor{black}{+17.3}kg (\textcolor{black}{+13.1}\%). 
The sensitivity to variations in the scramjet accelerator's aerodynamics is decreased slightly when compared to the sensitivity study with no fly-back, down to \textcolor{black}{-1.8} $\frac{\Delta kg}{\Delta\% C_{d}}$ compared to -2.0$\frac{\Delta kg}{\Delta\% C_{d}}$.
\textcolor{black}{This is due to the increased drag decreasing the total acceleration, which in turn decreases the ground distance necessary to cover during the fly-back, from \returnDistCdNinety km at 90\% drag, to \returnDistCdOneHundredTen km at 110\% drag, partially offsetting the detrimental effects of the increased drag on the performance of the launch system. }

The exergy efficiencies of all three stages are decreased significantly as the drag of the scramjet accelerator is increased. This decrease in efficiency is due to the increased drag losses of the first stage and scramjet accelerator, \WDoneCdOneHundredTen\% and \WDsecondCdOneHundredTen\% respectively at 110\%$C_D$, compared to \WDoneCdNinety\% and \WDsecondCdNinety\% respectively at 90\%$C_D$, and the increased availability losses (due to propulsive inefficiency) of the third stage when released from a lower speed, \PlossthreeCdOneHundredTen\% at 110\%$C_D$, compared to \PlossthreeCdNinety \% at 90\%$C_D$.
As was observed in the drag sensitivity study with no fly-back in Section \ref{sec:dragvariation}, the second-third stage separation angle shows an increase as the drag is increased, by \textcolor{black}{+0.9}$^\circ$ (\textcolor{black}{+7.1}\%) at 110\% drag, and decreasing by \textcolor{black}{-0.5}$^\circ$ (\textcolor{black}{-3.9}\%) at 90\% drag. In addition, the altitude of the second-third stage separation shows a clear trend, decreasing as the drag of the scramjet accelerator is increased, by \textcolor{black}{-0.98}km (-\textcolor{black}{2.29}\%) at 110\% drag, and increasing slightly as the drag is decreased, by \textcolor{black}{+0.27}km (\textcolor{black}{+0.63}\%) at 90\% drag.  
The release altitude and trajectory angle serve to initiate the first skip of the return trajectory in a generally consistent manner, so that the shape of the initial skip is similar across drag variation. This is due to the limiting nature of the initial skip, with the performance during this skip being mostly affected by the mass and lift of the scramjet accelerator. 
In all cases the angle of attack is reduced to 0$^\circ$ immediately during return to lessen the size of the initial skip, and is then raised to close to the maximum of 10$^\circ$ to prevent the dynamic pressure limit being exceeded. This consistency indicates that the initial skip of the return flight is driving the conditions at second-third stage release, and that it is primarily the control and structural limitations, rather than the aerodynamics of the scramjet accelerator, which determine the shape of this skip.


\subsection{Case 14: C-REST Specific Impulse Sensitivity with Fly-Back}


\begin{table}[ht] % updated 31/12/19
	\centering
	\begin{tabular}{l c c c c c c} 
		\hline \textbf{Trajectory Condition}   \qquad  $I_{SP,2}$:
		&90\%
		&95\%
		&100\%
		&105\%
		&110\%
		& $\Delta/\Delta$\%$I_{SP,2}$
		\\
		\hline \textbf{Payload to Orbit (kg)}
		& \textbf{\PayloadToOrbitIspNinety}
		& \textbf{\PayloadToOrbitIspNinetyFive}
		& \textbf{\PayloadToOrbitIspStandard}
		& \textbf{\PayloadToOrbitIspOneHundredFive}
		& \textbf{\PayloadToOrbitIspOneHundredTen}
		&\textbf{1.7}
		\\
		\textbf{Total $\eta_{exergy}$ (\%)}
		& \textbf{\totalExergyEffIspNinety}
		& \textbf{\totalExergyEffIspNinetyFive}
		& \textbf{\totalExergyEffIspStandard}
		& \textbf{\totalExergyEffIspOneHundredFive}
		& \textbf{\totalExergyEffIspOneHundredTen}
		& \textbf{0.00016}
		\\
		\hline 
		\textbf{1$^{st}$ Stage $\eta_{exergy}$ (\%)}
		& \textbf{\firstExergyEffIspNinety}
		& \textbf{\firstExergyEffIspNinetyFive}
		& \textbf{\firstExergyEffIspStandard}
		& \textbf{\firstExergyEffIspOneHundredFive}
		& \textbf{\firstExergyEffIspOneHundredTen}
		& -
		\\
		\textbf{Separation Alt, 1$\rightarrow$2 (km)}
		& \firstsecondSeparationAltIspNinety
		& \firstsecondSeparationAltIspNinetyFive
		& \firstsecondSeparationAltIspStandard
		& \firstsecondSeparationAltIspOneHundredFive
		& \firstsecondSeparationAltIspOneHundredTen
		& -
		\\
		\textbf{Separation v, 1$\rightarrow$2 (m/s)}
		& \firstsecondSeparationvIspNinety
		& \firstsecondSeparationvIspNinetyFive
		& \firstsecondSeparationvIspStandard
		& \firstsecondSeparationvIspOneHundredFive
		& \firstsecondSeparationvIspOneHundredTen
		& -
		\\
		\textbf{Separation $\gamma$, 1$\rightarrow$2 (deg)}
		& \firstsecondSeparationgammaIspNinety
		& \firstsecondSeparationgammaIspNinetyFive
		& \firstsecondSeparationgammaIspStandard
		& \firstsecondSeparationgammaIspOneHundredFive
		& \firstsecondSeparationgammaIspOneHundredTen
		& -
		\\
		\hline 
		\textbf{2$^{nd}$ Stage $\eta_{exergy}$ (\%)}
		& \textbf{\secondExergyEffIspNinety}
		& \textbf{\secondExergyEffIspNinetyFive}
		& \textbf{\secondExergyEffIspStandard}
		& \textbf{\secondExergyEffIspOneHundredFive}
		& \textbf{\secondExergyEffIspOneHundredTen}
		& \textbf{0.049}
		\\
		\textbf{Separation Alt, 2$\rightarrow$3 (km)}
		& \secondthirdSeparationAltIspNinety
		& \secondthirdSeparationAltIspNinetyFive
		& \secondthirdSeparationAltIspStandard
		& \secondthirdSeparationAltIspOneHundredFive
		& \secondthirdSeparationAltIspOneHundredTen
		& -
		\\
		\textbf{Separation $v$, 2$\rightarrow$3 (m/s)}
		& \secondthirdSeparationvIspNinety
		& \secondthirdSeparationvIspNinetyFive
		& \secondthirdSeparationvIspStandard
		& \secondthirdSeparationvIspOneHundredFive
		& \secondthirdSeparationvIspOneHundredTen
		&10.83
		\\
		\textbf{Separation $\gamma$, 2$\rightarrow$3 (deg)}
		& \secondthirdSeparationgammaIspNinety
		& \secondthirdSeparationgammaIspNinetyFive
		& \secondthirdSeparationgammaIspStandard
		& \secondthirdSeparationgammaIspOneHundredFive
		& \secondthirdSeparationgammaIspOneHundredTen
		&-0.06
		\\
		\textbf{2$^{nd}$ Stage Flight Time (s)}
		& \secondFlightTimeIspNinety
		& \secondFlightTimeIspNinetyFive
		& \secondFlightTimeIspStandard
		& \secondFlightTimeIspOneHundredFive
		& \secondFlightTimeIspOneHundredTen
		& -
		\\
		\textbf{2$^{nd}$ Stage Distance Flown (km)}
		& \SecondDistIspNinety
		& \SecondDistIspNinetyFive
		& \SecondDistIspStandard
		& \SecondDistIspOneHundredFive
		& \SecondDistIspOneHundredTen
		&4.04
		\\
		\textbf{2$^{nd}$ Stage Return Fuel (kg)}
		& \returnFuelIspNinety
		& \returnFuelIspNinetyFive
		& \returnFuelIspStandard
		& \returnFuelIspOneHundredFive
		& \returnFuelIspOneHundredTen
		&-1.39
		\\
		\textbf{2$^{nd}$ Stage Return Distance (km)}
		& \returnDistIspNinety
		& \returnDistIspNinetyFive
		& \returnDistIspStandard
		& \returnDistIspOneHundredFive
		& \returnDistIspOneHundredTen
		&11.3
		\\
		\hline 
		\textbf{3$^{rd}$ Stage $\eta_{exergy}$ (\%)}
		& \textbf{\thirddExergyEffIspNinety}
		& \textbf{\thirddExergyEffIspNinetyFive}
		& \textbf{\thirddExergyEffIspStandard}
		& \textbf{\thirddExergyEffIspOneHundredFive}
		& \textbf{\thirddExergyEffIspOneHundredTen}
		& \textbf{0.164}
		\\
		\textbf{3$^{rd}$ Stage $t$, $q >$ 5kpa (s)}
		& \thirdqOverFiveIspNinety
		& \thirdqOverFiveIspNinetyFive
		& \thirdqOverFiveIspStandard
		& \thirdqOverFiveIspOneHundredFive
		& \thirdqOverFiveIspOneHundredTen
		& -
		\\
		\textbf{3$^{rd}$ Stage Fuel Mass (kg)}
		& \thirdmFuelIspNinety
		& \thirdmFuelIspNinetyFive
		& \thirdmFuelIspStandard
		& \thirdmFuelIspOneHundredFive
		& \thirdmFuelIspOneHundredTen
		&-1.67
		\\
		\hline 
	\end{tabular} 
	
	\caption{Comparison of key trajectory parameters with variation in the specific impulse of the C-REST engines, with fly-back (Case 14).}
	\label{tab:comparison31}
\end{table}

The specific impulse of the scramjet accelerator is varied by $\pm10\%$ in order to assess the sensitivity of the optimised trajectory to the performance of the scramjet engines. 
Key parameters of the trajectories are summarised in Table \ref{tab:comparison31}, and comparison plots are shown in Appendix \ref{sec:app_comparison31}.
Raising the specific impulse of the C-REST engines increases the payload-to-orbit, by \textcolor{black}{+16.2}kg (\textcolor{black}{+12.3}\%) at 110\% $I_{SP}$, while lowering the specific impulse decreases the payload-to-orbit, by \textcolor{black}{-17.5}kg (\textcolor{black}{-13.2}\%) at 90\% $I_{SP}$. 
This produces a general trend in the payload-to-orbit of 1.7$\frac{\Delta kg}{\Delta \%I_{SP} }$, lower than the trend of 2.2$\frac{\Delta kg}{\Delta \%I_{SP} }$ observed in the sensitivity study without fly-back, in Section \ref{sec:ispsensitivitynoflyback}. 
This lowered sensitivity in the payload-to-orbit is due to a correspondingly lowered sensitivity in the exergy efficiency of the scramjet accelerator, of \textcolor{black}{0.049}$\frac{\Delta \% \eta}{\Delta \%I_{SP} }$, compared to \textcolor{black}{0.065}$\frac{\Delta \% \eta}{\Delta \%I_{SP} }$ in the sensitivity study without fly-back. This lowered sensitivity is due to the modified $I_{SP}$ having no effect on the performance of the scramjet accelerator during the unpowered portions of the fly-back trajectory, which serve to offset the overall variation in exergy efficiency. 


Similarly to the specific impulse sensitivity study without fly-back conducted in Section \ref{sec:ispsensitivitynoflyback}, the first-second separation conditions, as well as the exergy efficiency of the first stage, exhibit no clear trends. Following first-second separation, the general shape of the trajectory path of the scramjet accelerator does not change significantly with specific impulse variation, including the the pull-up altitude. As with the optimised trajectories with no fly-back, increasing the specific impulse of the scramjet engines by 10\% increases the speed at separation (by \textcolor{black}{+105}m/s, \textcolor{black}{+4.2}\%) and decreases the trajectory angle (by \textcolor{black}{-0.6}$^\circ$, \textcolor{black}{-4.7}\%), while decreasing the specific impulse of the scramjet engines by 10\% decreases the speed at second-third stage separation (by \textcolor{black}{-113}m/s, \textcolor{black}{-4.5}\%), and increases the trajectory angle (by \textcolor{black}{0.7}$^\circ$, \textcolor{black}{+5.5}\%).
The exergy efficiency of the third stage rocket increases as the exergy efficiency of the scramjet accelerator increases. This is in line with the trend which has been observed in all previous cases, that the increased separation speed increases the propulsive efficiency of the third stage, in turn increasing its overall efficiency. 






\subsection{Case 15: Scramjet Accelerator Mass Sensitivity with Fly-Back}\label{sec:m2var}
\noindent
The mass of the scramjet accelerator is varied by \textcolor{black}{$\pm$10}\% to investigate the sensitivity of the launch system performance to the structural mass of the second stage, with the inclusion of scramjet accelerator fly-back.
Table \ref{tab:comparison101} details key parameters of each trajectory, and Appendix \ref{sec:app_comparison101} shows comparison plots.
Varying the structural mass of the scramjet accelerator yields a change in maximum payload-mass to orbit of +\textcolor{black}{12.6}kg (+\textcolor{black}{9.5}\%) at 90\% mass, and -\textcolor{black}{14.9}kg (-\textcolor{black}{11.3}\%) at 110\% mass. The sensitivities of the launch system to the mass of the scramjet accelerator are increased when fly-back is included, due to the detrimental performance affects of the additional scramjet accelerator mass during fly-back, that must be compensated for. 

\begin{table}[ht] % UPDATED 11/1/20
	\centering
	\begin{tabular}{l c c c c c c} 
		\hline \textbf{Trajectory Condition}   \qquad  $m_{2}$:
		&90\%
		&95\%
		&100\%
		&105\%
		&110\%
		& $\Delta/\Delta$\%$m_{2}$
		\\
		\hline \textbf{Payload to Orbit (kg)}
		& \textbf{\PayloadToOrbitmSPARTANNinety}
		& \textbf{\PayloadToOrbitmSPARTANNinetyFive}
		& \textbf{\PayloadToOrbitmSPARTANStandard}
		& \textbf{\PayloadToOrbitmSPARTANOneHundredFive}
		& \textbf{\PayloadToOrbitmSPARTANOneHundredTen}
		&\textbf{-1.4}
		\\
		\textbf{Total $\eta_{exergy}$ (\%)}
		& \textbf{\totalExergyEffmSPARTANNinety}
		& \textbf{\totalExergyEffmSPARTANNinetyFive}
		& \textbf{\totalExergyEffmSPARTANStandard}
		& \textbf{\totalExergyEffmSPARTANOneHundredFive}
		& \textbf{\totalExergyEffmSPARTANOneHundredTen}
		& \textbf{-0.00015}
		\\
		\hline 
		\textbf{1$^{st}$ Stage $\eta_{exergy}$ (\%)}
		& \textbf{\firstExergyEffmSPARTANNinety}
		& \textbf{\firstExergyEffmSPARTANNinetyFive}
		& \textbf{\firstExergyEffmSPARTANStandard}
		& \textbf{\firstExergyEffmSPARTANOneHundredFive}
		& \textbf{\firstExergyEffmSPARTANOneHundredTen}
		& \textbf{-0.017}
		\\
		\textbf{Separation Alt, 1$\rightarrow$2 (km)}
		& \firstsecondSeparationAltmSPARTANNinety
		& \firstsecondSeparationAltmSPARTANNinetyFive
		& \firstsecondSeparationAltmSPARTANStandard
		& \firstsecondSeparationAltmSPARTANOneHundredFive
		& \firstsecondSeparationAltmSPARTANOneHundredTen
		&-0.05
		\\
		\textbf{Separation v, 1$\rightarrow$2 (m/s)}
		& \firstsecondSeparationvmSPARTANNinety
		& \firstsecondSeparationvmSPARTANNinetyFive
		& \firstsecondSeparationvmSPARTANStandard
		& \firstsecondSeparationvmSPARTANOneHundredFive
		& \firstsecondSeparationvmSPARTANOneHundredTen
		&-6.64
		\\
		\textbf{Separation $\gamma$, 1$\rightarrow$2 (deg)}
		& \firstsecondSeparationgammamSPARTANNinety
		& \firstsecondSeparationgammamSPARTANNinetyFive
		& \firstsecondSeparationgammamSPARTANStandard
		& \firstsecondSeparationgammamSPARTANOneHundredFive
		& \firstsecondSeparationgammamSPARTANOneHundredTen
		&0.11
		\\
		\hline 
		\textbf{2$^{nd}$ Stage $\eta_{exergy}$ (\%)}
		& \textbf{\secondExergyEffmSPARTANNinety}
		& \textbf{\secondExergyEffmSPARTANNinetyFive}
		& \textbf{\secondExergyEffmSPARTANStandard}
		& \textbf{\secondExergyEffmSPARTANOneHundredFive}
		& \textbf{\secondExergyEffmSPARTANOneHundredTen}
		& \textbf{-0.014}
		\\
		\textbf{Separation Alt, 2$\rightarrow$3 (km)}
		& \secondthirdSeparationAltmSPARTANNinety
		& \secondthirdSeparationAltmSPARTANNinetyFive
		& \secondthirdSeparationAltmSPARTANStandard
		& \secondthirdSeparationAltmSPARTANOneHundredFive
		& \secondthirdSeparationAltmSPARTANOneHundredTen
		& -
		\\
		\textbf{Separation $v$, 2$\rightarrow$3 (m/s)}
		& \secondthirdSeparationvmSPARTANNinety
		& \secondthirdSeparationvmSPARTANNinetyFive
		& \secondthirdSeparationvmSPARTANStandard
		& \secondthirdSeparationvmSPARTANOneHundredFive
		& \secondthirdSeparationvmSPARTANOneHundredTen
		&-7.26
		\\
		\textbf{Separation $\gamma$, 2$\rightarrow$3 (deg)}
		& \secondthirdSeparationgammamSPARTANNinety
		& \secondthirdSeparationgammamSPARTANNinetyFive
		& \secondthirdSeparationgammamSPARTANStandard
		& \secondthirdSeparationgammamSPARTANOneHundredFive
		& \secondthirdSeparationgammamSPARTANOneHundredTen
		& -
		\\
		\textbf{2$^{nd}$ Stage Flight Time (s)}
		& \secondFlightTimemSPARTANNinety
		& \secondFlightTimemSPARTANNinetyFive
		& \secondFlightTimemSPARTANStandard
		& \secondFlightTimemSPARTANOneHundredFive
		& \secondFlightTimemSPARTANOneHundredTen
		&2.55
		\\
		\textbf{2$^{nd}$ Stage Distance Flown (km)}
		& \SecondDistmSPARTANNinety
		& \SecondDistmSPARTANNinetyFive
		& \SecondDistmSPARTANStandard
		& \SecondDistmSPARTANOneHundredFive
		& \SecondDistmSPARTANOneHundredTen
		& -
		\\
		\textbf{2$^{nd}$ Stage Return Fuel (kg)}
		& \returnFuelmSPARTANNinety
		& \returnFuelmSPARTANNinetyFive
		& \returnFuelmSPARTANStandard
		& \returnFuelmSPARTANOneHundredFive
		& \returnFuelmSPARTANOneHundredTen
		& -
		\\
		\textbf{2$^{nd}$ Stage Return Distance (km)}
		& \returnDistmSPARTANNinety
		& \returnDistmSPARTANNinetyFive
		& \returnDistmSPARTANStandard
		& \returnDistmSPARTANOneHundredFive
		& \returnDistmSPARTANOneHundredTen
		&-18.12
		\\
		\hline 
		\textbf{3$^{rd}$ Stage $\eta_{exergy}$ (\%)}
		& \textbf{\thirddExergyEffmSPARTANNinety}
		& \textbf{\thirddExergyEffmSPARTANNinetyFive}
		& \textbf{\thirddExergyEffmSPARTANStandard}
		& \textbf{\thirddExergyEffmSPARTANOneHundredFive}
		& \textbf{\thirddExergyEffmSPARTANOneHundredTen}
		& \textbf{-0.137}
		\\
		\textbf{3$^{rd}$ Stage $t$, $q >$ 5kpa (s)}
		& \thirdqOverFivemSPARTANNinety
		& \thirdqOverFivemSPARTANNinetyFive
		& \thirdqOverFivemSPARTANStandard
		& \thirdqOverFivemSPARTANOneHundredFive
		& \thirdqOverFivemSPARTANOneHundredTen
		& -
		\\
		\textbf{3$^{rd}$ Stage Fuel Mass (kg)}
		& \thirdmFuelmSPARTANNinety
		& \thirdmFuelmSPARTANNinetyFive
		& \thirdmFuelmSPARTANStandard
		& \thirdmFuelmSPARTANOneHundredFive
		& \thirdmFuelmSPARTANOneHundredTen
		&1.39
		\\
		\hline 
	\end{tabular} 
	\caption{Comparison of key trajectory parameters with variation in the structural mass of the scramjet accelerator, with fly-back (Case 15).}
	\label{tab:comparison101}
\end{table}

\textcolor{black}{
As observed in Section \ref{sec:SpartanMassnoreturn}, increasing the structural mass of the scramjet accelerator generally decreases the altitude of first stage-scramjet accelerator separation, and decreases the speed and trajectory angle. 
In addition, increasing the structural mass of the scramjet accelerator decreases the exergy efficiency of the first stage rocket, due to the first stage rocket not accelerating as quickly as the mass of the scramjet accelerator is increased, causing the propulsive efficiency of the first stage to decrease (illustrated by Equation \ref{eq:rocketeff}).}
 As was observed in Section \ref{sec:SpartanMassnoreturn}, the lower speed of first-second stage separation means that when the scramjet accelerator mass is increased, the speed range over which the scramjet accelerator is accelerating is lower. This is beneficial for the specific impulse of the C-REST engines, which exhibit higher $I_{SP}$ at lower velocities. For this reason, when the scramjet accelerator mass is increased, the net specific impulse of the scramjet accelerator stays high for longer in the main segment of the trajectory, above 500s $I_{SP}$ for \textcolor{black}{441}s of its trajectory at 110\% mass, compared to \textcolor{black}{265}s above 500s $I_{SP}$ at 90\% mass. 
However, the higher scramjet accelerator mass decreases the overall acceleration of the scramjet accelerator, in turn decreasing the efficiency of the third stage due to decreased propulsive efficiency. As the mass of the scramjet accelerator increases, the height of the initial skip during fly-back is reduced, because of the additional energy required to slow the rate of descent of the scramjet accelerator as it's mass is increased. 



\subsection{Case 16: Scramjet Accelerator Fuel Mass Sensitivity with Fly-Back}

\begin{table}[ht]%updated 31/12/19
	\centering
\begin{tabular}{l c c c c c c} 
	\hline \textbf{Trajectory Condition}   \qquad  $m_{f,2}$:
	&90\%
	&95\%
	&100\%
	&105\%
	&110\%
	& $\Delta/\Delta$\%$m_{F,2}$
	\\
	\hline \textbf{Payload to Orbit (kg)}
	& \textbf{\PayloadToOrbitmFuelNinety}
	& \textbf{\PayloadToOrbitmFuelNinetyFive}
	& \textbf{\PayloadToOrbitmFuelStandard}
	& \textbf{\PayloadToOrbitmFuelOneHundredFive}
	& \textbf{\PayloadToOrbitmFuelOneHundredTen}
	&\textbf{0.6}
	\\
	\textbf{Total $\eta_{exergy}$ (\%)}
	& \textbf{\totalExergyEffmFuelNinety}
	& \textbf{\totalExergyEffmFuelNinetyFive}
	& \textbf{\totalExergyEffmFuelStandard}
	& \textbf{\totalExergyEffmFuelOneHundredFive}
	& \textbf{\totalExergyEffmFuelOneHundredTen}
	& -
	\\
	\hline 
	\textbf{1$^{st}$ Stage $\eta_{exergy}$ (\%)}
	& \textbf{\firstExergyEffmFuelNinety}
	& \textbf{\firstExergyEffmFuelNinetyFive}
	& \textbf{\firstExergyEffmFuelStandard}
	& \textbf{\firstExergyEffmFuelOneHundredFive}
	& \textbf{\firstExergyEffmFuelOneHundredTen}
	& \textbf{-0.005}
	\\
	\textbf{Separation Alt, 1$\rightarrow$2 (km)}
	& \firstsecondSeparationAltmFuelNinety
	& \firstsecondSeparationAltmFuelNinetyFive
	& \firstsecondSeparationAltmFuelStandard
	& \firstsecondSeparationAltmFuelOneHundredFive
	& \firstsecondSeparationAltmFuelOneHundredTen
	&-0.02
	\\
	\textbf{Separation v, 1$\rightarrow$2 (m/s)}
	& \firstsecondSeparationvmFuelNinety
	& \firstsecondSeparationvmFuelNinetyFive
	& \firstsecondSeparationvmFuelStandard
	& \firstsecondSeparationvmFuelOneHundredFive
	& \firstsecondSeparationvmFuelOneHundredTen
	&-2.01
	\\
	\textbf{Separation $\gamma$, 1$\rightarrow$2 (deg)}
	& \firstsecondSeparationgammamFuelNinety
	& \firstsecondSeparationgammamFuelNinetyFive
	& \firstsecondSeparationgammamFuelStandard
	& \firstsecondSeparationgammamFuelOneHundredFive
	& \firstsecondSeparationgammamFuelOneHundredTen
	& -
	\\
	\hline 
	\textbf{2$^{nd}$ Stage $\eta_{exergy}$ (\%)}
	& \textbf{\secondExergyEffmFuelNinety}
	& \textbf{\secondExergyEffmFuelNinetyFive}
	& \textbf{\secondExergyEffmFuelStandard}
	& \textbf{\secondExergyEffmFuelOneHundredFive}
	& \textbf{\secondExergyEffmFuelOneHundredTen}
	& \textbf{-0.013}
	\\
	\textbf{Separation Alt, 2$\rightarrow$3 (km)}
	& \secondthirdSeparationAltmFuelNinety
	& \secondthirdSeparationAltmFuelNinetyFive
	& \secondthirdSeparationAltmFuelStandard
	& \secondthirdSeparationAltmFuelOneHundredFive
	& \secondthirdSeparationAltmFuelOneHundredTen
	& -
	\\
	\textbf{Separation $v$, 2$\rightarrow$3 (m/s)}
	& \secondthirdSeparationvmFuelNinety
	& \secondthirdSeparationvmFuelNinetyFive
	& \secondthirdSeparationvmFuelStandard
	& \secondthirdSeparationvmFuelOneHundredFive
	& \secondthirdSeparationvmFuelOneHundredTen
	&4.14
	\\
	\textbf{Separation $\gamma$, 2$\rightarrow$3 (deg)}
	& \secondthirdSeparationgammamFuelNinety
	& \secondthirdSeparationgammamFuelNinetyFive
	& \secondthirdSeparationgammamFuelStandard
	& \secondthirdSeparationgammamFuelOneHundredFive
	& \secondthirdSeparationgammamFuelOneHundredTen
	&-0.03
	\\
	\textbf{2$^{nd}$ Stage Flight Time (s)}
	& \secondFlightTimemFuelNinety
	& \secondFlightTimemFuelNinetyFive
	& \secondFlightTimemFuelStandard
	& \secondFlightTimemFuelOneHundredFive
	& \secondFlightTimemFuelOneHundredTen
	&4.37
	\\
	\textbf{2$^{nd}$ Stage Distance Flown (km)}
	& \SecondDistmFuelNinety
	& \SecondDistmFuelNinetyFive
	& \SecondDistmFuelStandard
	& \SecondDistmFuelOneHundredFive
	& \SecondDistmFuelOneHundredTen
	&9.89
	\\
	\textbf{2$^{nd}$ Stage Return Fuel (kg)}
	& \returnFuelmFuelNinety
	& \returnFuelmFuelNinetyFive
	& \returnFuelmFuelStandard
	& \returnFuelmFuelOneHundredFive
	& \returnFuelmFuelOneHundredTen
	& -
	\\
	\textbf{2$^{nd}$ Stage Return Distance (km)}
	& \returnDistmFuelNinety
	& \returnDistmFuelNinetyFive
	& \returnDistmFuelStandard
	& \returnDistmFuelOneHundredFive
	& \returnDistmFuelOneHundredTen
	& -
	\\
	\hline 
	\textbf{3$^{rd}$ Stage $\eta_{exergy}$ (\%)}
	& \textbf{\thirddExergyEffmFuelNinety}
	& \textbf{\thirddExergyEffmFuelNinetyFive}
	& \textbf{\thirddExergyEffmFuelStandard}
	& \textbf{\thirddExergyEffmFuelOneHundredFive}
	& \textbf{\thirddExergyEffmFuelOneHundredTen}
	& \textbf{0.061}
	\\
	\textbf{3$^{rd}$ Stage $t$, $q >$ 5kpa (s)}
	& \thirdqOverFivemFuelNinety
	& \thirdqOverFivemFuelNinetyFive
	& \thirdqOverFivemFuelStandard
	& \thirdqOverFivemFuelOneHundredFive
	& \thirdqOverFivemFuelOneHundredTen
	&0.05
	\\
	\textbf{3$^{rd}$ Stage Fuel Mass (kg)}
	& \thirdmFuelmFuelNinety
	& \thirdmFuelmFuelNinetyFive
	& \thirdmFuelmFuelStandard
	& \thirdmFuelmFuelOneHundredFive
	& \thirdmFuelmFuelOneHundredTen
	&-0.63
	\\
	\hline 
\end{tabular} 
\caption{Comparison of key trajectory parameters with variation in the fuel mass of the scramjet accelerator, with fly-back (Case 16).}
\label{tab:comparison111}
\end{table}
\noindent	
The fuel mass of the scramjet accelerator is varied by $\pm$10\%, to investigate the sensitivity of the performance of the launch system to variations in the size of the fuel tanks within the scramjet accelerator. 
Appendix \ref{sec:app_comparison111} shows plots comparing each trajectory, and Table \ref{tab:comparison111} details comparisons of key trajectory parameters. 
When the fuel mass within the scramjet accelerator is increased by 10\%, the payload to orbit increases by \textcolor{black}{+6.4}kg (\textcolor{black}{+4.8}\%) and when the fuel mass is decreased by 10\%, the payload mass reduces by \textcolor{black}{-6.1}g (\textcolor{black}{-4.6}\%). The magnitude of the payload-to-orbit sensitivity is slightly reduced when compared to the sensitivity observed without fly-back, in Section \ref{sec:fuelmassNoReturn}, due to the beneficial effects of adding fuel being slightly offset by the necessity to use fuel for fly-back. 


\textcolor{black}{
The first stage shows similar trends in its trajectory when the fuel mass of the scramjet accelerator is varied to the trends observed with no fly-back in Section \ref{sec:fuelmassNoReturn}.}
As in Section \ref{sec:fuelmassNoReturn}, increasing the fuel mass of the scramjet accelerator decreases the exergy efficiency of the scramjet accelerator, by \textcolor{black}{-0.112}\%$\eta$ (\textcolor{black}{-2.98}\%) at 110\% fuel mass, and decreasing the fuel mass of the scramjet accelerator increases its exergy efficiency, by \textcolor{black}{+0.159}\%$\eta$ (\textcolor{black}{+4.23}\%) at 90\% fuel mass. Once again, the overall exergy efficiency of the system shows no distinct trend. 
As in Section \ref{sec:fuelmassNoReturn}, this is due to the increased period of acceleration causing the specific impulse of the C-REST engines to decrease. However, the overall energy availability is increased by the additional fuel mass, resulting in more overall exergy. This results in the overall energy imparted upon the third stage by the scramjet accelerator increasing, from \secondExergythirdStagemFuelNinety GJ at 90\% $m_{f}$, to \secondExergythirdStagemFuelOneHundredTen GJ at 110\% $m_{f}$, in turn increasing the payload-to-orbit.




\subsection{Case 17: Third Stage Mass Sensitivity with Fly-Back}

\begin{table}[ht]% updated 11/1/20
	\centering
	\begin{tabular}{l c c c c c c} 
		\hline \textbf{Trajectory Condition}   \qquad  $m_{3}$:
		&90\%
		&95\%
		&100\%
		&105\%
		&110\%
		& $\Delta/\Delta$\%$m_{3}$
		\\
		\hline \textbf{Payload to Orbit (kg)}
		& \textbf{\PayloadToOrbitmThreeNinety}
		& \textbf{\PayloadToOrbitmThreeNinetyFive}
		& \textbf{\PayloadToOrbitmThreeStandard}
		& \textbf{\PayloadToOrbitmThreeOneHundredFive}
		& \textbf{\PayloadToOrbitmThreeOneHundredTen}
		& -
		\\
		\textbf{Total $\eta_{exergy}$ (\%)}
		& \textbf{\totalExergyEffmThreeNinety}
		& \textbf{\totalExergyEffmThreeNinetyFive}
		& \textbf{\totalExergyEffmThreeStandard}
		& \textbf{\totalExergyEffmThreeOneHundredFive}
		& \textbf{\totalExergyEffmThreeOneHundredTen}
		& -
		\\
		\hline 
		\textbf{1$^{st}$ Stage $\eta_{exergy}$ (\%)}
		& \textbf{\firstExergyEffmThreeNinety}
		& \textbf{\firstExergyEffmThreeNinetyFive}
		& \textbf{\firstExergyEffmThreeStandard}
		& \textbf{\firstExergyEffmThreeOneHundredFive}
		& \textbf{\firstExergyEffmThreeOneHundredTen}
		& \textbf{-0.012}
		\\
		\textbf{Separation Alt, 1$\rightarrow$2 (km)}
		& \firstsecondSeparationAltmThreeNinety
		& \firstsecondSeparationAltmThreeNinetyFive
		& \firstsecondSeparationAltmThreeStandard
		& \firstsecondSeparationAltmThreeOneHundredFive
		& \firstsecondSeparationAltmThreeOneHundredTen
		&-0.04
		\\
		\textbf{Separation v, 1$\rightarrow$2 (m/s)}
		& \firstsecondSeparationvmThreeNinety
		& \firstsecondSeparationvmThreeNinetyFive
		& \firstsecondSeparationvmThreeStandard
		& \firstsecondSeparationvmThreeOneHundredFive
		& \firstsecondSeparationvmThreeOneHundredTen
		&-4.48
		\\
		\textbf{Separation $\gamma$, 1$\rightarrow$2 (deg)}
		& \firstsecondSeparationgammamThreeNinety
		& \firstsecondSeparationgammamThreeNinetyFive
		& \firstsecondSeparationgammamThreeStandard
		& \firstsecondSeparationgammamThreeOneHundredFive
		& \firstsecondSeparationgammamThreeOneHundredTen
		&0.05
		\\
		\hline 
		\textbf{2$^{nd}$ Stage $\eta_{exergy}$ (\%)}
		& \textbf{\secondExergyEffmThreeNinety}
		& \textbf{\secondExergyEffmThreeNinetyFive}
		& \textbf{\secondExergyEffmThreeStandard}
		& \textbf{\secondExergyEffmThreeOneHundredFive}
		& \textbf{\secondExergyEffmThreeOneHundredTen}
		& \textbf{0.062}
		\\
		\textbf{Separation Alt, 2$\rightarrow$3 (km)}
		& \secondthirdSeparationAltmThreeNinety
		& \secondthirdSeparationAltmThreeNinetyFive
		& \secondthirdSeparationAltmThreeStandard
		& \secondthirdSeparationAltmThreeOneHundredFive
		& \secondthirdSeparationAltmThreeOneHundredTen
		& -
		\\
		\textbf{Separation $v$, 2$\rightarrow$3 (m/s)}
		& \secondthirdSeparationvmThreeNinety
		& \secondthirdSeparationvmThreeNinetyFive
		& \secondthirdSeparationvmThreeStandard
		& \secondthirdSeparationvmThreeOneHundredFive
		& \secondthirdSeparationvmThreeOneHundredTen
		& -
		\\
		\textbf{Separation $\gamma$, 2$\rightarrow$3 (deg)}
		& \secondthirdSeparationgammamThreeNinety
		& \secondthirdSeparationgammamThreeNinetyFive
		& \secondthirdSeparationgammamThreeStandard
		& \secondthirdSeparationgammamThreeOneHundredFive
		& \secondthirdSeparationgammamThreeOneHundredTen
		& -
		\\
		\textbf{2$^{nd}$ Stage Flight Time (s)}
		& \secondFlightTimemThreeNinety
		& \secondFlightTimemThreeNinetyFive
		& \secondFlightTimemThreeStandard
		& \secondFlightTimemThreeOneHundredFive
		& \secondFlightTimemThreeOneHundredTen
		& -
		\\
		\textbf{2$^{nd}$ Stage Distance Flown (km)}
		& \SecondDistmThreeNinety
		& \SecondDistmThreeNinetyFive
		& \SecondDistmThreeStandard
		& \SecondDistmThreeOneHundredFive
		& \SecondDistmThreeOneHundredTen
		& -
		\\
		\textbf{2$^{nd}$ Stage Return Fuel (kg)}
		& \returnFuelmThreeNinety
		& \returnFuelmThreeNinetyFive
		& \returnFuelmThreeStandard
		& \returnFuelmThreeOneHundredFive
		& \returnFuelmThreeOneHundredTen
		& -
		\\
		\textbf{2$^{nd}$ Stage Return Distance (km)}
		& \returnDistmThreeNinety
		& \returnDistmThreeNinetyFive
		& \returnDistmThreeStandard
		& \returnDistmThreeOneHundredFive
		& \returnDistmThreeOneHundredTen
		&-16.57
		\\
		\hline 
		\textbf{3$^{rd}$ Stage $\eta_{exergy}$ (\%)}
		& \textbf{\thirddExergyEffmThreeNinety}
		& \textbf{\thirddExergyEffmThreeNinetyFive}
		& \textbf{\thirddExergyEffmThreeStandard}
		& \textbf{\thirddExergyEffmThreeOneHundredFive}
		& \textbf{\thirddExergyEffmThreeOneHundredTen}
		& \textbf{-0.271}
		\\
		\textbf{3$^{rd}$ Stage $t$, $q >$ 5kpa (s)}
		& \thirdqOverFivemThreeNinety
		& \thirdqOverFivemThreeNinetyFive
		& \thirdqOverFivemThreeStandard
		& \thirdqOverFivemThreeOneHundredFive
		& \thirdqOverFivemThreeOneHundredTen
		& -
		\\
		\textbf{3$^{rd}$ Stage Fuel Mass (kg)}
		& \thirdmFuelmThreeNinety
		& \thirdmFuelmThreeNinetyFive
		& \thirdmFuelmThreeStandard
		& \thirdmFuelmThreeOneHundredFive
		& \thirdmFuelmThreeOneHundredTen
		&31.94
		\\
		\hline 
	\end{tabular} 
	\caption{Comparison of key trajectory parameters with variation in the mass of the third stage, with fly-back (Case 17).}
	\label{tab:comparison81}
\end{table}
\noindent
The mass of the third stage rocket is varied by $\pm$10\%, to investigate the effects of the internal mass density of the third stage rocket when the fly-back of the scramjet accelerator is included. Table \ref{tab:comparison81} details key trajectory parameters, and Appendix \ref{sec:app_comparison81} shows trajectory comparison plots. As in Section \ref{sec:m3noReturn}, the varied mass is a combination of the fuel and structural mass of the third stage, and payload mass, representing the density of the components within the third stage. As previously, the heat shield mass is not varied, the structural mass held at 9\% of the total, non-heat shield mass, and the remaining mass variation is a combination of fuel and payload mass.

\textcolor{black}{
As was observed in Section \ref{sec:m3noReturn}, there is a non-linear payload mass-to-orbit trend with variations in the third stage mass. Once again, when the third stage mass is reduced, the payload mass is reduced, by -12.2kg (-9.2\%), and when the third stage mass is increased, the payload mass is also reduced, by a larger -45.2kg (-34.2\%). Once again, the pull-up of the scramjet accelerator is relatively consistent when the third stage mass is reduced, but is lowered significantly when the third stage mass is increased, as the second stage must compensate for the additional mass of the third stage. The magnitude of the response is very similar to the sensitivities with no fly-back, indicating that the fly-back does not significantly effect the sensitivity of the launch system performance with variations in the mass of the third stage rocket. 
}


\subsection{Case 18: Third Stage Specific Impulse Sensitivity with Fly-Back}


\begin{table}[ht]%11/1/20

	\centering
	\begin{tabular}{l c c c c c c} 
		\hline \textbf{Trajectory Condition}   \qquad  $I_{SP,3}$:
		&90\%
		&95\%
		&100\%
		&105\%
		&110\%
		& $\Delta/\Delta$\%$I_{SP,3}$
		\\
		\hline \textbf{Payload to Orbit (kg)}
		& \textbf{\PayloadToOrbitISPThreeNinety}
		& \textbf{\PayloadToOrbitISPThreeNinetyFive}
		& \textbf{\PayloadToOrbitISPThreeStandard}
		& \textbf{\PayloadToOrbitISPThreeOneHundredFive}
		& \textbf{\PayloadToOrbitISPThreeOneHundredTen}
		&\textbf{11.0}
		\\
		\textbf{Total $\eta_{exergy}$ (\%)}
		& \textbf{\totalExergyEffISPThreeNinety}
		& \textbf{\totalExergyEffISPThreeNinetyFive}
		& \textbf{\totalExergyEffISPThreeStandard}
		& \textbf{\totalExergyEffISPThreeOneHundredFive}
		& \textbf{\totalExergyEffISPThreeOneHundredTen}
		& \textbf{0.00109}
		\\
		\hline 
		\textbf{1$^{st}$ Stage $\eta_{exergy}$ (\%)}
		& \textbf{\firstExergyEffISPThreeNinety}
		& \textbf{\firstExergyEffISPThreeNinetyFive}
		& \textbf{\firstExergyEffISPThreeStandard}
		& \textbf{\firstExergyEffISPThreeOneHundredFive}
		& \textbf{\firstExergyEffISPThreeOneHundredTen}
		& -
		\\
		\textbf{Separation Alt, 1$\rightarrow$2 (km)}
		& \firstsecondSeparationAltISPThreeNinety
		& \firstsecondSeparationAltISPThreeNinetyFive
		& \firstsecondSeparationAltISPThreeStandard
		& \firstsecondSeparationAltISPThreeOneHundredFive
		& \firstsecondSeparationAltISPThreeOneHundredTen
		& -
		\\
		\textbf{Separation v, 1$\rightarrow$2 (m/s)}
		& \firstsecondSeparationvISPThreeNinety
		& \firstsecondSeparationvISPThreeNinetyFive
		& \firstsecondSeparationvISPThreeStandard
		& \firstsecondSeparationvISPThreeOneHundredFive
		& \firstsecondSeparationvISPThreeOneHundredTen
		& -
		\\
		\textbf{Separation $\gamma$, 1$\rightarrow$2 (deg)}
		& \firstsecondSeparationgammaISPThreeNinety
		& \firstsecondSeparationgammaISPThreeNinetyFive
		& \firstsecondSeparationgammaISPThreeStandard
		& \firstsecondSeparationgammaISPThreeOneHundredFive
		& \firstsecondSeparationgammaISPThreeOneHundredTen
		& -
		\\
		\hline 
		\textbf{2$^{nd}$ Stage $\eta_{exergy}$ (\%)}
		& \textbf{\secondExergyEffISPThreeNinety}
		& \textbf{\secondExergyEffISPThreeNinetyFive}
		& \textbf{\secondExergyEffISPThreeStandard}
		& \textbf{\secondExergyEffISPThreeOneHundredFive}
		& \textbf{\secondExergyEffISPThreeOneHundredTen}
		& -
		\\
		\textbf{Separation Alt, 2$\rightarrow$3 (km)}
		& \secondthirdSeparationAltISPThreeNinety
		& \secondthirdSeparationAltISPThreeNinetyFive
		& \secondthirdSeparationAltISPThreeStandard
		& \secondthirdSeparationAltISPThreeOneHundredFive
		& \secondthirdSeparationAltISPThreeOneHundredTen
		& -
		\\
		\textbf{Separation $v$, 2$\rightarrow$3 (m/s)}
		& \secondthirdSeparationvISPThreeNinety
		& \secondthirdSeparationvISPThreeNinetyFive
		& \secondthirdSeparationvISPThreeStandard
		& \secondthirdSeparationvISPThreeOneHundredFive
		& \secondthirdSeparationvISPThreeOneHundredTen
		& -
		\\
		\textbf{Separation $\gamma$, 2$\rightarrow$3 (deg)}
		& \secondthirdSeparationgammaISPThreeNinety
		& \secondthirdSeparationgammaISPThreeNinetyFive
		& \secondthirdSeparationgammaISPThreeStandard
		& \secondthirdSeparationgammaISPThreeOneHundredFive
		& \secondthirdSeparationgammaISPThreeOneHundredTen
		& -
		\\
		\textbf{2$^{nd}$ Stage Flight Time (s)}
		& \secondFlightTimeISPThreeNinety
		& \secondFlightTimeISPThreeNinetyFive
		& \secondFlightTimeISPThreeStandard
		& \secondFlightTimeISPThreeOneHundredFive
		& \secondFlightTimeISPThreeOneHundredTen
		& -
		\\
		\textbf{2$^{nd}$ Stage Distance Flown (km)}
		& \SecondDistISPThreeNinety
		& \SecondDistISPThreeNinetyFive
		& \SecondDistISPThreeStandard
		& \SecondDistISPThreeOneHundredFive
		& \SecondDistISPThreeOneHundredTen
		& -
		\\
		\textbf{2$^{nd}$ Stage Return Fuel (kg)}
		& \returnFuelISPThreeNinety
		& \returnFuelISPThreeNinetyFive
		& \returnFuelISPThreeStandard
		& \returnFuelISPThreeOneHundredFive
		& \returnFuelISPThreeOneHundredTen
		& -
		\\
		\textbf{2$^{nd}$ Stage Return Distance (km)}
		& \returnDistISPThreeNinety
		& \returnDistISPThreeNinetyFive
		& \returnDistISPThreeStandard
		& \returnDistISPThreeOneHundredFive
		& \returnDistISPThreeOneHundredTen
		& -
		\\
		\hline 
		\textbf{3$^{rd}$ Stage $\eta_{exergy}$ (\%)}
		& \textbf{\thirddExergyEffISPThreeNinety}
		& \textbf{\thirddExergyEffISPThreeNinetyFive}
		& \textbf{\thirddExergyEffISPThreeStandard}
		& \textbf{\thirddExergyEffISPThreeOneHundredFive}
		& \textbf{\thirddExergyEffISPThreeOneHundredTen}
		& \textbf{1.127}
		\\
		\textbf{3$^{rd}$ Stage $t$, $q >$ 5kpa (s)}
		& \thirdqOverFiveISPThreeNinety
		& \thirdqOverFiveISPThreeNinetyFive
		& \thirdqOverFiveISPThreeStandard
		& \thirdqOverFiveISPThreeOneHundredFive
		& \thirdqOverFiveISPThreeOneHundredTen
		& -
		\\
		\textbf{3$^{rd}$ Stage Fuel Mass (kg)}
		& \thirdmFuelISPThreeNinety
		& \thirdmFuelISPThreeNinetyFive
		& \thirdmFuelISPThreeStandard
		& \thirdmFuelISPThreeOneHundredFive
		& \thirdmFuelISPThreeOneHundredTen
		&-10.98
		\\
		\hline 
	\end{tabular} 
	\caption{Comparison of key trajectory parameters with variation in the specific impulse of the third stage, with fly-back (Case 18).}
	\label{tab:isp3}
\end{table}

\noindent
The specific impulse of the third stage rocket is varied by $\pm$10\% to investigate the sensitivity of the launch system to the performance of the third stage rocket, when the fly-back of the scramjet accelerator is included. 
Table \ref{tab:isp3} shows selected performance indicators, while Appendix \ref{sec:app_comparison91} shows comparison plots of the maximum payload-to-orbit trajectory at each third stage specific impulse. \textcolor{black}{Negative payload-to-orbit is produced at 90\% specific impulse, indicating that it is not possible for the third stage to reach the required orbit at this specific impulse. }

The sensitivity of the optimal trajectory to the third stage specific impulse with scramjet accelerator fly-back is very similar to that observed in the sensitivity study with no scramjet accelerator fly-back, in Section \ref{sec:isp3NoReturn}, with a sensitivity of \textcolor{black}{11.0}$\frac{\Delta m_{payload}}{\Delta\%I_{SP,3}}$ variation in both cases. 
These similar sensitivities indicate that the fly-back does not considerably effect the sensitivity of the launch system to variations in the third stage specific impulse.

\textcolor{black}{As observed in Section \ref{sec:isp3NoReturn}, there is a distinct cutoff as the specific impulse is lowered, at which the third stage no longer has the required performance to raise altitude consistently and exit the atmosphere effectively. Once again, the scramjet accelerator must compensate for the reduced third stage performance by lowering the pull-up at the end of its trajectory, increasing the speed and decreasing the trajectory angle at separation so that a greater speed is achieved at separation and the aerodynamics of the third stage are used more extensively.} 






\section{Comparison of Design Parameters with Fly-Back}\label{sec:thermal}%0.4 (q)/(50*0.01 (q%))/(1.5 (struct)/(4957*0.01 (%struct))) calculated like this. Do unit analysis if unsure.
\begin{figure}[th]
	\centering
	\textbf{Payload Variation (kg) Normalised to a 1\% Parameter Variation}\par\medskip
	\includegraphics[width=0.99\linewidth]{figures/6_FlyBack/BarChart}
	\caption{The sensitivity of the key design parameters of the launch system, including scramjet accelerator fly-back. Red and green coloured areas indicate decreases or increases in the magnitude of sensitivity respectively, compared to the sensitivity study without scramjet accelerator fly-back in Section \ref{sec:comparisonNoReturn}.}
	\label{fig:BarChartreturn}
\end{figure}
\noindent
The sensitivities of the performance of the launch system, including the fly-back of the scramjet accelerator, to a variety of design parameters have been presented in the preceding sections. Figure \ref{fig:BarChartreturn} shows a relative comparison of the payload-to-orbit sensitivity for each design parameter, normalised to a one percent variation in each respective parameter. The magnitude of each sensitivity is also compared with the sensitivity of the launch system performance without fly-back, detailed in Section \ref{sec:comparisonNoReturn}.

The sensitivity of the launch system to the \textcolor{black}{structural mass of the scramjet accelerator} is unchanged when fly-back is included. However, the increase in the sensitivity of the launch system to the \textcolor{black}{maximum allowable dynamic pressure}, to \textcolor{black}{1.4}$\frac{\Delta kg}{\Delta\%q_{max}}$, means that the potential beneficial effects of reducing the maximum dynamic pressure of the scramjet accelerator are reduced. The mass of the scramjet accelerator \textcolor{black}{must be decreased} by more than \textcolor{black}{-106.7}kg for each 1kPa \textcolor{black}{decrease} in the maximum dynamic pressure for the performance of the launch system will improve.
 The sensitivity of the launch system to the specific impulse of the scramjet accelerator is decreased significantly when the fly-back of the scramjet accelerator is included, to \textcolor{black}{1.7}$\frac{\Delta kg}{\Delta\%I_{SP,scramjet accelerator}}$, a decrease of \textcolor{black}{-0.5}$\frac{\Delta kg}{\Delta\%I_{SP,scramjet accelerator}}$ (\textcolor{black}{-22.7}\%) compared to the sensitivity without fly-back, while the sensitivity of the launch system to the scramjet accelerator's structural mass is unchanged, at \textcolor{black}{-1.3}$\frac{\Delta kg}{\Delta\%m_{scramjet accelerator}}$. Comparing these sensitivities, it is apparent that if the specific impulse of the scramjet accelerator can be increased by 1\% with less than \textcolor{black}{64.8kg} increase in the structural mass of the scramjet accelerator, then the overall performance of the launch system will be improved.  
Similarly, the sensitivity of the launch system to variation in the drag of the scramjet accelerator is reduced, to \textcolor{black}{-1.8}$\frac{\Delta kg}{\Delta\%C_{d,scramjet accelerator}}$. Comparing this sensitivity with the sensitivity to the \textcolor{black}{C-REST specific impulse}, the specific impulse of the C-REST engines must be improved by 1\% while increasing the drag of the scramjet accelerator by less than \textcolor{black}{0.9}\% due to shape variation, in order for the overall performance change to be beneficial. 

The \textcolor{black}{unchanged} sensitivity of the launch system performance to the structural mass of the scramjet accelerator, along with the decreased fuel mass sensitivity, means that so long as 1kg of fuel mass can be added with less than \textcolor{black}{1.5}kg of structural mass added, the performance of the launch system will improve. Additionally, the decreased sensitivity of the launch system to the drag of the scramjet accelerator means that so long as 1kg of fuel can be added to the scramjet accelerator, with a drag increase of less than 0.021\% due to increased size, then the maximum payload-to-orbit will increase \textcolor{black}{(assuming lift is unchanged)}. 

Once again, the payload variation effect of the third stage specific impulse is very high, at \textcolor{black}{11.6}$\frac{\Delta kg}{\Delta\%I_{SP,3}}$. However the variation of the specific impulse by even small amounts will require modifications to the engine and rocket internals, varying the mass and internal space requirements significantly.



\textcolor{black}{
	\section{Thermal Analysis of the Optimised Trajectory With Fly-Back}\label{sec:therm3}
}
\begin{figure}[!ht]
	\begin{subfigure}{.5\textwidth}
		\centering
		\includegraphics[width=0.99\linewidth]{figures/A1_uncertainty-analysis/TNoseReturn}
		\caption{The temperature time history at the nose (\textcolor{black}{1}).}
		
	\end{subfigure}
	\begin{subfigure}{.5\textwidth}
		\centering
		\includegraphics[width=0.99\linewidth]{figures/A1_uncertainty-analysis/TLEReturn}
		\caption{The temperature time history at the wing leading edge (\textcolor{black}{2}).}
		
	\end{subfigure}
	\begin{subfigure}{.5\textwidth}
		\centering
		\includegraphics[width=0.99\linewidth]{figures/A1_uncertainty-analysis/TPos1Return}
		\caption{The temperature time history on the nose cone (\textcolor{black}{3}).}
		
	\end{subfigure}
	\begin{subfigure}{.5\textwidth}
		\centering
		\includegraphics[width=0.99\linewidth]{figures/A1_uncertainty-analysis/TPos2Return}
		\caption{The temperature time history on the cowl (\textcolor{black}{4}).}
		
	\end{subfigure}
	\begin{subfigure}{.5\textwidth}
		\centering
		\includegraphics[width=0.99\linewidth]{figures/A1_uncertainty-analysis/TPos3Return}
		\caption{The temperature time history on the wing (\textcolor{black}{5}).}
		
	\end{subfigure}
	\begin{subfigure}{.5\textwidth}
		\centering
		\includegraphics[width=0.99\linewidth]{figures/A1_uncertainty-analysis/TPos4Return}
		\caption{The temperature time history on the tail (\textcolor{black}{6}).}
		
	\end{subfigure}
	\caption{Temperature time histories on the scramjet accelerator.}
	\label{fig:TrajTemp}
\end{figure}
\begin{figure}[ht]
	\begin{subfigure}{.495\textwidth}
		\centering
		\includegraphics[width=0.99\linewidth]{figures/A1_uncertainty-analysis/TNose3}
		\caption{The temperature time history at the nose (\textcolor{black}{1}).}
	\end{subfigure}
	\begin{subfigure}{.495\textwidth}
		\centering
		\includegraphics[width=0.99\linewidth]{figures/A1_uncertainty-analysis/T3onNose}
		\caption{The temperature time history on the nose cone (\textcolor{black}{2}).}
	\end{subfigure}
	\begin{subfigure}{.495\textwidth}
		\centering
		\includegraphics[width=0.99\linewidth]{figures/A1_uncertainty-analysis/T3Body}
		\caption{The temperature time history on the cylindrical body (\textcolor{black}{3}).}
	\end{subfigure}
	\caption{Temperature time histories on the third stage rocket with time, commencing at third stage release. }
	\label{fig:TrajTemp3}
\end{figure}
\noindent
To investigate the TPS thermal response of the representative launch system during flight, a 1-D heat analysis is conducted using the method described Section \ref{sec:therm}.
This analysis gives an indication of the temperature at the specified points around the representative launch system and third stage throughout their trajectories. 
Figure \ref{fig:TrajTemp} shows the temperature responses at the specified locations on the scramjet accelerator over the ascent and return trajectories in Case 11, corresponding to the TPS design and boundary conditions specified in Table \ref{tab:thermalprops}. The temperature on the outside of the scramjet accelerator rises throughout the acceleration, as expected, peaking just before third stage release. The temperature then generally decreases during descent, with smaller peaks during the 'skipping' manoeuvres, when the speed and dynamic pressure are raised.

On the scramjet accelerator, the tungsten nose reaches 2015.1K, and the Carbon-Carbon wing leading edge reaches 1665.8K. 
At the nose and leading edge it is assumed that there is no direct thermal connection to the internal insulation by the structure or heat shielding at the regions of maximum heat loads. The temperatures that have been observed are within the operational regimes of the external heat shielding materials\cite{Fitzer,Leonhardt}, and the maximum heating in the areas examined are assumed to be limited by the properties of the heat shielding itself. However, these high temperatures at the nose and leading edges must be taken into account when the TPS and internal connectors are designed, so that heat is effectively distributed away from the leading edge during flight. 

On the body of the scramjet accelerator, the interior of the nose cone reaches a maximum of 1363.1K, the cowl 1384.9K, the wing 1326.4K, and the tail 1142.3K.
These temperatures are within the operational regime of the heat shielding material, although the internal temperatures are at the limits of the operational regime of the internal multilayer insulation, which is nominally designed for operation in the 900-1100$^\circ$C (1173-1373K) range, although it has been tested at higher temperatures\cite{Kourtides}. These high temperatures may affect the mechanical strength of the internal insulation\cite{Kourtides}, and must be accounted for in the detailed internal design, so that the internal insulation does not fail when peak temperatures are reached. 

In addition to the temperatures approaching the limits of the multilayer insulation, the high temperatures will affect how the outer skin can be interfaced with the internal structure. 
In analytical studies based on experimental tests of the internal multilayer insulation at 1273K (1000$^\circ$C) approximate surface temperature, the innermost surface of the insulation has been shown to keep temperatures below 546K (273$^\circ$C)\cite{Kourtides} when the multilayer insulation with best performance (silica felt insulation\cite{Kourtides}) is used. Because the internal temperatures are above the experimentally tested temperature, the innermost surface temperature of the insulation is likely to be above 546K, necessitating complex internal structures and interfaces in order to provide mass-efficient mechanical strength, while distributing the thermal loading effectively.


%XXX Note: the temp is still increasing on the nose at the end, but I think this is just due to the equations breaking dopwn.. it is pretty much exoatmospheric by that point
Figure \ref{fig:TrajTemp3} shows the temperature response at the specified locations over the third stage rocket, while the rocket is accelerating with the heat shield attached. The temperature is initially low, due to the third stage being contained within the scramjet accelerator before release. The temperatures on the outside of the third stage rocket increase for most of its acceleration, reducing only at the very end, when the angle of attack is reduced, and the atmosphere is extremely thin. 
The temperatures on the third stage rocket are significantly lower than those experienced by the scramjet accelerator. The nose tip of the third stage reaches a maximum inner surface temperature of 953.5K, while the nose cone and body reach maximum inner surface temperatures of 475.9K and 779.8K respectively. These relatively low temperatures indicate that the heat shielding design specified in Table \ref{tab:thermalprops} may be over-engineered for a third stage released following a pull-up manoeuvre, and it may be possible to reduce the mass of the heat shield by reducing the thickness of the tungsten and C-C protection on the nose, or by changing the material entirely. 


%Max temps: T0 T1

%nose:
%1945.4358   1947.8085 

%LE
%1665.796    1695.4998 

%on body:
%Pos 1 - 1361.0127  1435.0191
%Pos 2 - 1337.0232  1350.7068 
%pos 3 - 1384.9077  1517.4993 
%pos 4 - 1142.2337  1290.9281 




\subsection{Thermal Protection Exploration}

The design of the thermal protection systems within an airbreathing launch system is critical to the successful operation of the system. While this work assumes that the TPS is sufficient for operation, there is a great deal of study to be done on hypersonic vehicle TPS, and the TPS design will need go through a detailed investigation and analysis process before a flight-ready design is developed. 
As the modelling of the launch system improves, its is likely that the TPS will be redesigned and heat shielding redistributed to optimise the weight of each vehicle while accounting for heating distribution. The following sections explore the effect of changing the physical properties of the external heat shielding, applying active cooling, and the impact of imposing peak heat load limits on the launch system trajectory and performance. 

\subsubsection{Scramjet Accelerator TPS Design Exploration}

The simplest method of modifying the TPS of the launch system is to modify the properties of the external heat shielding. 
To illustrate the effects of these variations, the thickness, density, thermal conductivity and specific heat of the C-C heat shield on the scramjet engine shroud (Position \textcolor{black}{4}) are varied by $\pm$10\%. The shroud is chosen in order to measure the effect of the heat shielding properties and design at the region of highest heating with direct connection to the internal insulation and structure. The maximum inner surface temperature recorded is shown in Table \ref{tab:tpsscramjet}. These variations indicate an inner surface temperature sensitivity of -2.7K/\% to variation in thickness, -1.9K/\% to variation in density, and 0.6 K/\% to variation in thermal conductivity when varied from the standard CCAT T-300 properties. 
The effects of a 10\% variation in any of these properties is not particularly large. This indicates that if the maximum inner surface temperature is to be reduced in an area of high heating through modification of the passive heat shielding, the heat shield design or properties in that area must be modified significantly to produce any notable effects. 
%XXX  density is the same as specific heat 
\begin{table}[ht]
	\centering
	\begin{tabular}{|c|c|c|c|c|c|}
		\hline  & -10\% & -5\% & Standard & +5\% & +10\% \\ 
		\hline Thickness & 1418.0 & 1401.4 & 1384.9  &   1375.6&   1363.6\\ 
		\hline Density & 1408.6 &  1396.3  & 1384.9 &  1378.0&  1369.9\\ 
		\hline Thermal Conductivity &  1380.4 & 1382.9 & 1384.9 &  1388.9& 1393.6\\ 
		\hline 
	\end{tabular}
	
	\caption{Inner surface temperatures in kelvin on the shroud of the scramjet accelerator with variations in heat shielding properties.}
	\label{tab:tpsscramjet}
\end{table}

If passive TPS is not sufficient, active cooling may be introduced in areas of high heat loads. This active cooling can take many forms, such as regenerative, film, and transpiration cooling\cite{Zhu2018}, all of which are currently being actively studied for applications to hypersonic vehicles. For the most critical areas of high heating on the scramjet body, regenerative cooling is assumed to be the most practical method of active cooling, and the complex effects of this cooling are approximated as a set rate of heat transfer away from the inner face that directly reduces the temperature of the inner face of the heat shielding. Inner face temperatures on the shroud with heat transfer rates from 0 to 20kW/m$^2$ away from the inner face are compared in Table \ref{tab:regenerativecooling}. The inner face temperature shows a sensitivity of -5.2 K/kW to a constant rate of heat transfer applied to the inner face. 
\begin{table}[ht]
	\centering
	\begin{tabular}{|c|c|c|c|c|c|}
		\hline Heat Transfer Rate & Standard (0kW/m$^2$) & 5kW/m$^2$ & 10kW/m$^2$ & 15kW/m$^2$ & 20kW/m$^2$ \\ 
		\hline Maximum T (K) & 1384.9 & 1359.3 & 1333.4 & 1307.4 & 1281.2 \\ 
		\hline
	\end{tabular}	
	\caption{Maximum inner surface temperature with a constant rate of heat transfer away from the inner surface of the shroud.}
	\label{tab:regenerativecooling}	
\end{table}

\subsection{Case 19: Heating limited Trajectory}
\begin{table}[ht]
	
	\centering
	\begin{tabular}{l c c c c c } 
		\hline \textbf{Peak Heat Load (Cold Wall)} (kW/m$^2$):
		& Standard (1770)
		& 1700
		& 1600
		& 1500
		& 1400
		\\
		\hline \textbf{Payload to Orbit (kg)}
		& \textbf{\PayloadToOrbitheatLimStandard}
		& \textbf{\PayloadToOrbitheatLimSeventeenHundred}
		& \textbf{\PayloadToOrbitheatLimSixteenHundred}
		& \textbf{\PayloadToOrbitheatLimFifteenHundred}
		& \textbf{\PayloadToOrbitheatLimFourteenHundred}
		\\
		\textbf{Total $\eta_{exergy}$ (\%)}
		& \textbf{\totalExergyEffheatLimStandard}
		& \textbf{\totalExergyEffheatLimSeventeenHundred}
		& \textbf{\totalExergyEffheatLimSixteenHundred}
		& \textbf{\totalExergyEffheatLimFifteenHundred}
		& \textbf{\totalExergyEffheatLimFourteenHundred}
		\\
		\hline 
		\textbf{1$^{st}$ Stage $\eta_{exergy}$ (\%)}
		& \textbf{\firstExergyEffheatLimStandard}
		& \textbf{\firstExergyEffheatLimSeventeenHundred}
		& \textbf{\firstExergyEffheatLimSixteenHundred}
		& \textbf{\firstExergyEffheatLimFifteenHundred}
		& \textbf{\firstExergyEffheatLimFourteenHundred}
		\\
		\textbf{Separation Alt, 1$\rightarrow$2 (km)}
		& \firstsecondSeparationAltheatLimStandard
		& \firstsecondSeparationAltheatLimSeventeenHundred
		& \firstsecondSeparationAltheatLimSixteenHundred
		& \firstsecondSeparationAltheatLimFifteenHundred
		& \firstsecondSeparationAltheatLimFourteenHundred
		\\
		\textbf{Separation v, 1$\rightarrow$2 (m/s)}
		& \firstsecondSeparationvheatLimStandard
		& \firstsecondSeparationvheatLimSeventeenHundred
		& \firstsecondSeparationvheatLimSixteenHundred
		& \firstsecondSeparationvheatLimFifteenHundred
		& \firstsecondSeparationvheatLimFourteenHundred
		\\
		\textbf{Separation $\gamma$, 1$\rightarrow$2 (deg)}
		& \firstsecondSeparationgammaheatLimStandard
		& \firstsecondSeparationgammaheatLimSeventeenHundred
		& \firstsecondSeparationgammaheatLimSixteenHundred
		& \firstsecondSeparationgammaheatLimFifteenHundred
		& \firstsecondSeparationgammaheatLimFourteenHundred
		\\
		\hline 
		\textbf{2$^{nd}$ Stage $\eta_{exergy}$ (\%)}
		& \textbf{\secondExergyEffheatLimStandard}
		& \textbf{\secondExergyEffheatLimSeventeenHundred}
		& \textbf{\secondExergyEffheatLimSixteenHundred}
		& \textbf{\secondExergyEffheatLimFifteenHundred}
		& \textbf{\secondExergyEffheatLimFourteenHundred}
		\\
		\textbf{Separation Alt, 2$\rightarrow$3 (km)}
		& \secondthirdSeparationAltheatLimStandard
		& \secondthirdSeparationAltheatLimSeventeenHundred
		& \secondthirdSeparationAltheatLimSixteenHundred
		& \secondthirdSeparationAltheatLimFifteenHundred
		& \secondthirdSeparationAltheatLimFourteenHundred
		\\
		\textbf{Separation $v$, 2$\rightarrow$3 (m/s)}
		& \secondthirdSeparationvheatLimStandard
		& \secondthirdSeparationvheatLimSeventeenHundred
		& \secondthirdSeparationvheatLimSixteenHundred
		& \secondthirdSeparationvheatLimFifteenHundred
		& \secondthirdSeparationvheatLimFourteenHundred
		\\
		\textbf{Separation $\gamma$, 2$\rightarrow$3 (deg)}
		& \secondthirdSeparationgammaheatLimStandard
		& \secondthirdSeparationgammaheatLimSeventeenHundred
		& \secondthirdSeparationgammaheatLimSixteenHundred
		& \secondthirdSeparationgammaheatLimFifteenHundred
		& \secondthirdSeparationgammaheatLimFourteenHundred
		\\
		\textbf{2$^{nd}$ Stage Flight Time (s)}
		& \secondFlightTimeheatLimStandard
		& \secondFlightTimeheatLimSeventeenHundred
		& \secondFlightTimeheatLimSixteenHundred
		& \secondFlightTimeheatLimFifteenHundred
		& \secondFlightTimeheatLimFourteenHundred
		\\
		\textbf{2$^{nd}$ Stage Distance Flown (km)}
		& \SecondDistheatLimStandard
		& \SecondDistheatLimSeventeenHundred
		& \SecondDistheatLimSixteenHundred
		& \SecondDistheatLimFifteenHundred
		& \SecondDistheatLimFourteenHundred
		\\
		\textbf{2$^{nd}$ Stage Return Fuel (kg)}
		& \returnFuelheatLimStandard
		& \returnFuelheatLimSeventeenHundred
		& \returnFuelheatLimSixteenHundred
		& \returnFuelheatLimFifteenHundred
		& \returnFuelheatLimFourteenHundred
		\\
		\textbf{2$^{nd}$ Stage Return Distance (km)}
		& \returnDistheatLimStandard
		& \returnDistheatLimSeventeenHundred
		& \returnDistheatLimSixteenHundred
		& \returnDistheatLimFifteenHundred
		& \returnDistheatLimFourteenHundred
		\\
		\hline 
		\textbf{3$^{rd}$ Stage $\eta_{exergy}$ (\%)}
		& \textbf{\thirddExergyEffheatLimStandard}
		& \textbf{\thirddExergyEffheatLimSeventeenHundred}
		& \textbf{\thirddExergyEffheatLimSixteenHundred}
		& \textbf{\thirddExergyEffheatLimFifteenHundred}
		& \textbf{\thirddExergyEffheatLimFourteenHundred}
		\\
		\textbf{3$^{rd}$ Stage $t$, $q >$ 5kpa (s)}
		& \thirdqOverFiveheatLimStandard
		& \thirdqOverFiveheatLimSeventeenHundred
		& \thirdqOverFiveheatLimSixteenHundred
		& \thirdqOverFiveheatLimFifteenHundred
		& \thirdqOverFiveheatLimFourteenHundred
		\\
		\textbf{3$^{rd}$ Stage Fuel Mass (kg)}
		& \thirdmFuelheatLimStandard
		& \thirdmFuelheatLimSeventeenHundred
		& \thirdmFuelheatLimSixteenHundred
		& \thirdmFuelheatLimFifteenHundred
		& \thirdmFuelheatLimFourteenHundred
		\\
		\hline 
	\end{tabular} 
	\caption{Scramjet stage stagnation point heating limited trajectories, optimised for maximum payload-to-orbit with stagnation point heating calculated using a cold wall approximation. }
	\label{tab:stagLim}
\end{table}
\begin{figure}[!ht]
	\centering
	\includegraphics[width=0.9\linewidth]{H:/github-home/LODESTAR-revisions/Results/mode151/SecondStageComparison}
	\caption{Comparison of scramjet accelerator trajectories with varying stagnation point heating limits. Maximum heat transfer is shown for each heating limit, with more restrictive heating limits exhibiting flight at maximum heating for long periods. }
	\label{fig:SecondStageheatlimComparison}
\end{figure}

\noindent
The aerothermal effects on a launch vehicle may also be reduced by limiting the heat loads on the vehicle during flight. However, performing a 1-D heat transfer analysis within a trajectory optimisation loop is prohibitively expensive computationally. 
As such, to investigate the effects of limiting the maximum aerothermal heating on the scramjet accelerator in a computationally efficient manner, the stagnation point heat flux is calculated in-loop using a cold wall approximation where the wall temperature is much smaller than the adiabatic wall temperature ($T_W/T_{AW} \approx 0$)\cite{Dirkx}. This calculation of the heat flux is conservative, however it is useful for conducting a quantitative comparison of peak heat loading. Stagnation point heat flux limits of 1700, 1600, 1500 and 1400kW/m$^2$ are applied, based on the maximum stagnation point heat flux for a maximum payload-to-orbit trajectory with return (Case 11) of 1770kW/m$^2$, calculated using a cold-wall approximation. Optimised trajectories calculated with these heat flux limits applied to the scramjet accelerator stage are presented in Table \ref{tab:stagLim}, and the trajectories of the scramjet stages are shown in Figure \ref{fig:SecondStageheatlimComparison}. 

From Table \ref{tab:stagLim} and Figure \ref{fig:SecondStageheatlimComparison} it can be observed that limiting the stagnation point heat flux has a significant effect on the optimised trajectory. However, it is also evident that the effects of the 1700kW/m$^2$ and 1600kW/m$^2$ limits are small, compared to the effects of 1500kW/m$^2$ and 1400kW/m$^2$ limits. 
This sharp change in trajectory shapes is caused by a distinct cut-off point at which the scramjet stage is no longer able to perform the manoeuvres necessary for a pull-up. When stagnation point heat flux is limited to 1700 or 1600kW/m$^2$, the scramjet stage compensates by flying a slower, higher trajectory after release from the first stage rocket, reaching maximum dynamic pressure later, and performing a larger pull-up. This flight path limits the speed of the scramjet stage and manoeuvres into a favourable position, while still performing a pull-up manoeuvre. In these cases, maximum heat flux is reached at the start of the pull-up manoeuvre. 

However, when the stagnation point heating rate is limited to 1500kW/m$^2$ or 1400kW/m$^2$, the trajectory changes significantly. For these limits, the beginning of the scramjet stage trajectory conforms with the non-limited optimal flight path, but the stagnation point heating limit causes the scramjet stage to leave maximum dynamic pressure and climb in the middle of the trajectory, reducing L/D and manoeuvrability, and reducing the ability of the scramjet accelerator stage to perform a pull-up manoeuvre. 

Generally, these optimised trajectory results indicate that heating limits imposed on the trajectory of the scramjet accelerator vehicle result in the vehicle flying a significantly less efficient trajectory in order to reduce heat loads.
For small heating limitations this may be suitable, however, in general, it is important to design an efficient airbreathing vehicle to be able to tolerate the heat loads required throughout the entirety of an optimal payload-to-orbit flight path. 



\subsection{Third Stage Heat Shielding Design Exploration}\label{sec:thirdstageheat}
\begin{table}[!ht]
	\centering
	\begin{tabular}{|c|c|c|c|c|c|}
		\hline Property & -10\% & -5\% & Standard & +5\% & +10\% \\ 
		\hline Thickness & 486.4 & 480.7 & 475.9 &  471.7 & 468.1  \\ 
		\hline Density &  489.9 & 482.6 & 475.9 & 469.6 &  464.3\\ 
		\hline Thermal Conductivity & 480.0 & 477.9 & 475.9 & 474.1 & 472.5 \\ 
		\hline 
	\end{tabular} 
	
	\caption{Inner face temperatures in kelvin with variations in third stage heat shielding properties.}
	\label{tab:tpsthirdstage}
\end{table}
\noindent
To explore the design of the third stage heat shielding, various properties of the C-C on the conical nose cone are varied by $\pm$10\%, and the temperature time histories along the optimised payload-to-orbit trajectory with fly-back (Case 11) are calculated. The maximum temperatures observed on the inner surface at the centreline of the nose cone heat shielding are shown in Table \ref{tab:tpsthirdstage}. These variations show an inner surface temperature sensitivity of -0.9 K/\% to variation in thickness, -1.3 K/\% to variation in density, and -0.4 K/\% to variation in thermal conductivity when varied from the standard CCAT T-300 properties. These variations are relatively small, however, part of the reason for this is that the heat transfer, and the temperatures, are relatively low. 






\subsection{Case 20: The Effects of a Pull-Up on the Third Stage Heat Shield Design}\label{sec:TPSredesign}
\begin{table}[!ht]
	\centering
	
	\begin{tabular}{l c c} 
		& Reduced m$_{HS}$
		& Standard m$_{HS}$ (Case 11)
		\\
		\hline \textbf{Payload to Orbit (kg)}
		& \textbf{\PayloadToOrbitTPSreduced}
		& \textbf{\PayloadToOrbitStandard}
		\\
		\textbf{Total $\eta_{exergy}$ (\%)}
		& \textbf{\totalExergyEffTPSreduced}
		& \textbf{\totalExergyEffStandard}
		\\
		\hline 
		\textbf{1$^{st}$ Stage $\eta_{exergy}$ (\%)}
		& \textbf{\firstExergyEffTPSreduced}
		& \textbf{\firstExergyEffStandard}
		\\
		\textbf{Separation Alt, 1$\rightarrow$2 (km)}
		& \firstsecondSeparationAltTPSreduced
		& \firstsecondSeparationAltStandard
		\\
		\textbf{Separation v, 1$\rightarrow$2 (m/s)}
		& \firstsecondSeparationvTPSreduced
		& \firstsecondSeparationvStandard
		\\
		\textbf{Separation $\gamma$, 1$\rightarrow$2 (deg)}
		& \firstsecondSeparationgammaTPSreduced
		& \firstsecondSeparationgammaStandard
		\\
		\hline 
		\textbf{2$^{nd}$ Stage $\eta_{exergy}$ (\%)}
		& \textbf{\secondExergyEffTPSreduced}
		& \textbf{\secondExergyEffStandard}
		\\
		\textbf{Separation Alt, 2$\rightarrow$3 (km)}
		& \secondthirdSeparationAltTPSreduced
		& \secondthirdSeparationAltStandard
		\\
		\textbf{Separation $v$, 2$\rightarrow$3 (m/s)}
		& \secondthirdSeparationvTPSreduced
		& \secondthirdSeparationvStandard
		\\
		\textbf{Separation $\gamma$, 2$\rightarrow$3 (deg)}
		& \secondthirdSeparationgammaTPSreduced
		& \secondthirdSeparationgammaStandard
		\\
		\textbf{2$^{nd}$ Stage Flight Time (s)}
		& \secondFlightTimeTPSreduced
		& \secondFlightTimeStandard
		\\
		\textbf{2$^{nd}$ Stage Distance Flown (km)}
		& \SecondDistTPSreduced
		& \SecondDistStandard
		\\
		\textbf{2$^{nd}$ Stage Return Fuel (kg)}
		& \returnFuelTPSreduced
		& \returnFuelStandard
		\\
		\textbf{2$^{nd}$ Stage Return Distance (km)}
		& \returnDistTPSreduced
		& \returnDistStandard
		\\
		\hline 
		\textbf{3$^{rd}$ Stage $\eta_{exergy}$ (\%)}
		& \textbf{\thirddExergyEffTPSreduced}
		& \textbf{\thirddExergyEffStandard}
		\\
		\textbf{3$^{rd}$ Stage $t$, $q >$ 5kpa (s)}
		& \thirdqOverFiveTPSreduced
		& \thirdqOverFiveStandard
		\\
		\textbf{3$^{rd}$ Stage Fuel Mass (kg)}
		& \thirdmFuelTPSreduced
		& \thirdmFuelStandard
		\\
		\hline 
	\end{tabular} 
	
	\caption{Summary of a payload-to-orbit optimised trajectory with a 46.2kg third stage heat shield.}
	\label{tab:heatshieldreduced}
	
\end{table}
\noindent
The third stage rocket in this study has been designed to be released from the end of a constant dynamic pressure trajectory, in line with design studies of the SPARTAN\cite{Preller2018a}. This third stage rocket has a heat shield mass of 124.6kg. However, a pull-up at the end of the scramjet stage acceleration has been shown throughout Chapters \ref{chapter:Ascent} and \ref{chapter:Flyback} to be an integral part of the payload optimal trajectory for a rocket-scramjet-rocket launch system. In addition to maximising payload, this pull-up also has significant aerothermal benefits for the third stage rocket, reducing the heat loading on the third stage significantly, and potentially allowing for the heat shielding of the third stage to be reduced. When separated from the end of a pull-up manoeuvre during the maximum payload-to-orbit trajectory, the third stage experienced maximum temperatures of 953.5K on the nose tip, 475.9K on the nose cone, and 779.8K on the body. These temperatures are much smaller than the temperatures experienced after release from a constant dynamic pressure trajectory (Case 23), of 2254.8K on the nose tip, 1511.7K on the nose cone, and 1415.8k on the body. The much lower temperatures after release from the end of a pull-up manoeuvre indicate that the heat shield may be over-designed for a maximum payload-to-orbit optimised trajectory.

To illustrate the effects of redesigning the heat shielding of the third stage rocket, the third stage heat shield thickness is reduced. The effects of the pull-up on the heat shield temperatures are too large to apply the linear sensitivity relationship developed in Section \ref{sec:thirdstageheat}. Because of this, the thicknesses of the nose tip, nose cone and cylindrical body are all reduced proportionally to the comparative reduction in temperature when the third stage is released after a pull-up, when compared to a third stage released without a pull-up. This resulted in a heat shield weighing 46.2kg. An optimised trajectory was produced using this heat shield mass, with results summarised in Table \ref{tab:heatshieldreduced}. The payload-to-orbit increases to \PayloadToOrbitTPSreduced, an increase of 15.3kg (11.6\%) when compared to an optimised trajectory with the full heat shield mass. This increase illustrates the further payload-to-orbit benefits of a pull-up manoeuvre beyond simply increasing the operational performance of the launch system, by reducing the thermal loading on the third stage, and thus decreasing the necessary heat shielding mass. 

%-const q with fly-back
%
%compare max temps to pull-up
%
%recuce heat shield thickness by amount corresponding to indicated trend in previous section
%
%re-do optimised trajectory with new heat shield mass
%
%
%nose tip:
%874 - pull-up
%
%2163.4634 - constq
%
%40\%
%
%nosecone:
%439 - pull-up
%
%1336.6335 - const q
%
%32.8\%
%
%body:
%
%377 - pull-up
%
%1047.8395 - constq
%
%36\%
%
%
%modified heat shield weight: 42.4kg
%
%simulation with reduced heat shield: 149.3kg to orbit


\section{Summary}


In this chapter, the maximum payload-to-orbit trajectory for the representative rocket-scramjet-rocket system has been calculated, with the inclusion of the fly-back of the scramjet accelerator stage. It was found that this launch system is able to deliver \PayloadToOrbitStandard kg of payload to sun synchronous orbit (with a 97.5\% confidence interval of 6.8-233.6kg), while successfully returning the scramjet-powered stage to the initial launch site. 
This return flight decreases the nominal payload-to-orbit by \textcolor{black}{-24.3kg (-15.5\%)} when compared to a trajectory with no fly-back, but removes the need for the costly and time consuming transportation of the scramjet accelerator after launch that would be necessary if landing at a downrange location.
During the return flight, the scramjet engines are powered on three times, in total using \returnFuelStandard kg of fuel for the return flight, \textcolor{black}{16.5}\% of the scramjet accelerator's total fuel.

It was found that when the fly-back of the scramjet accelerator is included in the optimal trajectory calculation, the first stage of the launch system pitches in an easterly direction. 
The launch system exhibits a first-second separation point of \firstsecondSeparationAltStandard km and a trajectory angle of \firstsecondSeparationgammaStandard $^\circ$, \textcolor{black}{similar to the maximum payload-to-orbit trajectory with no fly-back. 
When the return flight is included,} the higher first-second stage separation point both increases the exergy efficiency of the first stage, and allows the scramjet accelerator time to increase its bank angle, so that when the scramjet accelerator descends to a high dynamic pressure it is able to change its heading angle rapidly. The scramjet accelerator maintains a high bank angle throughout its trajectory, executing a banking manoeuvre, and staying close to its maximum dynamic pressure. 
This banking manoeuvre requires higher angles of attack, increasing the drag of the scramjet accelerator, but also reduces the ground distance necessary for the return of the scramjet accelerator, decreasing the amount of fuel necessary for fly-back and increasing the overall efficiency of the scramjet accelerator. 
At the end of its acceleration, the scramjet accelerator was found to exhibit a pull-up manoeuvre before the separation of the third stage, in a similar fashion to the maximum payload-to-orbit trajectory with no fly-back. 

The fly-back of the scramjet accelerator is found to be separated into three stages; an initial turn, a boost phase, and an approach. 
The initial turn takes place immediately after separation, and consists of the scramjet accelerator banking heavily in order to decrease altitude after separation and manoeuvre the heading angle back towards the initial launch site. 
During the boost-skip phase the scramjet accelerator exhibits multiple `skipping' manoeuvres. These skipping manoeuvres have been shown in previous literature to extend the flight range of hypersonic vehicles\cite{Moshman2014,Darby2011,Toso2015,Tetlow1992,Eggers1957,Kanda2007,Chai2015}, and serve to reduce the amount of fuel used during the fly-back.
The skipping manoeuvres allow the scramjet engines to be powered on at the points where the specific impulse of the C-REST engines are highest, maximising the performance of the scramjet accelerator, and minimising the fuel necessary for return. 
During the approach phase, the trajectory of the scramjet accelerator is smoothed, and the scramjet accelerator glides to the landing point. 
 The optimal trajectory terminates when scramjet accelerator reaches 0.1km altitude at a speed of \textcolor{black}{95.2}m/s. After this point, it is assumed that the scramjet accelerator lands on a traditional runway.  
This result indicates that it is feasible to return a hypersonic launch vehicle separated at a high Mach number to its initial launch site, using only the high-speed airbreathing engines.  

The sensitivity of the launch system and its trajectory to various design parameters has been investigated. 
The payload-to-orbit sensitivity of the launch system to variations in the specific impulse, drag, structural mass and fuel mass of the scramjet accelerator was found to decrease when fly-back is included, compared to the sensitivity study with no fly-back. This decreased sensitivity indicates that the fly-back of the scramjet accelerator offsets some of the payload-to-orbit variation due to changes in these parameters, \textcolor{black}{particularly due to increased speed at third stage separation or down-range distance making the return flight of the scramjet accelerator more difficult}. \textcolor{black}{However, the sensitivity to the maximum dynamic pressure was found to increase, due to the more rapid fuel use when the dynamic pressure limit is increased lowering the flight time of the scramjet accelerator, in turn reducing the distance flown during return. Decreased performance of the third stage was found to affect the second-third stage separation point significantly, past a certain point. This indicates that when the third stage performs poorly, the performance trade-off between the scramjet accelerator and third stage shifts in favour of the scramjet accelerator, so that the scramjet accelerator prioritises flying a more efficient trajectory, rather than delivering the third stage to a higher release point.}


\textcolor{black}{
	A 1-D heat conduction analysis was undertaken to investigate the aerothermal effects of flying a payload-to-orbit optimised trajectory on the Representative Launch System. 
	Various points around the scramjet stage and third stage were investigated, and it was found that the maximum temperatures during flight are close to the operable limits of the internal insulation materials, for the reference TPS design and locations defined in Table \ref{tab:thermalprops}, reinforcing the need for prioritising detailed thermal analysis and internal design for airbreathing launch systems. 
	To investigate the TPS design and inform future study direction, the thickness, density and thermal conductivity of the passive heat shielding were varied, and it was found that modification of these properties produces only small effects on the maximum temperatures observed. The addition of active cooling modelled via a constant rate heat sink was tested, and produced significant temperature reductions. 
	Stagnation point heating limitations were imposed on the optimal trajectory, which had only a small effect at limits of 1700 and 1600kW/m$^2$ (Compared to a 1770kW/m$^2$ maximum heat loading when unconstrained), requiring manoeuvring of the scramjet accelerator at the beginning of the trajectory. However, heating rate limits of 1500 and 1400kW/m$^2$ produced a distinct cutoff point at which the scramjet accelerator could no longer manoeuvre sufficiently during its acceleration to both reduce heat loads and perform a pull-up manoeuvre, and the performance of the launch system degraded significantly. 
	Finally, to illustrate the possible design benefits of a pull-up manoeuvre, the third stage heat shielding was reduced in thickness on each segment proportionally to the reduction in temperature caused by a pull-up manoeuvre. A trajectory optimised for payload-to-orbit with this reduced heat shield mass exhibited an increase in payload mass to \PayloadToOrbitTPSreduced kg, a +15.3 (+11.6\%) increase over a trajectory optimised with the standard heat shield mass.}
