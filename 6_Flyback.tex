% numerical.tex

\cleardoublepage
\chapter{Optimised Trajectory Including Fly-Back}\label{chapter:Flyback}

\textcolor{red}{I should be clear that the method of calculating the payload mass to orbit does not affect the optimised trajectory}

This chapter presents the maximum payload-to-orbit trajectory of the rocket-scramjet-rocket launch system, with the fly-back of the SPARTAN included within the optimal trajectory calculation performed by LODESTAR. 
Flying back the SPARTAN for landing at the initial launch site is one of the primary enabling factors in the cost efficient operation of the launch system. If the SPARTAN is launched onto a trajectory from which it is not able to fly-back, it must perform a downrange landing, likely at an Indonesian airfield when launched northerly from north Australia. This would necessitate transporting the SPARTAN back to Australia, a costly and time consuming process, and would require for international landing facilities to be available. 
Flying back the SPARTAN during the launch process removes the need for costly transportation from a downrange launch site, and allows for rapid refurbishment and re-use.
In addition, if a launch site is used from which there is no downrange landing site, the SPARTAN must necessarily fly-back to the initial launch site. 


\begin{figure}[ht]
	\centering
	\includegraphics[width=1\linewidth]{../LODESTAR_FINAL/Results/mode11/GroundTrackStandard}
	\caption{Maximum payload-to-orbit trajectory path with the inclusion of SPARTAN fly-back (Case 11). Initial heading angle of -12.44$^\circ$.}
	\label{fig:GroundTrackStandard}
\end{figure}

The fly-back of the SPARTAN requires turning-around the SPARTAN after third stage separation, covering the necessary ground distance for return, and decelerating to reduce the speed of the SPARTAN to landing approach velocity, while maintaining a suitable descent angle to allow for a controlled approach. 
The return of the SPARTAN to the initial launch site is included in the optimisation process to assess whether it is possible for the fly-back of the SPARTAN to be achieved as a part of the launch process, and to maximise the overall payload-to-orbit efficiency of the launch system. This is compared to the optimised, maximum payload-to-orbit trajectory without fly-back (detailed in Chapter \ref{chapter:Ascent}) to assess the detrimental effects of the fly-back on the performance of the launch system. 
A sensitivity analysis is conducted, in a similar fashion to Chapter \ref{chapter:Ascent}. 
This sensitivity analysis allows the influence of the fly-back of the SPARTAN on the design sensitivities of the launch system to be analysed.


\section{Case 11: Combined SPARTAN Ascent-Descent \& Third Stage}
\begin{table}[ht]
	\centering
\begin{tabular}{l c } 
	\hline \textbf{Trajectory Condition}
	&Value 
	\\
	\hline \textbf{Payload to Orbit (kg)}
	& \textbf{\PayloadToOrbitStandard}
	\\
	\textbf{Total $\eta_{exergy}$ (\%)}
	& \textbf{\totalExergyEffStandard}
	\\
	\hline 
	\textbf{1$^{st}$ Stage $\eta_{exergy}$ (\%)}
	& \textbf{\firstExergyEffStandard}
	\\

	\textbf{Separation Alt, 1$\rightarrow$2 (km)}
	& \firstsecondSeparationAltStandard
	\\
	\textbf{Separation v, 1$\rightarrow$2 (m/s)}
	& \firstsecondSeparationvStandard
	\\
	\textbf{Separation $\gamma$, 1$\rightarrow$2 (deg)}
	& \firstsecondSeparationgammaStandard
	\\
	\hline 
	\textbf{2$^{nd}$ Stage $\eta_{exergy}$ (\%)}
	& \textbf{\secondExergyEffStandard}
	\\

	\textbf{Separation Alt, 2$\rightarrow$3 (km)}
	& \secondthirdSeparationAltStandard
	\\
	\textbf{Separation $v$, 2$\rightarrow$3 (m/s)}
	& \secondthirdSeparationvStandard
	\\
	\textbf{Separation $\gamma$, 2$\rightarrow$3 (deg)}
	& \secondthirdSeparationgammaStandard
	\\

	\textbf{2$^{nd}$ Stage Distance Flown (km)}
	& \SecondDistStandard
	\\
	\textbf{2$^{nd}$ Stage Return Fuel (kg)}
	& \returnFuelStandard
	\\
	\textbf{2$^{nd}$ Stage Return Distance (km)}
	& \returnDistStandard
	\\
	\hline 
	\textbf{3$^{rd}$ Stage $\eta_{exergy}$ (\%)}
	& \textbf{\thirddExergyEffStandard}
	\\

	\textbf{3$^{rd}$ Stage $t$, $q >$ 5kpa (s)}
	& \thirdqOverFiveStandard
	\\
	\textbf{3$^{rd}$ Stage max $\alpha$ (deg)}
	& \thirdmaxAoAStandard
	\\
	\textbf{3$^{rd}$ Stage Fuel Mass (kg)}
	& \thirdmFuelStandard
	\\
	\hline 
\end{tabular} 
\caption{Selected trajectory conditions for a maximum payload-to-orbit trajectory including SPARTAN fly-back (Case 11).}
\end{table}

LODESTAR is used to optimise the trajectory of the rocket-scramjet-rocket launch system, including the return of the SPARTAN to its initial launch site. The optimised trajectory is shown in Figure \ref{fig:GroundTrackStandard}. 
The rocket-scramjet-rocket launch system is shown to be able to successfully launch a small satellite to sun synchronous orbit, 
while flying-back the SPARTAN to the initial launch site location, and approaching the landing site at appropriately low altitude and velocity to allow for landing. 
The optimised trajectory attains a payload mass to SSO of \PayloadToOrbitStandard kg, a -19.0kg (-10.0\%) reduction in payload mass compared to the optimised ascent-only trajectory, detailed in Chapter \ref{chapter:Ascent}. 
The benefits of flying back the SPARTAN to its initial launch site, compared to the alternative of transporting the SPARTAN back to the launch site from a remote landing, are likely to far outweigh this associated reduction in payload. 



\section{Ascent Trajectory}

\begin{figure}[ht]
	\centering
	\includegraphics[width=0.9\linewidth]{../LODESTAR_FINAL/Results/mode11/FirstStageStandard}
	\caption{The first stage of the optimised maximum payload-to-orbit trajectory with SPARTAN fly-back (Case 11). }
	\label{fig:FirstStageStandard}
\end{figure}
When the fly-back of the SPARTAN is included in the trajectory optimisation, the shape of the ascent trajectory of the launch system is altered significantly, compared to the ascent-only trajectory, detailed in Chapter \ref{chapter:Ascent}.
 The first stage initially pitches towards the east, beginning at a heading angle of -12.4$^\circ$.
 After pitchover, the first stage gradually reduces the angle of attack to a minimum of -0.47$^\circ$ at 30.9s flight time, in order to make small adjustments to the pitch profile while the velocity is low. After this, the first stage angle of attack returns to 0$^\circ$ at 42.9s flight time, and is maintained for 16.4s.
 The angle of attack is then reduced, to a minimum of -3.58$^\circ$ in order to adjust the altitude and trajectory angle, before increasing back to 0$^\circ$ at first stage-SPARTAN separation. 
 The SPARTAN is released in an easterly direction, at a heading angle of -12.4$^\circ$, an altitude of \firstsecondSeparationAltStandard km, and a trajectory angle of \firstsecondSeparationgammaStandard $^\circ$. 
 This altitude of first stage-SPARTAN separation is 3.02km (+12.5\%) higher than the first stage-SPARTAN separation point with no fly-back, with a trajectory angle at separation which is +2.5$^\circ$ (+80.6\%) higher. 
 This higher release point requires less aerodynamic manoeuvring of the first stage, and enables the first stage to be efficiently launched with a higher fuel mass of 17943kg, an increase of +758kg (+4.4\%) compared to the trajectory without fly-back. This additional fuel increases the total acceleration of the first stage,
 in turn increasing the exergy efficiency of the first stage rocket by +0.308\%$\eta$ (+4.9\%) due to a higher propulsion efficiency, and allowing the first stage to achieve a higher velocity at separation (an increase of +64m/s, +4.3\%). 
 
  \begin{figure}[!ht]
  	\centering
  	\includegraphics[width=1\linewidth]{../LODESTAR_FINAL/Results/mode11/SecondStageStandard}
  	\caption{The acceleration of the SPARTAN flying an optimised maximum payload-to-orbit trajectory with SPARTAN fly-back (Case 11). }
  	\label{fig:SecondStageStandard}
  \end{figure}
 The higher altitude, larger trajectory angle, and increased velocity at the first stage-SPARTAN separation point causes an altitude raising manoeuvre at the beginning of the SPARTAN's acceleration, which is significantly higher than the altitude raising manoeuvre with no fly-back. This altitude raising manoeuvre takes the SPARTAN to an altitude of 29.59km at 31.44s, and decreases the dynamic pressure of the SPARTAN to 29.1kPa, allowing time for the bank angle of the SPARTAN to be increased. 
 After the first stage-SPARTAN separation, the bank angle is increased, at the maximum change rate, to 44.2$^\circ$, which aids the SPARTAN in decreasing its altitude. As the altitude of the SPARTAN begins to reduce, the bank angle stops increasing and the angle of attack is raised to 3.24$^\circ$ to increase lift, slowing the descent of the SPARTAN. 
 The bank angle then begins to increase once more, and as the SPARTAN reaches close to its maximum dynamic pressure at 109.8s, the bank angle reaches an initial maximum of 56.8$^\circ$. 

After this point, the bank angle of the SPARTAN is maintained between 50.4$^\circ$ and 58.6$^\circ$, exhibiting higher bank angles towards the latter part of the ascent. At the end of the SPARTAN's acceleration, the bank angle is reduced, so that the third stage is released at 0$^\circ$ bank angle. This 0$^\circ$ bank angle is defined as a constraint on the end of the trajectory, to ensure that the third stage rocket is released in the vertical plane, and is able to manoeuvre to orbit. 


The angle of attack of the SPARTAN is significantly higher over the course of the maximum payload-to-orbit trajectory with fly-back inclusion, compared to maximum payload-to-orbit trajectory with no fly-back, detailed in Section \ref{sec:optimisednoreturn}. These significantly  higher angles of attack are a result of the high bank angle of the SPARTAN throughout its trajectory, which cause the lift of the SPARTAN to be partially used for changing the heading of the SPARTAN, rather than providing vertical force. 
 The higher angles of attack result in the optimal trajectory of the SPARTAN following a close to maximum dynamic pressure path for most of the duration of its trajectory, without the altitude raising manoeuvre observed in Section \ref{sec:optimisednoreturn}.
 The increase in angle of attack means that the SPARTAN no longer flies within the homogeneous region of the specific impulse of the C-REST engines. instead the flight conditions are close to a region where an increase in angle of attack causes a sharp decrease in specific impulse, illustrated in Figure \ref{fig:NetIspStandard}. 
This indicates that at Mach 7 and 8, the angle of attack, and consequently the allowable bank angle, of the SPARTAN is being limited by the performance of the C-REST engines. 
 The SPARTAN stays close to its maximum dynamic pressure until a pull-up is performed at 365.8s flight time. 
 \begin{figure}[!ht]
 	\centering
 	\includegraphics[width=0.9\linewidth]{../LODESTAR_FINAL/Results/mode11/NetIspStandard}
 	\caption{Net $I_{SP}$ contours for the SPARTAN at Mach numbers between 6 and 8, showing the optimised trajectory path (Case 11). }
 	\label{fig:NetIspStandard}
 \end{figure}

The higher angles of attack flown by the SPARTAN also have the consequence of decreasing the net specific impulse of the SPARTAN during its acceleration, with the maximum specific impulse being decreased by -2.5\%.
The overall exergy efficiency of the SPARTAN is decreased, to \secondExergyEffStandard\%$\eta$, a decrease of -0.729\%$\eta$ (-15.4\%) compared to the maximum payload-to-orbit trajectory with no fly-back. This exergy efficiency decrease is due partially to the decrease in the specific impulse of the scramjet engines, but more significantly is attributed to the fuel necessary for the return flight resulting in less fuel being available for the ascent of the SPARTAN, and thus less `useful' work being attained from the total fuel mass.
A total fuel mass of 1294kg is used during the SPARTAN's acceleration, out of a total of 1562kg of available fuel. This reduction in fuel mass used, along with the reduction in net specific impulse due to the higher angle of attack values, reduces the velocity at SPARTAN-third stage separation by -106m/s (-3.9\%) compared to the maximum payload-to-orbit case with no SPARTAN fly-back. The SPARTAN pulls up to \secondthirdSeparationAltStandard km altitude and \secondthirdSeparationgammaStandard $^\circ$ trajectory angle before the SPARTAN-third stage separation, a difference of only -0.8km (-1.9\%) and +0.2$^\circ$ (+1.8\%) compared to the maximum payload-to-orbit trajectory without fly-back, indicating that the inclusion of fly-back does not have a large effect on the magnitude of the pull-up manoeuvre. 

The exergy efficiency of the third stage is decreased by -1.845\%$\eta$ (-9.8\%) when compared to the maximum payload-to-orbit trajectory with no SPARTAN fly-back. This lowered efficiency is primarily due to the lower velocity of the third stage release, which increases the losses of the third stage due to propulsive inefficiencies. 


\begin{figure}[ht!]
\centering
\includegraphics[width=1\linewidth]{../LODESTAR_FINAL/Results/mode11/ThirdStageStandard}
\caption{The third stage trajectory of an optimised maximum payload-to-orbit trajectory with SPARTAN fly-back (Case 11). }
\label{fig:ThirdStageStandard}
\end{figure}


\section{Fly-Back Trajectory}

\begin{figure}[ht!]
	\centering
	\includegraphics[width=1\linewidth]{../LODESTAR_FINAL/Results/mode11/ReturnStandard}
	\caption{The fly-back trajectory of the SPARTAN flying an optimised maximum payload-to-orbit trajectory (Case 11). }
	\label{fig:ReturnStandard}
\end{figure}

The optimised fly-back trajectory is shown in Figure \ref{fig:ReturnStandard}.
The SPARTAN is shown to be capable of fly-back, using \returnFuelStandard kg of fuel, 17.2\% of the total fuel.
Throughout its fly-back the SPARTAN performs distinct skipping manoeuvres, and ignites the scramjet engines a total of three times. 
These skips are consistent with previous research which has shown that a periodic skipping trajectory increases the downrange distance achievable by hypersonic vehicles both during powered and unpowered flight\cite{Moshman2014,Darby2011,Toso2015,Tetlow1992,Eggers1957,Kanda2007,Chai2015}, and serve to reduce the fuel necessary for the return flight. 

It is observed that the optimised trajectory exhibits characteristics which can be separated into three distinct segments; 1. initial turn, 2. boost-skip, and 3. approach, as indicated in Figure \ref{fig:ReturnStandard}. 
 
\subsubsection{ Initial Turn}
The SPARTAN separates from the third stage rocket at a bank angle of 0$^\circ$, and then increases its bank angle at close to the maximum change rate until 108.7s return flight time, at which point 81.7$^\circ$ bank angle is reached. This high bank angle serves to rapidly change the heading of the SPARTAN, in order to minimise the down-range distance flown, and reduce the fuel necessary for fly-back. 
The angle of attack is kept low during this time, in order to minimise the size of the initial skip. 
As the SPARTAN reaches the zenith of its initial skip, at 66.1s flight time and 60.0km altitude, the angle of attack is rapidly increased, up to a maximum of 8.76$^\circ$. 
This increase in angle of attack, along with the aid of a subsequent reduction in the bank angle to 67.5$^\circ$, generates additional lift to slow the descent of the SPARTAN into the trough of the first skip, ensuring that the dynamic pressure limit is not exceeded. 


\subsubsection{ Boost-Skip}\label{sec:boost}
At 182.8s flight time, the scramjet engines are ignited. The C-REST engines are powered-on in the trough between the first and second skips, at a point of high potential specific impulse, and initially burn for 22s. During the initial burn, the L/D of the SPARTAN increases significantly, due to the scramjet engine flow paths of the SPARTAN generating thrust, rather than drag. 
This increase in L/D raises the altitude of the SPARTAN and, along with the bank angle of 62.2$^\circ$, changes the heading of the SPARTAN significantly. 
The burn is limited by the lower inlet dynamic pressure limit of the C-REST engines, of 20kPa, and terminates at 204.8s flight time. After the initial burn ends, the angle of attack of the SPARTAN is decreased to 3.2$^\circ$, and the SPARTAN executes its second skip. Once the SPARTAN is descending again, and as soon as the dynamic pressure is high enough for C-REST engine operation at 339.9s return flight time, the scramjet engines are once again ignited.
During the second burn, the angle of attack of the SPARTAN is increased, to modify the temperature and Mach number at the inlet of the C-REST engines so that the maximum specific impulse is obtained from the C-REST engines during the burn. 
The angle of attack varies between 4.2$^\circ$ to 3.3$^\circ$ during the second burn, and the L/D is once again raised significantly, initiating the third skip. 
This skip raises the altitude of the SPARTAN to 54.6km, before it decreases once again. 
The third and last burn is initiated at 536.7s and lasts until 579.0s, when the remaining fuel has been depleted. Before the third burn, the angle of attack is decreased, so that it varies between 4.5$^\circ$ and 3.7$^\circ$ during the burn. These angle of attack values are similar to those observed during the second burn, indicating that these angle of attack values obtain a high specific impulse from the C-REST engines, this can be observed in Figure \ref{fig:returnIspStandard}, which shows the specific impulse profile of the return flight during the boost-skip phase. 

After the third burn phase, the angle of attack is initially controlled so that the skipping trajectory of the SPARTAN is dampened.
Immediately after the third burn phase, the angle of attack is reduced, to 2.82$^\circ$. This reduction coincides with the ascent portion of the fourth skip, reducing the lift, and the amount of altitude gained. 
As the zenith of the forth skip is reached, the angle of attack is increased to 7.2$^\circ$, increasing the lift, and once again slowing the descent. 
This high angle of attack is sustained until 748.2s at which point the angle of attack is reduced again significantly, to 2.6$^\circ$, reducing the size of the fifth skip. At 871.2s, the angle of attack is again raised, to 5$^\circ$, initiating the sixth and last skip.
It is notable that the sixth skip is initiated in this way, as previously in the unpowered portion of the trajectory the angle of attack is being utilised to damped the skipping motion. This indicates that some degree of skipping is desirable after the final scramjet burn, and that the angle of attack is being controlled to produce optimally sized skips. 

\subsubsection{ Approach}

After the final small skip, at 993.3s flight time, the angle of attack is adjusted, so that a gradual, controlled descent is initiated. 
After the skip phase, as the vehicle is approaching Mach 1, the angle of attack is reduced gradually to bring the SPARTAN down to 1km altitude, in a controlled manner. At 1227.0s, the bank angle is increased, in order to perform a final adjustment of the heading angle, to bring the SPARTAN to the desired end location. 
The SPARTAN reaches 1km altitude at -26.7$^\circ$ trajectory angle and 120.0m/s velocity (Mach 0.356). It is assumed that the SPARTAN is able to perform a landing manoeuvre after this point. 





\section{Energy Usage Analysis}


\begin{table}[ht]
	\centering
	\begin{tabular}{l c c} 
		\hline \textbf{Trajectory Condition}
		& No Fly-Back
		& With Fly-Back
		\\
		\textbf{First Stage Fuel Exergy} 
		&\textbf{\firstEnergyStandardNoReturn} GJ
		&\textbf{\firstEnergyStandard} GJ
		\\
		
		\textcolor{blue}{KE + PE of Payload}
		& \firstWpayloadStandardNoReturn \% (0.25 GJ)
		& \firstWpayloadStandard \% (0.25 GJ)
		\\
		\textcolor{red}{KE + PE of  2$^{nd}$ \& 3$^{rd}$ Stage}
		& \firstWnextStageStandardNoReturn \% (12.87 GJ) & \firstWnextStageStandard \% (14.12 GJ)
		\\
		\textcolor{red}{Overcoming Drag} 
		& \WDoneStandardNoReturn \% (3.44 GJ) & \WDoneStandard \% (2.92 GJ)
		\\
		\textcolor{red}{KE + PE of 1$^{st}$ Stage Structural Mass} 
		& \WoneStandardNoReturn \% (2.19 GJ) & \WoneStandard \% (2.39 GJ)
		\\
		
		
		\textcolor{red}{Propulsion Inefficiency} 
		& \PlossoneCombinedStandardNoReturn \% (189.78 GJ) & \PlossoneCombinedStandard \% (197.96 GJ)
		\\ 
		\textbf{SPARTAN Fuel Exergy} 
		& \textbf{\secondEnergyStandardNoReturn} GJ & \textbf{\secondEnergyStandard} GJ
		\\
		\textcolor{blue}{KE + PE of Payload}
		& \secondWpayloadStandardNoReturn \% (0.51 GJ) & \secondWpayloadStandard \% (0.38J)
		\\
		\textcolor{red}{KE + PE of 3$^{rd}$ Stage}
		& \secondWnextStageStandardNoReturn \% (8.33 GJ) & \secondWnextStageStandard \% (7.09 GJ)
		\\
		\textcolor{red}{Overcoming Drag}
		& \WDsecondStandardNoReturn \% (36.76 GJ) & \WDsecondStandard \% (31.56 GJ)
		\\
		\textcolor{red}{KE + PE of SPARTAN Structural Mass}  
		& \WsecondStandardNoReturn \% (13.28 GJ) & \WsecondStandard \% (11.22 GJ)
		\\
		\textcolor{red}{Propulsion Inefficiency}  
		& \PlosssecondCombinedStandardNoReturn \% (128.51 GJ) & \PlosssecondCombinedStandard \% (104.97 GJ)
		\\
		
		
		
		\textbf{Return Fuel Exergy} 
		& - & \textbf{\returnEnergyStandard} GJ
		\\
		KE + PE of SPARTAN Structural Mass
		& - & \WreturnStandard \% (18.40 GJ)
		\\
		\textcolor{red}{Overcoming Drag}
		& - & \WDreturnStandard \% (30.94 GJ)
		\\
		
		\textcolor{red}{Propulsion Inefficiency}  
		& - & \PlossreturnCombinedStandard \% (19.61 GJ)
		\\
		
		
		\textbf{Third Stage Fuel Exergy}  
		& \textbf{\thirdEnergyStandardNoReturn} GJ & \textbf{\thirdEnergyStandard} GJ
		\\
		\textcolor{blue}{KE + PE of Payload}  
		&\thirddExergyEffStandardNoReturn \% (6.42 GJ) &\thirddExergyEffStandard \% (5.83 GJ)
		\\
		\textcolor{red}{Overcoming Drag}  
		& \WDthreeStandardNoReturn \% (0.20 GJ) & \WDthreeStandard \% (0.22 GJ)
		\\
		\textcolor{red}{KE + PE  of 3$^{rd}$ Stage Structural Mass}  
		& \WthreeStandardNoReturn \% (9.56 GJ) & \WthreeStandard \% (9.63 GJ)
		\\
		
		
		
		\textcolor{red}{KE + PE of Heat Shield}  
		
		& \WHSthreeStandardNoReturn \% (1.02 GJ) & \WHSthreeStandard \% (1.04 GJ)
		\\
		
		\textcolor{red}{Propulsion Inefficiency}  
		& \PlossthreeCombinedStandardNoReturn \% (17.09 GJ) & \PlossthreeCombinedStandard \% (17.80 GJ)
		\\
		\hline 
	\end{tabular} 
	\caption{An energy usage breakdown of the ascent trajectories, both with, and without, SPARTAN fly-back (Cases 11 \& 2). Blue indicates a 'productive' energy usage, whereas red indicates energy 'wastage'. Negative energy indicates energy being supplied.}
	\label{tab:effStandard}
\end{table}


An energy usage analysis is conducted for a maximum payload-to-orbit trajectory, including the fly-back of the SPARTAN. This is compared to the energy usage breakdown of the optimised trajectory without the fly-back of the SPARTAN in Table \ref{tab:effStandard}. Similarly to Section \ref{sec:exergy1}, the energy used to accelerate the payload is shown, along with the energy imparted to the successive stages; the energy used overcoming drag; the energy used imparting energy to the structural mass of each stage, which is separated; and the energy lost due to propulsion inefficiency. 


The fly-back of the SPARTAN reduces the fuel, and thus the fuel exergy, available to the SPARTAN during ascent.
This lower exergy, along with the altered manoeuvrability needs of the SPARTAN when the fly-back is included, causes the altitude and trajectory angle at the first stage-SPARTAN separation to be raised. The increased altitude and trajectory angle at separation increases the fuel mass that the first stage rocket is able to use efficiently, and also increases the exergy efficiency of the first stage, partly compensating for the decrease in the efficiency of the SPARTAN due to fly-back. Overall, when the fly-back is included, more of the exergy of the first stage is utilised imparting energy upon the combination of the payload and the successive stages, at 6.6\% (14.37GJ), compared to 6.292\% ( 13.12GJ) without SPARTAN fly-back. This is due to the rocket flying a more efficient trajectory, with lower drag and propulsive losses, terminating at a higher altitude and velocity.

When the fly-back is included, the SPARTAN-third stage separation occurs at a lower altitude and velocity, and the lower fuel exergy of the SPARTAN during its ascent results in less energy being imparted upon the payload and third stage by the SPARTAN (7.47GJ), compared to the trajectory without fly-back (8.84GJ). 
The lower, slower separation point when fly-back is included causes the losses of the third stage to increase from all sources. The propulsive inefficiency losses are particularly affected, increasing from \PlossthreeCombinedStandardNoReturn \% (17.09 GJ) without fly-back to \PlossthreeCombinedStandard \% (17.80 GJ) with fly-back, due to the lower velocity of the separation point, which decreases the propulsive efficiency of the third stage (illustrated by Equation \ref{eq:rocketeff}).
When flying a trajectory where the SPARTAN's fly-back is included, the drag losses during the ascent of the SPARTAN consist of a larger percentage of the ascent fuel exergy usage  (\WDsecondStandard \%, compared to \WDsecondStandardNoReturn \% without fly-back). This is despite the lower velocity range over which the SPARTAN is accelerating when fly-back is included, and is due to the less favourable first stage-SPARTAN separation conditions, as well as the high banking throughout the acceleration. 


The energy necessary to return the SPARTAN to its initial launch location is provided by both the fuel used during the return (32.15GJ), as well as the kinetic and potential energy imparted upon the SPARTAN during its ascent (18.40GJ). Significantly more energy is required to overcome drag during the return (30.94GJ) than is available from the kinetic and potential energy of the SPARTAN (18.40 GJ), illustrating the necessity for igniting the scramjet engines during the return flight. 





\section{Design Sensitivity Analysis}\label{sec:sensitivity}

It has been shown that the fly-back of the SPARTAN accelerator has a significant effect on the performance of the rocket-scramjet-rocket launch system, and that the maximum payload-to-orbit optimised trajectory changes significantly to compensate for the additional requirement of successfully returning the SPARTAN stage. This section investigates the sensitivity of the launch system to changes in the vehicle design, with the fly-back of the SPARTAN included. This sensitivity study varies the following:
\begin{itemize}
	\item Case 12: Dynamic Pressure, 
	\item Case 13: Specific Impulse,
	\item Case 14: SPARTAN Drag,
	\item Case 15: SPARTAN Mass,
	\item Case 16: SPARTAN Fuel Mass,
	\item Case 17: Third Stage Mass,
	\item Case 18: Third Stage Thrust.
\end{itemize}
As in Section \ref{sec:sensitivityNoReturn}, the effect of third stage drag is negligible. For this reason, variation in the third stage drag is omitted from this study. 

The launch system is able to successfully place a small satellite in orbit for every performance condition which has been tested, while returning the SPARTAN to its initial launch location for landing. 
Every maximum payload-to-orbit optimised trajectory exhibits considerable banking during the SPARTAN's ascent trajectory, as well as a pull-up of the SPARTAN before third stage release. 
In every case the optimised return flight path exhibits initial turn, boost-skip and approach phases. 
However, the height, and duration of the second skip of the return phase varies between cases, exhibiting no clear trend across the majority of the sensitivity studies which have been performed.

The first stage-SPARTAN separation angle and altitude show no clear trend in any of the sensitivity studies performed, except for the third stage mass variation, in contrast to the sensitivity studies with no fly-back, detailed in Section \ref{sec:sensitivityNoReturn}, in which the SPARTAN mass and drag parameters change the first stage separation point significantly. All of the optimised trajectory solutions show a distinct initial altitude raising manoeuvre performed by the SPARTAN, however, the size is inconsistent across optimised trajectory solutions, indicating that this manoeuvre is no longer solely a product of an efficiency trade-off between the first stage pitching and SPARTAN engine efficiency.
In the maximum payload-to-orbit optimised trajectories calculated during the sensitivity analysis, it is observed that the trajectory angle at the first stage-SPARTAN separation varies significantly between the optimised trajectories, with no discernible trend. When the SPARTAN is released at a high trajectory angle, the first stage is able to use more fuel, and fly a more efficient trajectory. In contrast to the trajectory with no fly-back,
releasing the SPARTAN at a higher trajectory angle and altitude causes it to spend a significant amount of time in a low dynamic pressure environment, giving time for the bank angle to increase. The high bank angle is utilised during the descent of the SPARTAN onto the maximum dynamic pressure path, to rapidly change the heading of the SPARTAN. This mitigates some of the reduction in efficiency caused by a higher first stage-SPARTAN separation point.
 A lower release angle results in the first stage flying a slightly less efficient trajectory. However, a lower release angle also results in the SPARTAN using its fuel more rapidly, and manoeuvring more at the beginning of its trajectory, which results in the fly-back requiring less fuel. 
The trade-off between first stage efficiency and the initial operational efficiency of the SPARTAN appears to be close, and 
for each particular trajectory optimisation one or the other is favoured with no clear trend. 


It is also observed that there are two distinct return trajectory shapes for the return trajectory of the SPARTAN. The more common return trajectory shape has been shown in the preceding section, and consists of three or more large skips to begin the return trajectory. The second trajectory shape exhibits a small second skip, with the first two burns very closely spaced, or combined into one longer burn. An example of this second type of return trajectory is shown in Figure \ref{fig:ReturnComparison10}. During the first two burns, a higher bank angle is maintained when compared to the large skip trajectory shape, however, after the first two burns are completed, the bank angle is reduced more rapidly. 
During simulations, it was observed that on occasion, the optimal return trajectory type would change as the initial guess or problem setup was altered, with no significant change in the payload-to-orbit capabilities of the launch system. This variability suggests that there is minimal difference between the two shapes of return trajectory, and that both can potentially lead to efficient return flights. 





\subsection{Case 12: Maximum Dynamic Pressure Sensitivity with Fly-Back}

\begin{table}[ht]
	\centering
	\begin{tabular}{l c c c c c c} 
		\hline \textbf{Trajectory Condition}   \qquad  $q_{max}$:
		&40kPa
		&45kPa
		&50kPa
		&55kPa
		&60kPa
		& $\Delta/\Delta$/\%$q_{max}$
		\\
		\hline \textbf{Payload to Orbit (kg)}
		& \textbf{\PayloadToOrbitqForty}
		& \textbf{\PayloadToOrbitqFortyFive}
		& \textbf{\PayloadToOrbitqStandard}
		& \textbf{\PayloadToOrbitqFiftyFive}
		& \textbf{\PayloadToOrbitqSixty}
		&\textbf{0.4}
		\\
		\textbf{Payload Variation (\%)}
		& \PayloadVarqForty
		& \PayloadVarqFortyFive
		& \PayloadVarqStandard
		& \PayloadVarqFiftyFive
		& \PayloadVarqSixty
		&0.25
		\\
		\textbf{Total $\eta_{exergy}$ (\%)}
		& \textbf{\totalExergyEffqForty}
		& \textbf{\totalExergyEffqFortyFive}
		& \textbf{\totalExergyEffqStandard}
		& \textbf{\totalExergyEffqFiftyFive}
		& \textbf{\totalExergyEffqSixty}
		& \textbf{3e-05}
		\\
		\hline 
		\textbf{1$^{st}$ Stage $\eta_{exergy}$ (\%)}
		& \textbf{\firstExergyEffqForty}
		& \textbf{\firstExergyEffqFortyFive}
		& \textbf{\firstExergyEffqStandard}
		& \textbf{\firstExergyEffqFiftyFive}
		& \textbf{\firstExergyEffqSixty}
		& -
		\\
		\textbf{Separation Alt, 1$\rightarrow$2 (km)}
		& \firstsecondSeparationAltqForty
		& \firstsecondSeparationAltqFortyFive
		& \firstsecondSeparationAltqStandard
		& \firstsecondSeparationAltqFiftyFive
		& \firstsecondSeparationAltqSixty
		& -
		\\
		\textbf{Separation v, 1$\rightarrow$2 (m/s)}
		& \firstsecondSeparationvqForty
		& \firstsecondSeparationvqFortyFive
		& \firstsecondSeparationvqStandard
		& \firstsecondSeparationvqFiftyFive
		& \firstsecondSeparationvqSixty
		& -
		\\
		\textbf{Separation $\gamma$, 1$\rightarrow$2 (deg)}
		& \firstsecondSeparationgammaqForty
		& \firstsecondSeparationgammaqFortyFive
		& \firstsecondSeparationgammaqStandard
		& \firstsecondSeparationgammaqFiftyFive
		& \firstsecondSeparationgammaqSixty
		& -
		\\
		\hline 
		\textbf{2$^{nd}$ Stage $\eta_{exergy}$ (\%)}
		& \textbf{\secondExergyEffqForty}
		& \textbf{\secondExergyEffqFortyFive}
		& \textbf{\secondExergyEffqStandard}
		& \textbf{\secondExergyEffqFiftyFive}
		& \textbf{\secondExergyEffqSixty}
		& -
		\\
		\textbf{Separation Alt, 2$\rightarrow$3 (km)}
		& \secondthirdSeparationAltqForty
		& \secondthirdSeparationAltqFortyFive
		& \secondthirdSeparationAltqStandard
		& \secondthirdSeparationAltqFiftyFive
		& \secondthirdSeparationAltqSixty
		&-0.02
		\\
		\textbf{Separation $v$, 2$\rightarrow$3 (m/s)}
		& \secondthirdSeparationvqForty
		& \secondthirdSeparationvqFortyFive
		& \secondthirdSeparationvqStandard
		& \secondthirdSeparationvqFiftyFive
		& \secondthirdSeparationvqSixty
		&1.99
		\\
		\textbf{Separation $\gamma$, 2$\rightarrow$3 (deg)}
		& \secondthirdSeparationgammaqForty
		& \secondthirdSeparationgammaqFortyFive
		& \secondthirdSeparationgammaqStandard
		& \secondthirdSeparationgammaqFiftyFive
		& \secondthirdSeparationgammaqSixty
		& -
		\\
	
		\textbf{2$^{nd}$ Stage Distance Flown (km)}
		& \SecondDistqForty
		& \SecondDistqFortyFive
		& \SecondDistqStandard
		& \SecondDistqFiftyFive
		& \SecondDistqSixty
		&-3.99
		\\
		\textbf{2$^{nd}$ Stage Return Fuel (kg)}
		& \returnFuelqForty
		& \returnFuelqFortyFive
		& \returnFuelqStandard
		& \returnFuelqFiftyFive
		& \returnFuelqSixty
		& -
		\\
		\textbf{2$^{nd}$ Stage Return Distance (km)}
		& \returnDistqForty
		& \returnDistqFortyFive
		& \returnDistqStandard
		& \returnDistqFiftyFive
		& \returnDistqSixty
		&-5.98
		\\
		\hline 
		\textbf{3$^{rd}$ Stage $\eta_{exergy}$ (\%)}
		& \textbf{\thirddExergyEffqForty}
		& \textbf{\thirddExergyEffqFortyFive}
		& \textbf{\thirddExergyEffqStandard}
		& \textbf{\thirddExergyEffqFiftyFive}
		& \textbf{\thirddExergyEffqSixty}
		& \textbf{0.042}
		\\
		\textbf{3$^{rd}$ Stage $t$, $q >$ 5kpa (s)}
		& \thirdqOverFiveqForty
		& \thirdqOverFiveqFortyFive
		& \thirdqOverFiveqStandard
		& \thirdqOverFiveqFiftyFive
		& \thirdqOverFiveqSixty
		& -
		\\
		\textbf{3$^{rd}$ Stage max $\alpha$ (deg)}
		& \thirdmaxAoAqForty
		& \thirdmaxAoAqFortyFive
		& \thirdmaxAoAqStandard
		& \thirdmaxAoAqFiftyFive
		& \thirdmaxAoAqSixty
		&0
		\\
		\textbf{3$^{rd}$ Stage Fuel Mass (kg)}
		& \thirdmFuelqForty
		& \thirdmFuelqFortyFive
		& \thirdmFuelqStandard
		& \thirdmFuelqFiftyFive
		& \thirdmFuelqSixty
		&-0.42
		\\
		\hline 
	\end{tabular} 
	\caption{Comparison of key trajectory parameters with variation in the maximum dynamic pressure of the SPARTAN, with fly-back (Case 12).}
	\label{tab:qvarreturn}
\end{table}


The maximum dynamic pressure allowable during flight is varied by $\pm$20\% in order to determine the sensitivity of the launch system to the structural and thermal limitations of the SPARTAN.  
Table \ref{tab:qvarreturn} shows a summary of the key parameters of each optimised trajectory, and trajectory comparison plots are shown in Appendix \ref{sec:app_comparison21}. The variation in each trajectory parameter per \% of the dynamic pressure is shown, if there is a clear trend. The payload-to-orbit of the launch system improves by +9.4kg (+5.51\%) at 60kPa, and decreases by -7.7kg (-4.52\%) at 40kPa.
The overall exergy efficiency of the system increases as the maximum dynamic pressure increases, by +0.083\%$\eta$ at 60kPa, and decreases as the maximum dynamic pressure decreases, by -0.067\%$\eta$ at 40kpa. 
No significant variation is observed between sensitivity studies with or without the fly-back included in the sensitivity of the launch system to the maximum dynamic pressure of the SPARTAN, by percentage.

When fly-back is included, no trends are observed in the exergy efficiencies of the first stage or SPARTAN, due to maximum dynamic pressure variation. Compared to the sensitivity study with no fly-back, the trade-offs between the efficiencies of the stages include the manoeuvrability of the SPARTAN, which dictates the fuel used during the return flight. This additional factor produces more complicated energy trade-offs, resulting in differing optimal trajectory shapes. This can particularly be observed in the 45kPa maximum dynamic pressure trajectory, which exhibits significantly different trade-offs between each stage, when compared to the other cases. 
 The 45kPa maximum dynamic pressure simulation shows a low exergy efficiency for the first stage than would be suggested by the general exergy efficiency trends. 
 However, the 45kPa simulation trades the performance of the first stage to achieve greater manoeuvrability at the beginning of the SPARTAN's trajectory, resulting in less fuel being used during fly-back, and a higher overall exergy efficiency for the SPARTAN. The first stage-SPARTAN separation occurs at a lower altitude and trajectory angle compared to the other simulations, allowing the acceleration to be achieved more quickly at the start of the trajectory, and enabling the SPARTAN to manoeuvre more effectively at the beginning of its trajectory. 
 As a consequence, the 45kPa simulation uses only 257.0kg of fuel during the fly-back, against the general trend of the return fuel usage. 

The exergy efficiency of the SPARTAN exhibits a higher general sensitivity to variations in the maximum dynamic pressure, when fly back is included.
Increasing the maximum dynamic pressure improves the manoeuvring capabilities of the SPARTAN and increases the acceleration rate during ascent, which leads to a smaller flight time, and less ground coverage, generally reducing the amount of fuel necessary for fly-back, once more with the exception of the 45kPa case. This is a factor not present in the sensitivity study without fly-back, however, this does not significantly impact on the overall payload-to-orbit sensitivity of the launch system. 




\subsection{Case 13: SPARTAN Drag Sensitivity with Fly-Back}\label{sec:dragvar}


\begin{table}[ht]
	\centering
	\begin{tabular}{l c c c c c c} 
		\hline \textbf{Trajectory Condition}   \qquad  $C_{d,2}$:
		&90\%
		&95\%
		&100\%
		&105\%
		&110\%
		& $\Delta/\Delta$/\%$C_{d,2}$
		\\
		\hline \textbf{Payload to Orbit (kg)}
		& \textbf{\PayloadToOrbitCdNinety}
		& \textbf{\PayloadToOrbitCdNinetyFive}
		& \textbf{\PayloadToOrbitCdStandard}
		& \textbf{\PayloadToOrbitCdOneHundredFive}
		& \textbf{\PayloadToOrbitCdOneHundredTen}
		&\textbf{-1.5}
		\\
		\textbf{Payload Variation (\%)}
		& \PayloadVarCdNinety
		& \PayloadVarCdNinetyFive
		& \PayloadVarCdStandard
		& \PayloadVarCdOneHundredFive
		& \PayloadVarCdOneHundredTen
		&-0.9
		\\
		\textbf{Total $\eta_{exergy}$ (\%)}
		& \textbf{\totalExergyEffCdNinety}
		& \textbf{\totalExergyEffCdNinetyFive}
		& \textbf{\totalExergyEffCdStandard}
		& \textbf{\totalExergyEffCdOneHundredFive}
		& \textbf{\totalExergyEffCdOneHundredTen}
		& \textbf{-0.00013}
		\\
		\hline 
		\textbf{1$^{st}$ Stage $\eta_{exergy}$ (\%)}
		& \textbf{\firstExergyEffCdNinety}
		& \textbf{\firstExergyEffCdNinetyFive}
		& \textbf{\firstExergyEffCdStandard}
		& \textbf{\firstExergyEffCdOneHundredFive}
		& \textbf{\firstExergyEffCdOneHundredTen}
		& \textbf{-0.025}
		\\
		\textbf{Separation Alt, 1$\rightarrow$2 (km)}
		& \firstsecondSeparationAltCdNinety
		& \firstsecondSeparationAltCdNinetyFive
		& \firstsecondSeparationAltCdStandard
		& \firstsecondSeparationAltCdOneHundredFive
		& \firstsecondSeparationAltCdOneHundredTen
		& -
		\\
		\textbf{Separation v, 1$\rightarrow$2 (m/s)}
		& \firstsecondSeparationvCdNinety
		& \firstsecondSeparationvCdNinetyFive
		& \firstsecondSeparationvCdStandard
		& \firstsecondSeparationvCdOneHundredFive
		& \firstsecondSeparationvCdOneHundredTen
		&-3.66
		\\
		\textbf{Separation $\gamma$, 1$\rightarrow$2 (deg)}
		& \firstsecondSeparationgammaCdNinety
		& \firstsecondSeparationgammaCdNinetyFive
		& \firstsecondSeparationgammaCdStandard
		& \firstsecondSeparationgammaCdOneHundredFive
		& \firstsecondSeparationgammaCdOneHundredTen
		& -
		\\
		\hline 
		\textbf{2$^{nd}$ Stage $\eta_{exergy}$ (\%)}
		& \textbf{\secondExergyEffCdNinety}
		& \textbf{\secondExergyEffCdNinetyFive}
		& \textbf{\secondExergyEffCdStandard}
		& \textbf{\secondExergyEffCdOneHundredFive}
		& \textbf{\secondExergyEffCdOneHundredTen}
		& \textbf{-0.034}
		\\
		\textbf{Separation Alt, 2$\rightarrow$3 (km)}
		& \secondthirdSeparationAltCdNinety
		& \secondthirdSeparationAltCdNinetyFive
		& \secondthirdSeparationAltCdStandard
		& \secondthirdSeparationAltCdOneHundredFive
		& \secondthirdSeparationAltCdOneHundredTen
		&-0.04
		\\
		\textbf{Separation $v$, 2$\rightarrow$3 (m/s)}
		& \secondthirdSeparationvCdNinety
		& \secondthirdSeparationvCdNinetyFive
		& \secondthirdSeparationvCdStandard
		& \secondthirdSeparationvCdOneHundredFive
		& \secondthirdSeparationvCdOneHundredTen
		&-9.5
		\\
		\textbf{Separation $\gamma$, 2$\rightarrow$3 (deg)}
		& \secondthirdSeparationgammaCdNinety
		& \secondthirdSeparationgammaCdNinetyFive
		& \secondthirdSeparationgammaCdStandard
		& \secondthirdSeparationgammaCdOneHundredFive
		& \secondthirdSeparationgammaCdOneHundredTen
		& -
		\\
	
		\textbf{2$^{nd}$ Stage Distance Flown (km)}
		& \SecondDistCdNinety
		& \SecondDistCdNinetyFive
		& \SecondDistCdStandard
		& \SecondDistCdOneHundredFive
		& \SecondDistCdOneHundredTen
		& -
		\\
		\textbf{2$^{nd}$ Stage Return Fuel (kg)}
		& \returnFuelCdNinety
		& \returnFuelCdNinetyFive
		& \returnFuelCdStandard
		& \returnFuelCdOneHundredFive
		& \returnFuelCdOneHundredTen
		& -
		\\
		\textbf{2$^{nd}$ Stage Return Distance (km)}
		& \returnDistCdNinety
		& \returnDistCdNinetyFive
		& \returnDistCdStandard
		& \returnDistCdOneHundredFive
		& \returnDistCdOneHundredTen
		& -
		\\
		\hline 
		\textbf{3$^{rd}$ Stage $\eta_{exergy}$ (\%)}
		& \textbf{\thirddExergyEffCdNinety}
		& \textbf{\thirddExergyEffCdNinetyFive}
		& \textbf{\thirddExergyEffCdStandard}
		& \textbf{\thirddExergyEffCdOneHundredFive}
		& \textbf{\thirddExergyEffCdOneHundredTen}
		& \textbf{-0.149}
		\\
		\textbf{3$^{rd}$ Stage $t$, $q >$ 5kpa (s)}
		& \thirdqOverFiveCdNinety
		& \thirdqOverFiveCdNinetyFive
		& \thirdqOverFiveCdStandard
		& \thirdqOverFiveCdOneHundredFive
		& \thirdqOverFiveCdOneHundredTen
		& -
		\\
		\textbf{3$^{rd}$ Stage max $\alpha$ (deg)}
		& \thirdmaxAoACdNinety
		& \thirdmaxAoACdNinetyFive
		& \thirdmaxAoACdStandard
		& \thirdmaxAoACdOneHundredFive
		& \thirdmaxAoACdOneHundredTen
		& -
		\\
		\textbf{3$^{rd}$ Stage Fuel Mass (kg)}
		& \thirdmFuelCdNinety
		& \thirdmFuelCdNinetyFive
		& \thirdmFuelCdStandard
		& \thirdmFuelCdOneHundredFive
		& \thirdmFuelCdOneHundredTen
		&1.53
		\\
		\hline 
	\end{tabular} 
	\caption{Comparison of key trajectory parameters with variation in the drag of the SPARTAN, with fly-back (Case 13).}
	\label{tab:comparison41}
\end{table}


The coefficient of drag is varied by $\pm$10\% to investigate the effect of variation in the aerodynamic design of the SPARTAN on the performance of the launch system, when the fly-back of the SPARTAN is included. Appendix \ref{sec:app_comparison41} presents trajectory comparison plots, and Table \ref{tab:comparison41} compares key parameters of each trajectory. 
Increasing the drag of the SPARTAN by 10\% decreases the payload-to-orbit by -14.0kg (-8.2\%), while decreasing the drag by 10\% increases the payload-to-orbit by +18.3kg (+10.8\%). 
The sensitivity to variations in the SPARTAN's aerodynamics is decreased when compared to the sensitivity study with no fly-back, down to -1.5 $\frac{\Delta kg}{\Delta\% C_{d}}$ (-0.9$\frac{\Delta \%}{\Delta\% C_{d}}$) compared to -1.9$\frac{\Delta kg}{\Delta\% C_{d}}$  (-0.99$\frac{\Delta \%}{\Delta\% C_{d}}$). 
This is due to the increased drag decreasing the total acceleration, which in turn generally decreases the ground distance necessary to cover during the fly-back, partially offsetting the detrimental effects of the increased drag on the performance of the launch system. The 95\% drag case is the exception to the trend in fly-back distance. This case exhibits different trade-offs between the ascent and fly-back of the SPARTAN, manoeuvring less so that a more efficient ascent trajectory is flown, while requiring a longer, and less fuel efficient fly-back. 

The exergy efficiencies of all three stages are decreased significantly as the drag of the SPARTAN is increased. This decrease in efficiency is due to the increased drag losses of the first stage and SPARTAN, \WDoneCdOneHundredTen\% and \WDsecondCdOneHundredTen\% respectively at 110\%$C_D$, compared to \WDoneCdNinety\% and \WDsecondCdNinety\% respectively at 90\%$C_D$, and the increased propulsive inefficiency losses of the third stage when released from a lower velocity, \PlossthreeCombinedCdOneHundredTen\% at 110\%$C_D$, compared to \PlossthreeCombinedCdNinety \% at 90\%$C_D$.
As was observed in the drag sensitivity study with no fly-back in Section \ref{sec:dragvariation}, the SPARTAN-third stage separation angle shows a general increase as the drag is increased, increasing by +0.7$^\circ$ (+6.4\%) at 110\% drag, and decreasing by -0.2$^\circ$ (-1.8\%) at 90\% drag. In addition, the altitude of the SPARTAN-third stage separation shows a clear trend, decreasing slightly as the drag of the SPARTAN is increased, by -0.1km (-0.24\%) at 110\% drag, and increasing slightly as the drag is decreased, by +0.79km (+1.93\%) at 90\% drag.  
The release altitude and trajectory angle serve to initiate the first skip of the return trajectory in a consistent manner, so that the shape of the initial skip is very similar with drag variation. In all cases the angle of attack is reduced to 0$^\circ$ immediately during return to lessen the size of the initial skip, and is then raised to close to the maximum of 10$^\circ$ to prevent the dynamic pressure limit being exceeded. This consistency indicates that the initial skip of the return flight is driving the conditions at SPARTAN-third stage release, and that it is primarily the control and structural limitations, rather than the aerodynamics of the SPARTAN, which determine the shape of this skip. 


\subsection{Case 14: C-REST Specific Impulse Sensitivity with Fly-Back}


\begin{table}[ht]
	\centering
	\begin{tabular}{l c c c c c c} 
		\hline \textbf{Trajectory Condition}   \qquad  $I_{SP,2}$:
		&90\%
		&90\%
		&100\%
		&105\%
		&110\%
		& $\Delta/\Delta$\%$I_{SP,2}$
		\\
		\hline \textbf{Payload to Orbit (kg)}
		& \textbf{\PayloadToOrbitIspNinety}
		& \textbf{\PayloadToOrbitIspNinetyFive}
		& \textbf{\PayloadToOrbitIspStandard}
		& \textbf{\PayloadToOrbitIspOneHundredFive}
		& \textbf{\PayloadToOrbitIspOneHundredTen}
		&\textbf{1.7}
		\\
		\textbf{Payload Variation (\%)}
		& \PayloadVarIspNinety
		& \PayloadVarIspNinetyFive
		& \PayloadVarIspStandard
		& \PayloadVarIspOneHundredFive
		& \PayloadVarIspOneHundredTen
		&0.98
		\\
		\textbf{Total $\eta_{exergy}$ (\%)}
		& \textbf{\totalExergyEffIspNinety}
		& \textbf{\totalExergyEffIspNinetyFive}
		& \textbf{\totalExergyEffIspStandard}
		& \textbf{\totalExergyEffIspOneHundredFive}
		& \textbf{\totalExergyEffIspOneHundredTen}
		& \textbf{0.00015}
		\\
		\hline 
		\textbf{1$^{st}$ Stage $\eta_{exergy}$ (\%)}
		& \textbf{\firstExergyEffIspNinety}
		& \textbf{\firstExergyEffIspNinetyFive}
		& \textbf{\firstExergyEffIspStandard}
		& \textbf{\firstExergyEffIspOneHundredFive}
		& \textbf{\firstExergyEffIspOneHundredTen}
		& -
		\\
		\textbf{Separation Alt, 1$\rightarrow$2 (km)}
		& \firstsecondSeparationAltIspNinety
		& \firstsecondSeparationAltIspNinetyFive
		& \firstsecondSeparationAltIspStandard
		& \firstsecondSeparationAltIspOneHundredFive
		& \firstsecondSeparationAltIspOneHundredTen
		& -
		\\
		\textbf{Separation v, 1$\rightarrow$2 (m/s)}
		& \firstsecondSeparationvIspNinety
		& \firstsecondSeparationvIspNinetyFive
		& \firstsecondSeparationvIspStandard
		& \firstsecondSeparationvIspOneHundredFive
		& \firstsecondSeparationvIspOneHundredTen
		& -
		\\
		\textbf{Separation $\gamma$, 1$\rightarrow$2 (deg)}
		& \firstsecondSeparationgammaIspNinety
		& \firstsecondSeparationgammaIspNinetyFive
		& \firstsecondSeparationgammaIspStandard
		& \firstsecondSeparationgammaIspOneHundredFive
		& \firstsecondSeparationgammaIspOneHundredTen
		& -
		\\
		\hline 
		\textbf{2$^{nd}$ Stage $\eta_{exergy}$ (\%)}
		& \textbf{\secondExergyEffIspNinety}
		& \textbf{\secondExergyEffIspNinetyFive}
		& \textbf{\secondExergyEffIspStandard}
		& \textbf{\secondExergyEffIspOneHundredFive}
		& \textbf{\secondExergyEffIspOneHundredTen}
		& \textbf{0.051}
		\\
		\textbf{Separation Alt, 2$\rightarrow$3 (km)}
		& \secondthirdSeparationAltIspNinety
		& \secondthirdSeparationAltIspNinetyFive
		& \secondthirdSeparationAltIspStandard
		& \secondthirdSeparationAltIspOneHundredFive
		& \secondthirdSeparationAltIspOneHundredTen
		& -
		\\
		\textbf{Separation $v$, 2$\rightarrow$3 (m/s)}
		& \secondthirdSeparationvIspNinety
		& \secondthirdSeparationvIspNinetyFive
		& \secondthirdSeparationvIspStandard
		& \secondthirdSeparationvIspOneHundredFive
		& \secondthirdSeparationvIspOneHundredTen
		&11.13
		\\
		\textbf{Separation $\gamma$, 2$\rightarrow$3 (deg)}
		& \secondthirdSeparationgammaIspNinety
		& \secondthirdSeparationgammaIspNinetyFive
		& \secondthirdSeparationgammaIspStandard
		& \secondthirdSeparationgammaIspOneHundredFive
		& \secondthirdSeparationgammaIspOneHundredTen
		&-0.09
		\\
	
		\textbf{2$^{nd}$ Stage Distance Flown (km)}
		& \SecondDistIspNinety
		& \SecondDistIspNinetyFive
		& \SecondDistIspStandard
		& \SecondDistIspOneHundredFive
		& \SecondDistIspOneHundredTen
		& -
		\\
		\textbf{2$^{nd}$ Stage Return Fuel (kg)}
		& \returnFuelIspNinety
		& \returnFuelIspNinetyFive
		& \returnFuelIspStandard
		& \returnFuelIspOneHundredFive
		& \returnFuelIspOneHundredTen
		& -
		\\
		\textbf{2$^{nd}$ Stage Return Distance (km)}
		& \returnDistIspNinety
		& \returnDistIspNinetyFive
		& \returnDistIspStandard
		& \returnDistIspOneHundredFive
		& \returnDistIspOneHundredTen
		&10.03
		\\
		\hline 
		\textbf{3$^{rd}$ Stage $\eta_{exergy}$ (\%)}
		& \textbf{\thirddExergyEffIspNinety}
		& \textbf{\thirddExergyEffIspNinetyFive}
		& \textbf{\thirddExergyEffIspStandard}
		& \textbf{\thirddExergyEffIspOneHundredFive}
		& \textbf{\thirddExergyEffIspOneHundredTen}
		& \textbf{0.162}
		\\
		\textbf{3$^{rd}$ Stage $t$, $q >$ 5kpa (s)}
		& \thirdqOverFiveIspNinety
		& \thirdqOverFiveIspNinetyFive
		& \thirdqOverFiveIspStandard
		& \thirdqOverFiveIspOneHundredFive
		& \thirdqOverFiveIspOneHundredTen
		& -
		\\
		\textbf{3$^{rd}$ Stage max $\alpha$ (deg)}
		& \thirdmaxAoAIspNinety
		& \thirdmaxAoAIspNinetyFive
		& \thirdmaxAoAIspStandard
		& \thirdmaxAoAIspOneHundredFive
		& \thirdmaxAoAIspOneHundredTen
		& -
		\\
		\textbf{3$^{rd}$ Stage Fuel Mass (kg)}
		& \thirdmFuelIspNinety
		& \thirdmFuelIspNinetyFive
		& \thirdmFuelIspStandard
		& \thirdmFuelIspOneHundredFive
		& \thirdmFuelIspOneHundredTen
		&-1.67
		\\
		\hline 
	\end{tabular} 
	
	\caption{Comparison of key trajectory parameters with variation in the specific impulse of the C-REST engines, with fly-back (Case 14).}
	\label{tab:comparison31}
\end{table}

The specific impulse of the SPARTAN is varied by $\pm10\%$ in order to assess the sensitivity of the optimised trajectory to the performance of the scramjet engines. 
Key parameters of the trajectories are summarised in Table \ref{tab:comparison31}, and comparison plots are shown in Appendix \ref{sec:app_comparison31}.
Raising the specific impulse of the C-REST engines increases the payload-to-orbit, by +18.3kg (+10.71\%) at 110\% $I_{SP}$, while lowering the specific impulse decreases the payload-to-orbit, by -14.7kg (-8.63\%) at 90\% $I_{SP}$. 
This produces a general trend in the payload-to-orbit of 1.7$\frac{\Delta kg}{\Delta \%I_{SP} }$, lower than the trend of 2.2$\frac{\Delta kg}{\Delta \%I_{SP} }$ observed in the sensitivity study without fly-back, in Section \ref{sec:ispsensitivitynoflyback}.
This lowered sensitivity in the payload-to-orbit is due to a correspondingly lowered sensitivity in the exergy efficiency of the SPARTAN, of 0.051$\frac{\Delta \% \eta}{\Delta \%I_{SP} }$, compared to 0.065$\frac{\Delta \% \eta}{\Delta \%I_{SP} }$ in the sensitivity study without fly-back. This lowered sensitivity is due to the modified $I_{SP}$ having no effect on the performance of the SPARTAN during the unpowered portions of the fly-back trajectory, which serve to offset the overall variation in exergy efficiency. 


Similarly to the specific impulse sensitivity study without fly-back conducted in Section \ref{sec:ispsensitivitynoflyback}, the first stage-SPARTAN separation conditions, as well as the exergy efficiency of the first stage, exhibit no clear trends. Following first stage-SPARTAN separation, the shape of the trajectory path of the SPARTAN does not change significantly with specific impulse variation, including the the pull-up altitude. As with the optimised trajectories with no fly-back, increasing the specific impulse of the scramjet engines by 10\% increases the velocity at separation (by +114m/s, +4.4\%) and decreases the trajectory angle (by -0.7$^\circ$, 6.4\%), while decreasing the specific impulse of the scramjet engines by 10\% decreases the velocity at SPARTAN-third stage separation (by -106m/s, -4.1\%), and increases \textcolor{red}{XXX Decreases?} the trajectory angle (by -1.1$^\circ$, -10\%).
The exergy efficiency of the third stage rocket increases as the exergy efficiency of the SPARTAN increases. This is in line with the trend which has been observed in all previous cases, that the increased separation velocity increases the propulsive efficiency of the third stage, increasing its performance. 






\subsection{Case 15: SPARTAN Mass Sensitivity with Fly-Back}\label{sec:m2var}


\begin{table}[ht]
\centering
\begin{tabular}{l c c c c c c} 
	\hline \textbf{Trajectory Condition}   \qquad  $m_{2}$:
	&95\%
	&97.5\%
	&100\%
	&102.5\%
	&105\%
	& $\Delta/\Delta$\%$m_{2}$
	\\
	\hline \textbf{Payload to Orbit (kg)}
	& \textbf{\PayloadToOrbitmSPARTANNinetyFive}
	& \textbf{\PayloadToOrbitmSPARTANNinetySevenFive}
	& \textbf{\PayloadToOrbitmSPARTANStandard}
	& \textbf{\PayloadToOrbitmSPARTANOneHundredTwoFive}
	& \textbf{\PayloadToOrbitmSPARTANOneHundredFive}
	&\textbf{-1.4}
	\\
	\textbf{Payload Variation (\%)}
	& \PayloadVarmSPARTANNinetyFive
	& \PayloadVarmSPARTANNinetySevenFive
	& \PayloadVarmSPARTANStandard
	& \PayloadVarmSPARTANOneHundredTwoFive
	& \PayloadVarmSPARTANOneHundredFive
	&-0.8
	\\
	\textbf{Total $\eta_{exergy}$ (\%)}
	& \textbf{\totalExergyEffmSPARTANNinetyFive}
	& \textbf{\totalExergyEffmSPARTANNinetySevenFive}
	& \textbf{\totalExergyEffmSPARTANStandard}
	& \textbf{\totalExergyEffmSPARTANOneHundredTwoFive}
	& \textbf{\totalExergyEffmSPARTANOneHundredFive}
	& \textbf{-0.00012}
	\\
	\hline 
	\textbf{1$^{st}$ Stage $\eta_{exergy}$ (\%)}
	& \textbf{\firstExergyEffmSPARTANNinetyFive}
	& \textbf{\firstExergyEffmSPARTANNinetySevenFive}
	& \textbf{\firstExergyEffmSPARTANStandard}
	& \textbf{\firstExergyEffmSPARTANOneHundredTwoFive}
	& \textbf{\firstExergyEffmSPARTANOneHundredFive}
	& \textbf{-0.028}
	\\
	\textbf{Separation Alt, 1$\rightarrow$2 (km)}
	& \firstsecondSeparationAltmSPARTANNinetyFive
	& \firstsecondSeparationAltmSPARTANNinetySevenFive
	& \firstsecondSeparationAltmSPARTANStandard
	& \firstsecondSeparationAltmSPARTANOneHundredTwoFive
	& \firstsecondSeparationAltmSPARTANOneHundredFive
	& -
	\\
	\textbf{Separation v, 1$\rightarrow$2 (m/s)}
	& \firstsecondSeparationvmSPARTANNinetyFive
	& \firstsecondSeparationvmSPARTANNinetySevenFive
	& \firstsecondSeparationvmSPARTANStandard
	& \firstsecondSeparationvmSPARTANOneHundredTwoFive
	& \firstsecondSeparationvmSPARTANOneHundredFive
	&-7.84
	\\
	\textbf{Separation $\gamma$, 1$\rightarrow$2 (deg)}
	& \firstsecondSeparationgammamSPARTANNinetyFive
	& \firstsecondSeparationgammamSPARTANNinetySevenFive
	& \firstsecondSeparationgammamSPARTANStandard
	& \firstsecondSeparationgammamSPARTANOneHundredTwoFive
	& \firstsecondSeparationgammamSPARTANOneHundredFive
	& -
	\\
	\hline 
	\textbf{2$^{nd}$ Stage $\eta_{exergy}$ (\%)}
	& \textbf{\secondExergyEffmSPARTANNinetyFive}
	& \textbf{\secondExergyEffmSPARTANNinetySevenFive}
	& \textbf{\secondExergyEffmSPARTANStandard}
	& \textbf{\secondExergyEffmSPARTANOneHundredTwoFive}
	& \textbf{\secondExergyEffmSPARTANOneHundredFive}
	& \textbf{-0.016}
	\\
	\textbf{Separation Alt, 2$\rightarrow$3 (km)}
	& \secondthirdSeparationAltmSPARTANNinetyFive
	& \secondthirdSeparationAltmSPARTANNinetySevenFive
	& \secondthirdSeparationAltmSPARTANStandard
	& \secondthirdSeparationAltmSPARTANOneHundredTwoFive
	& \secondthirdSeparationAltmSPARTANOneHundredFive
	&-0.06
	\\
	\textbf{Separation $v$, 2$\rightarrow$3 (m/s)}
	& \secondthirdSeparationvmSPARTANNinetyFive
	& \secondthirdSeparationvmSPARTANNinetySevenFive
	& \secondthirdSeparationvmSPARTANStandard
	& \secondthirdSeparationvmSPARTANOneHundredTwoFive
	& \secondthirdSeparationvmSPARTANOneHundredFive
	&-8.52
	\\
	\textbf{Separation $\gamma$, 2$\rightarrow$3 (deg)}
	& \secondthirdSeparationgammamSPARTANNinetyFive
	& \secondthirdSeparationgammamSPARTANNinetySevenFive
	& \secondthirdSeparationgammamSPARTANStandard
	& \secondthirdSeparationgammamSPARTANOneHundredTwoFive
	& \secondthirdSeparationgammamSPARTANOneHundredFive
	& -
	\\

	\textbf{2$^{nd}$ Stage Distance Flown (km)}
	& \SecondDistmSPARTANNinetyFive
	& \SecondDistmSPARTANNinetySevenFive
	& \SecondDistmSPARTANStandard
	& \SecondDistmSPARTANOneHundredTwoFive
	& \SecondDistmSPARTANOneHundredFive
	& -
	\\
	\textbf{2$^{nd}$ Stage Return Fuel (kg)}
	& \returnFuelmSPARTANNinetyFive
	& \returnFuelmSPARTANNinetySevenFive
	& \returnFuelmSPARTANStandard
	& \returnFuelmSPARTANOneHundredTwoFive
	& \returnFuelmSPARTANOneHundredFive
	& -
	\\
	\textbf{2$^{nd}$ Stage Return Distance (km)}
	& \returnDistmSPARTANNinetyFive
	& \returnDistmSPARTANNinetySevenFive
	& \returnDistmSPARTANStandard
	& \returnDistmSPARTANOneHundredTwoFive
	& \returnDistmSPARTANOneHundredFive
	& -
	\\
	\hline 
	\textbf{3$^{rd}$ Stage $\eta_{exergy}$ (\%)}
	& \textbf{\thirddExergyEffmSPARTANNinetyFive}
	& \textbf{\thirddExergyEffmSPARTANNinetySevenFive}
	& \textbf{\thirddExergyEffmSPARTANStandard}
	& \textbf{\thirddExergyEffmSPARTANOneHundredTwoFive}
	& \textbf{\thirddExergyEffmSPARTANOneHundredFive}
	& \textbf{-0.132}
	\\
	\textbf{3$^{rd}$ Stage $t$, $q >$ 5kpa (s)}
	& \thirdqOverFivemSPARTANNinetyFive
	& \thirdqOverFivemSPARTANNinetySevenFive
	& \thirdqOverFivemSPARTANStandard
	& \thirdqOverFivemSPARTANOneHundredTwoFive
	& \thirdqOverFivemSPARTANOneHundredFive
	& -
	\\
	\textbf{3$^{rd}$ Stage max $\alpha$ (deg)}
	& \thirdmaxAoAmSPARTANNinetyFive
	& \thirdmaxAoAmSPARTANNinetySevenFive
	& \thirdmaxAoAmSPARTANStandard
	& \thirdmaxAoAmSPARTANOneHundredTwoFive
	& \thirdmaxAoAmSPARTANOneHundredFive
	& -
	\\
	\textbf{3$^{rd}$ Stage Fuel Mass (kg)}
	& \thirdmFuelmSPARTANNinetyFive
	& \thirdmFuelmSPARTANNinetySevenFive
	& \thirdmFuelmSPARTANStandard
	& \thirdmFuelmSPARTANOneHundredTwoFive
	& \thirdmFuelmSPARTANOneHundredFive
	&1.36
	\\
	\hline 
\end{tabular} 
\caption{Comparison of key trajectory parameters with variation in the structural mass of the SPARTAN, with fly-back (Case 15).}
\label{tab:comparison101}
\end{table}


The mass of the SPARTAN is varied by $\pm$5\% to investigate the sensitivity of the launch system performance to the structural mass of the second stage, with the inclusion of SPARTAN fly-back. As in Section \ref{sec:SpartanMassnoreturn}, the mass is varied by only $\pm$5\% in order to limit the variation in the velocity of the first stage-SPARTAN separation.
Table \ref{tab:comparison101} details key parameters of each trajectory, and Appendix \ref{sec:app_comparison101} shows comparison plots.
Varying the structural mass of the SPARTAN yields a change in maximum payload-mass to orbit of +7.2kg (+4.22\%) at 95\% mass, and -6.0kg (-3.56\%) at 105\% mass. 

The first stage-SPARTAN separation conditions show no significant trend with variation in the mass of the SPARTAN, except for differing velocities due to the first stage accelerating a varied total mass.
As observed in Section \ref{sec:SpartanMassnoreturn}, the structural mass of the SPARTAN is increased, the exergy efficiency of the first stage decreases, from 8.780\%$\eta$ at 95\% structural mass, to 8.356\%$\eta$ at 105\% structural mass. This is due to the first stage rocket not accelerating as quickly as the mass of the SPARTAN is increased, causing the propulsive efficiency of the first stage to decrease (illustrated by Equation \ref{eq:rocketeff}).
 However, the sensitivity of the exergy efficiency of the first stage rocket to variation in the mass of the SPARTAN is lower when compared to the sensitivity study with no fly-back (described in Section \ref{sec:SpartanMassnoreturn}), at -0.028$\frac{\Delta \%\eta}{\Delta\% m_{SPARTAN}}$, compared to -0.043 $\frac{\Delta \%\eta}{\Delta\% m_{SPARTAN}}$ with no fly-back. Additionally, this exergy efficiency variation is solely due to variations in the propulsive efficiency of the first stage, with no consistent variation in drag losses. This indicates that the mass of the SPARTAN does not cause an efficiency trade-off between the first stage and the SPARTAN, when the fly-back of the SPARTAN is included. This is in contrast to the trade-off observed in the sensitivity study without fly-back, in Section \ref{sec:SpartanMassnoreturn}, and is due to the first stage-SPARTAN separation conditions being partly determined by the requirement of the SPARTAN to bank when the fly-back is included. 
 
 As was observed in Section \ref{sec:SpartanMassnoreturn}, the lower velocity of first stage-SPARTAN separation means that when the SPARTAN mass is increased, the velocity range over which the SPARTAN is accelerating is lower. This is beneficial for the specific impulse of the C-REST engines, which exhibit higher $I_{SP}$ at lower velocities. For this reason, when the SPARTAN mass is increased, the specific impulse of the SPARTAN stays high for longer, above 500s $I_{SP}$ for 278.8s of its trajectory at 105\% mass, compared to 254.4s above 500s $I_{SP}$ at 95\% mass. 
However, the higher SPARTAN mass decreases the overall acceleration of the SPARTAN, in turn decreasing the efficiency of the third stage due to increased propulsive losses.
In contrast to the mass sensitivity study without fly-back, the SPARTAN-third stage separation point shows a trend of decreasing altitude, of -0.06$\frac{\Delta km}{\Delta\% m_{SPARTAN}}$. As was observed when the drag of the SPARTAN was varied in Section \ref{sec:dragvar}, this is in order to maintain a consistently sized initial skip during the return trajectory.


\subsection{Case 16: SPARTAN Fuel Mass Sensitivity with Fly-Back}

\begin{table}[ht]
	\centering
\begin{tabular}{l c c c c c c} 
	\hline \textbf{Trajectory Condition}   \qquad  $m_{f,2}$:
	&90\%
	&95\%
	&100\%
	&105\%
	&110\%
	& $\Delta/\Delta$\%$m_{F,2}$
	\\
	\hline \textbf{Payload to Orbit (kg)}
	& \textbf{\PayloadToOrbitmFuelNinety}
	& \textbf{\PayloadToOrbitmFuelNinetyFive}
	& \textbf{\PayloadToOrbitmFuelStandard}
	& \textbf{\PayloadToOrbitmFuelOneHundredFive}
	& \textbf{\PayloadToOrbitmFuelOneHundredTen}
	&\textbf{0.7}
	\\
	\textbf{Payload Variation (\%)}
	& \PayloadVarmFuelNinety
	& \PayloadVarmFuelNinetyFive
	& \PayloadVarmFuelStandard
	& \PayloadVarmFuelOneHundredFive
	& \PayloadVarmFuelOneHundredTen
	&0.4
	\\
	\textbf{Total $\eta_{exergy}$ (\%)}
	& \textbf{\totalExergyEffmFuelNinety}
	& \textbf{\totalExergyEffmFuelNinetyFive}
	& \textbf{\totalExergyEffmFuelStandard}
	& \textbf{\totalExergyEffmFuelOneHundredFive}
	& \textbf{\totalExergyEffmFuelOneHundredTen}
	& -
	\\
	\hline 
	\textbf{1$^{st}$ Stage $\eta_{exergy}$ (\%)}
	& \textbf{\firstExergyEffmFuelNinety}
	& \textbf{\firstExergyEffmFuelNinetyFive}
	& \textbf{\firstExergyEffmFuelStandard}
	& \textbf{\firstExergyEffmFuelOneHundredFive}
	& \textbf{\firstExergyEffmFuelOneHundredTen}
	& -
	\\
	\textbf{Separation Alt, 1$\rightarrow$2 (km)}
	& \firstsecondSeparationAltmFuelNinety
	& \firstsecondSeparationAltmFuelNinetyFive
	& \firstsecondSeparationAltmFuelStandard
	& \firstsecondSeparationAltmFuelOneHundredFive
	& \firstsecondSeparationAltmFuelOneHundredTen
	& -
	\\
	\textbf{Separation v, 1$\rightarrow$2 (m/s)}
	& \firstsecondSeparationvmFuelNinety
	& \firstsecondSeparationvmFuelNinetyFive
	& \firstsecondSeparationvmFuelStandard
	& \firstsecondSeparationvmFuelOneHundredFive
	& \firstsecondSeparationvmFuelOneHundredTen
	&-2.82
	\\
	\textbf{Separation $\gamma$, 1$\rightarrow$2 (deg)}
	& \firstsecondSeparationgammamFuelNinety
	& \firstsecondSeparationgammamFuelNinetyFive
	& \firstsecondSeparationgammamFuelStandard
	& \firstsecondSeparationgammamFuelOneHundredFive
	& \firstsecondSeparationgammamFuelOneHundredTen
	& -
	\\
	\hline 
	\textbf{2$^{nd}$ Stage $\eta_{exergy}$ (\%)}
	& \textbf{\secondExergyEffmFuelNinety}
	& \textbf{\secondExergyEffmFuelNinetyFive}
	& \textbf{\secondExergyEffmFuelStandard}
	& \textbf{\secondExergyEffmFuelOneHundredFive}
	& \textbf{\secondExergyEffmFuelOneHundredTen}
	& \textbf{-0.013}
	\\
	\textbf{Separation Alt, 2$\rightarrow$3 (km)}
	& \secondthirdSeparationAltmFuelNinety
	& \secondthirdSeparationAltmFuelNinetyFive
	& \secondthirdSeparationAltmFuelStandard
	& \secondthirdSeparationAltmFuelOneHundredFive
	& \secondthirdSeparationAltmFuelOneHundredTen
	& -
	\\
	\textbf{Separation $v$, 2$\rightarrow$3 (m/s)}
	& \secondthirdSeparationvmFuelNinety
	& \secondthirdSeparationvmFuelNinetyFive
	& \secondthirdSeparationvmFuelStandard
	& \secondthirdSeparationvmFuelOneHundredFive
	& \secondthirdSeparationvmFuelOneHundredTen
	&4.73
	\\
	\textbf{Separation $\gamma$, 2$\rightarrow$3 (deg)}
	& \secondthirdSeparationgammamFuelNinety
	& \secondthirdSeparationgammamFuelNinetyFive
	& \secondthirdSeparationgammamFuelStandard
	& \secondthirdSeparationgammamFuelOneHundredFive
	& \secondthirdSeparationgammamFuelOneHundredTen
	&-0.04
	\\

	\textbf{2$^{nd}$ Stage Distance Flown (km)}
	& \SecondDistmFuelNinety
	& \SecondDistmFuelNinetyFive
	& \SecondDistmFuelStandard
	& \SecondDistmFuelOneHundredFive
	& \SecondDistmFuelOneHundredTen
	&12.29
	\\
	\textbf{2$^{nd}$ Stage Return Fuel (kg)}
	& \returnFuelmFuelNinety
	& \returnFuelmFuelNinetyFive
	& \returnFuelmFuelStandard
	& \returnFuelmFuelOneHundredFive
	& \returnFuelmFuelOneHundredTen
	& -
	\\
	\textbf{2$^{nd}$ Stage Return Distance (km)}
	& \returnDistmFuelNinety
	& \returnDistmFuelNinetyFive
	& \returnDistmFuelStandard
	& \returnDistmFuelOneHundredFive
	& \returnDistmFuelOneHundredTen
	& -
	\\
	\hline 
	\textbf{3$^{rd}$ Stage $\eta_{exergy}$ (\%)}
	& \textbf{\thirddExergyEffmFuelNinety}
	& \textbf{\thirddExergyEffmFuelNinetyFive}
	& \textbf{\thirddExergyEffmFuelStandard}
	& \textbf{\thirddExergyEffmFuelOneHundredFive}
	& \textbf{\thirddExergyEffmFuelOneHundredTen}
	& \textbf{0.066}
	\\
	\textbf{3$^{rd}$ Stage $t$, $q >$ 5kpa (s)}
	& \thirdqOverFivemFuelNinety
	& \thirdqOverFivemFuelNinetyFive
	& \thirdqOverFivemFuelStandard
	& \thirdqOverFivemFuelOneHundredFive
	& \thirdqOverFivemFuelOneHundredTen
	& -
	\\
	\textbf{3$^{rd}$ Stage max $\alpha$ (deg)}
	& \thirdmaxAoAmFuelNinety
	& \thirdmaxAoAmFuelNinetyFive
	& \thirdmaxAoAmFuelStandard
	& \thirdmaxAoAmFuelOneHundredFive
	& \thirdmaxAoAmFuelOneHundredTen
	& -
	\\
	\textbf{3$^{rd}$ Stage Fuel Mass (kg)}
	& \thirdmFuelmFuelNinety
	& \thirdmFuelmFuelNinetyFive
	& \thirdmFuelmFuelStandard
	& \thirdmFuelmFuelOneHundredFive
	& \thirdmFuelmFuelOneHundredTen
	&-0.69
	\\
	\hline 
\end{tabular} 
\caption{Comparison of key trajectory parameters with variation in the fuel mass of the SPARTAN, with fly-back (Case 16).}
\label{tab:comparison111}
\end{table}
	
The fuel mass of the SPARTAN is varied by $\pm$10\%, to investigate the sensitivity of the performance of the launch system to variations in the size of the fuel tanks within the SPARTAN. 
Appendix \ref{sec:app_comparison111} shows plots comparing each trajectory, and Table \ref{tab:comparison111} details comparisons of key trajectory parameters. 
When the fuel mass within the SPARTAN is increased by 10\%, the payload to orbit increases by +9.6kg (+5.61\%) and when the fuel mass is decreased by 10\%, the payload mass reduces by -4.6kg (-2.73\%). The magnitude of the payload-to-orbit sensitivity is very similar to the sensitivity observed without fly-back, in Section \ref{sec:fuelmassNoReturn}, indicating that the addition of fly-back does not have a significant effect on the sensitivity of the launch system to variations in the fuel mass of the SPARTAN. 

The first stage shows no significant trend in its trajectory when the fuel mass of the SPARTAN is varied, besides a decrease in the overall acceleration due to the additional mass. This is in contrast to the trends observed in Section \ref{sec:fuelmassNoReturn}, and is due to the additional factor of the banking of the SPARTAN driving the first stage-SPARTAN separation conditions, as observed in Section \ref{sec:m2var}.
As in Section \ref{sec:fuelmassNoReturn}, increasing the fuel mass of the SPARTAN decreases the exergy efficiency of the SPARTAN, by -0.037\%$\eta$ (-0.9\%) at 110\% fuel mass, and decreasing the fuel mass of the SPARTAN increases its exergy efficiency, by +0.219\%$\eta$ (+5.5\%) at 90\% fuel mass. Once again, the overall exergy efficiency of the system shows no distinct trend. 
As in Section \ref{sec:fuelmassNoReturn}, this is due to the increased period of acceleration causing the specific impulse of the C-REST engines to decrease. However, the overall energy availability is increased by the additional fuel mass, resulting in more overall exergy. This results in the overall energy imparted upon the third stage by the SPARTAN increasing, from \secondExergythirdStagemFuelNinety GJ at 90\% $m_{f}$, to \secondExergythirdStagemFuelOneHundredTen GJ at 110\% $m_{f}$, in turn increasing the payload-to-orbit.




\subsection{Case 17: Third Stage Mass Sensitivity with Fly-Back}

\begin{table}[ht]
	\centering
	\begin{tabular}{l c c c c c c} 
		\hline \textbf{Trajectory Condition}   \qquad  $m_{3}$:
		&90\%
		&95\%
		&100\%
		&105\%
		&110\%
		& $\Delta/\Delta$\%$m_{3}$
		\\
		\hline \textbf{Payload to Orbit (kg)}
		& \textbf{\PayloadToOrbitmThreeNinety}
		& \textbf{\PayloadToOrbitmThreeNinetyFive}
		& \textbf{\PayloadToOrbitmThreeStandard}
		& \textbf{\PayloadToOrbitmThreeOneHundredFive}
		& \textbf{\PayloadToOrbitmThreeOneHundredTen}
		&\textbf{1}
		\\
		\textbf{Payload Variation (\%)}
		& \PayloadVarmThreeNinety
		& \PayloadVarmThreeNinetyFive
		& \PayloadVarmThreeStandard
		& \PayloadVarmThreeOneHundredFive
		& \PayloadVarmThreeOneHundredTen
		&0.56
		\\
		\textbf{Total $\eta_{exergy}$ (\%)}
		& \textbf{\totalExergyEffmThreeNinety}
		& \textbf{\totalExergyEffmThreeNinetyFive}
		& \textbf{\totalExergyEffmThreeStandard}
		& \textbf{\totalExergyEffmThreeOneHundredFive}
		& \textbf{\totalExergyEffmThreeOneHundredTen}
		& \textbf{8e-05}
		\\
		\hline 
		\textbf{1$^{st}$ Stage $\eta_{exergy}$ (\%)}
		& \textbf{\firstExergyEffmThreeNinety}
		& \textbf{\firstExergyEffmThreeNinetyFive}
		& \textbf{\firstExergyEffmThreeStandard}
		& \textbf{\firstExergyEffmThreeOneHundredFive}
		& \textbf{\firstExergyEffmThreeOneHundredTen}
		& \textbf{-0.019}
		\\
		\textbf{Separation Alt, 1$\rightarrow$2 (km)}
		& \firstsecondSeparationAltmThreeNinety
		& \firstsecondSeparationAltmThreeNinetyFive
		& \firstsecondSeparationAltmThreeStandard
		& \firstsecondSeparationAltmThreeOneHundredFive
		& \firstsecondSeparationAltmThreeOneHundredTen
		&-0.14
		\\
		\textbf{Separation v, 1$\rightarrow$2 (m/s)}
		& \firstsecondSeparationvmThreeNinety
		& \firstsecondSeparationvmThreeNinetyFive
		& \firstsecondSeparationvmThreeStandard
		& \firstsecondSeparationvmThreeOneHundredFive
		& \firstsecondSeparationvmThreeOneHundredTen
		&-5.06
		\\
		\textbf{Separation $\gamma$, 1$\rightarrow$2 (deg)}
		& \firstsecondSeparationgammamThreeNinety
		& \firstsecondSeparationgammamThreeNinetyFive
		& \firstsecondSeparationgammamThreeStandard
		& \firstsecondSeparationgammamThreeOneHundredFive
		& \firstsecondSeparationgammamThreeOneHundredTen
		& -
		\\
		\hline 
		\textbf{2$^{nd}$ Stage $\eta_{exergy}$ (\%)}
		& \textbf{\secondExergyEffmThreeNinety}
		& \textbf{\secondExergyEffmThreeNinetyFive}
		& \textbf{\secondExergyEffmThreeStandard}
		& \textbf{\secondExergyEffmThreeOneHundredFive}
		& \textbf{\secondExergyEffmThreeOneHundredTen}
		& \textbf{0.029}
		\\
		\textbf{Separation Alt, 2$\rightarrow$3 (km)}
		& \secondthirdSeparationAltmThreeNinety
		& \secondthirdSeparationAltmThreeNinetyFive
		& \secondthirdSeparationAltmThreeStandard
		& \secondthirdSeparationAltmThreeOneHundredFive
		& \secondthirdSeparationAltmThreeOneHundredTen
		& -
		\\
		\textbf{Separation $v$, 2$\rightarrow$3 (m/s)}
		& \secondthirdSeparationvmThreeNinety
		& \secondthirdSeparationvmThreeNinetyFive
		& \secondthirdSeparationvmThreeStandard
		& \secondthirdSeparationvmThreeOneHundredFive
		& \secondthirdSeparationvmThreeOneHundredTen
		&-6.25
		\\
		\textbf{Separation $\gamma$, 2$\rightarrow$3 (deg)}
		& \secondthirdSeparationgammamThreeNinety
		& \secondthirdSeparationgammamThreeNinetyFive
		& \secondthirdSeparationgammamThreeStandard
		& \secondthirdSeparationgammamThreeOneHundredFive
		& \secondthirdSeparationgammamThreeOneHundredTen
		&0.09
		\\
	
		\textbf{2$^{nd}$ Stage Distance Flown (km)}
		& \SecondDistmThreeNinety
		& \SecondDistmThreeNinetyFive
		& \SecondDistmThreeStandard
		& \SecondDistmThreeOneHundredFive
		& \SecondDistmThreeOneHundredTen
		&-3.92
		\\
		\textbf{2$^{nd}$ Stage Return Fuel (kg)}
		& \returnFuelmThreeNinety
		& \returnFuelmThreeNinetyFive
		& \returnFuelmThreeStandard
		& \returnFuelmThreeOneHundredFive
		& \returnFuelmThreeOneHundredTen
		& -
		\\
		\textbf{2$^{nd}$ Stage Return Distance (km)}
		& \returnDistmThreeNinety
		& \returnDistmThreeNinetyFive
		& \returnDistmThreeStandard
		& \returnDistmThreeOneHundredFive
		& \returnDistmThreeOneHundredTen
		& -
		\\
		\hline 
		\textbf{3$^{rd}$ Stage $\eta_{exergy}$ (\%)}
		& \textbf{\thirddExergyEffmThreeNinety}
		& \textbf{\thirddExergyEffmThreeNinetyFive}
		& \textbf{\thirddExergyEffmThreeStandard}
		& \textbf{\thirddExergyEffmThreeOneHundredFive}
		& \textbf{\thirddExergyEffmThreeOneHundredTen}
		& \textbf{-0.071}
		\\
		\textbf{3$^{rd}$ Stage $t$, $q >$ 5kpa (s)}
		& \thirdqOverFivemThreeNinety
		& \thirdqOverFivemThreeNinetyFive
		& \thirdqOverFivemThreeStandard
		& \thirdqOverFivemThreeOneHundredFive
		& \thirdqOverFivemThreeOneHundredTen
		& -
		\\
		\textbf{3$^{rd}$ Stage max $\alpha$ (deg)}
		& \thirdmaxAoAmThreeNinety
		& \thirdmaxAoAmThreeNinetyFive
		& \thirdmaxAoAmThreeStandard
		& \thirdmaxAoAmThreeOneHundredFive
		& \thirdmaxAoAmThreeOneHundredTen
		& -
		\\
		\textbf{3$^{rd}$ Stage Fuel Mass (kg)}
		& \thirdmFuelmThreeNinety
		& \thirdmFuelmThreeNinetyFive
		& \thirdmFuelmThreeStandard
		& \thirdmFuelmThreeOneHundredFive
		& \thirdmFuelmThreeOneHundredTen
		&29.07
		\\
		\hline 
	\end{tabular} 
	\caption{Comparison of key trajectory parameters with variation in the mass of the third stage, with fly-back (Case 17).}
	\label{tab:comparison81}
\end{table}

The mass of the third stage rocket is varied by $\pm$10\%, to investigate the effects of the internal mass density of the third stage rocket, when the fly-back of the SPARTAN is included. Table \ref{tab:comparison81} details key trajectory parameters, and Appendix \ref{sec:app_comparison81} shows trajectory comparison plots. As in Section \ref{sec:m3noReturn}, the varied mass is a combination of the fuel and structural mass of the third stage, and payload mass, representing the density of the components within the third stage. As previously, the heat shield mass is not varied, the structural mass held at 9\% of the total, non-heat shield mass, and the remaining mass variation is a combination of fuel and payload mass.

Increasing the third stage mass by 10\% causes a corresponding increase in the payload-to-orbit of +9.9kg (+5.76\%), while decreasing the third stage mass by 10\% causes a decrease in payload-to-orbit of -9.4kg (-5.54\%). 
This payload-to-orbit mass sensitivity is slightly higher than the third stage mass sensitivity without fly-back, detailed in Section \ref{sec:m3noReturn}. In addition, the payload-to-orbit is lower when the fly-back of the SPARTAN is included, resulting in a significantly higher percentage payload increase, at 0.56$\frac{\Delta m_{payload}}{\Delta\%m_3}$, compared to 0.47$\frac{\Delta m_{payload}}{\Delta\%m_3}$ without fly-back. 
Similarly to the sensitivity study without fly-back, in Section \ref{sec:m3noReturn}, the exergy efficiency of the SPARTAN increases as the mass of the third stage increases, by +0.337\%$\eta$ (+8.4\%) at 110\% $m_3$, and decreases when the third stage mass is decreased, by -0.236\%$\eta$ (-5.9\%) at 90\% $m_3$.
This trend is caused by the higher third stage mass decreasing the acceleration of the SPARTAN, so that overall, the specific impulse of the SPARTAN stays higher. 
The sensitivity of the exergy efficiency of the SPARTAN is lower when fly-back is included, at $0.029\frac{\Delta\%\eta}{\Delta\%m_3}$, compared to $0.037\frac{\Delta\%\eta}{\Delta\%m_3}$ without fly-back. This lowered sensitivity is due to the efficiency of the fly-back relying on the performance of the SPARTAN alone, which is not directly affected by variations in the third stage mass.  
 The exergy efficiency sensitivity of the third stage is also decreased to -$0.071\frac{\Delta\%\eta}{\Delta\%m_3}$ when fly-back is included, compared to -$0.095\frac{\Delta\%\eta}{\Delta\%m_3}$ without fly-back. 
However, this lowered efficiency trend still results in an increased payload mass-to-orbit sensitivity compared to the trajectory case without fly-back. This is due to the lower payload-to-orbit when fly-back is included, which results in more fuel within the third stage, and consequently that each percentage of exergy efficiency gained or lost results in a larger total payload mass change. 

\subsection{Case 18: Third Stage Specific Impulse Sensitivity with Fly-Back}


\begin{table}[ht]
	% note the names of the variables are wrong, and should go from 95-105
	\centering
	\begin{tabular}{l c c c c c c} 
		\hline \textbf{Trajectory Condition}   \qquad  $I_{SP,3}$:
		&95\%
		&97.5\%
		&100\%
		&102.5\%
		&105\%
		& $\Delta/\Delta$\%$I_{SP,3}$
		\\
		\hline \textbf{Payload to Orbit (kg)}
		& \textbf{\PayloadToOrbitISPThreeNinety}
		& \textbf{\PayloadToOrbitISPThreeNinetyFive}
		& \textbf{\PayloadToOrbitISPThreeStandard}
		& \textbf{\PayloadToOrbitISPThreeOneHundredFive}
		& \textbf{\PayloadToOrbitISPThreeOneHundredTen}
		&\textbf{4.6}
		\\
		\textbf{Payload Variation (\%)}
		& \PayloadVarISPThreeNinety
		& \PayloadVarISPThreeNinetyFive
		& \PayloadVarISPThreeStandard
		& \PayloadVarISPThreeOneHundredFive
		& \PayloadVarISPThreeOneHundredTen
		&2.68
		\\
		\textbf{Total $\eta_{exergy}$ (\%)}
		& \textbf{\totalExergyEffISPThreeNinety}
		& \textbf{\totalExergyEffISPThreeNinetyFive}
		& \textbf{\totalExergyEffISPThreeStandard}
		& \textbf{\totalExergyEffISPThreeOneHundredFive}
		& \textbf{\totalExergyEffISPThreeOneHundredTen}
		& \textbf{0.00043}
		\\
		\hline 
		\textbf{1$^{st}$ Stage $\eta_{exergy}$ (\%)}
		& \textbf{\firstExergyEffISPThreeNinety}
		& \textbf{\firstExergyEffISPThreeNinetyFive}
		& \textbf{\firstExergyEffISPThreeStandard}
		& \textbf{\firstExergyEffISPThreeOneHundredFive}
		& \textbf{\firstExergyEffISPThreeOneHundredTen}
		& \textbf{0.002}
		\\
		\textbf{Separation Alt, 1$\rightarrow$2 (km)}
		& \firstsecondSeparationAltISPThreeNinety
		& \firstsecondSeparationAltISPThreeNinetyFive
		& \firstsecondSeparationAltISPThreeStandard
		& \firstsecondSeparationAltISPThreeOneHundredFive
		& \firstsecondSeparationAltISPThreeOneHundredTen
		& -
		\\
		\textbf{Separation v, 1$\rightarrow$2 (m/s)}
		& \firstsecondSeparationvISPThreeNinety
		& \firstsecondSeparationvISPThreeNinetyFive
		& \firstsecondSeparationvISPThreeStandard
		& \firstsecondSeparationvISPThreeOneHundredFive
		& \firstsecondSeparationvISPThreeOneHundredTen
		& -
		\\
		\textbf{Separation $\gamma$, 1$\rightarrow$2 (deg)}
		& \firstsecondSeparationgammaISPThreeNinety
		& \firstsecondSeparationgammaISPThreeNinetyFive
		& \firstsecondSeparationgammaISPThreeStandard
		& \firstsecondSeparationgammaISPThreeOneHundredFive
		& \firstsecondSeparationgammaISPThreeOneHundredTen
		& -
		\\
		\hline 
		\textbf{2$^{nd}$ Stage $\eta_{exergy}$ (\%)}
		& \textbf{\secondExergyEffISPThreeNinety}
		& \textbf{\secondExergyEffISPThreeNinetyFive}
		& \textbf{\secondExergyEffISPThreeStandard}
		& \textbf{\secondExergyEffISPThreeOneHundredFive}
		& \textbf{\secondExergyEffISPThreeOneHundredTen}
		& -
		\\
		\textbf{Separation Alt, 2$\rightarrow$3 (km)}
		& \secondthirdSeparationAltISPThreeNinety
		& \secondthirdSeparationAltISPThreeNinetyFive
		& \secondthirdSeparationAltISPThreeStandard
		& \secondthirdSeparationAltISPThreeOneHundredFive
		& \secondthirdSeparationAltISPThreeOneHundredTen
		& -
		\\
		\textbf{Separation $v$, 2$\rightarrow$3 (m/s)}
		& \secondthirdSeparationvISPThreeNinety
		& \secondthirdSeparationvISPThreeNinetyFive
		& \secondthirdSeparationvISPThreeStandard
		& \secondthirdSeparationvISPThreeOneHundredFive
		& \secondthirdSeparationvISPThreeOneHundredTen
		& -
		\\
		\textbf{Separation $\gamma$, 2$\rightarrow$3 (deg)}
		& \secondthirdSeparationgammaISPThreeNinety
		& \secondthirdSeparationgammaISPThreeNinetyFive
		& \secondthirdSeparationgammaISPThreeStandard
		& \secondthirdSeparationgammaISPThreeOneHundredFive
		& \secondthirdSeparationgammaISPThreeOneHundredTen
		& -
		\\
	
		\textbf{2$^{nd}$ Stage Distance Flown (km)}
		& \SecondDistISPThreeNinety
		& \SecondDistISPThreeNinetyFive
		& \SecondDistISPThreeStandard
		& \SecondDistISPThreeOneHundredFive
		& \SecondDistISPThreeOneHundredTen
		&1.75
		\\
		\textbf{2$^{nd}$ Stage Return Fuel (kg)}
		& \returnFuelISPThreeNinety
		& \returnFuelISPThreeNinetyFive
		& \returnFuelISPThreeStandard
		& \returnFuelISPThreeOneHundredFive
		& \returnFuelISPThreeOneHundredTen
		& -
		\\
		\textbf{2$^{nd}$ Stage Return Distance (km)}
		& \returnDistISPThreeNinety
		& \returnDistISPThreeNinetyFive
		& \returnDistISPThreeStandard
		& \returnDistISPThreeOneHundredFive
		& \returnDistISPThreeOneHundredTen
		& -
		\\
		\hline 
		\textbf{3$^{rd}$ Stage $\eta_{exergy}$ (\%)}
		& \textbf{\thirddExergyEffISPThreeNinety}
		& \textbf{\thirddExergyEffISPThreeNinetyFive}
		& \textbf{\thirddExergyEffISPThreeStandard}
		& \textbf{\thirddExergyEffISPThreeOneHundredFive}
		& \textbf{\thirddExergyEffISPThreeOneHundredTen}
		& \textbf{0.479}
		\\
		\textbf{3$^{rd}$ Stage $t$, $q >$ 5kpa (s)}
		& \thirdqOverFiveISPThreeNinety
		& \thirdqOverFiveISPThreeNinetyFive
		& \thirdqOverFiveISPThreeStandard
		& \thirdqOverFiveISPThreeOneHundredFive
		& \thirdqOverFiveISPThreeOneHundredTen
		&0.14
		\\
		\textbf{3$^{rd}$ Stage max $\alpha$ (deg)}
		& \thirdmaxAoAISPThreeNinety
		& \thirdmaxAoAISPThreeNinetyFive
		& \thirdmaxAoAISPThreeStandard
		& \thirdmaxAoAISPThreeOneHundredFive
		& \thirdmaxAoAISPThreeOneHundredTen
		& -
		\\
		\textbf{3$^{rd}$ Stage Fuel Mass (kg)}
		& \thirdmFuelISPThreeNinety
		& \thirdmFuelISPThreeNinetyFive
		& \thirdmFuelISPThreeStandard
		& \thirdmFuelISPThreeOneHundredFive
		& \thirdmFuelISPThreeOneHundredTen
		&-4.57
		\\
		\hline 
	\end{tabular} 
	\caption{Comparison of key trajectory parameters with variation in the specific impulse of the third stage, with fly-back (Case 18).}
	\label{tab:isp3}
\end{table}

The specific impulse of the third stage rocket is varied by $\pm$5\% to investigate the sensitivity of the launch system to the performance of the third stage rocket, when the fly-back of the SPARTAN is included. 
Table \ref{tab:isp3} shows selected performance indicators, while Appendix \ref{sec:app_comparison91} shows comparison plots of the maximum payload-to-orbit trajectory at each third stage specific impulse. 

The sensitivity of the optimal trajectory to the third stage specific impulse with SPARTAN fly-back is very similar to that observed in the sensitivity study with no SPARTAN fly-back, in Section \ref{sec:isp3NoReturn}, with a sensitivity of 4.6$\frac{\Delta m_{payload}}{\Delta\%I_{SP,3}}$ variation. 
 These similar sensitivities indicate that the fly-back does not considerably effect the sensitivity of the launch system to variations in the third stage specific impulse, and that the third stage specific impulse has a consistent magnitude of effect at lower separation velocities. 
The trajectory of the first stage does not change significantly as the specific impulse of the third stage is varied. 
The exergy efficiency of the first stage shows a slight trend, however, this is very small. 
The trajectory SPARTAN shows no distinct trends as the specific impulse of the third stage is varied, except for a general decrease in the third stage separation angle.
As the specific impulse of the third stage is increased, SPARTAN-third stage separation angle and the angle of attack schedule of the third stage are modified, so that 90km altitude is reached at circularisation conditions in all cases. This causes an increase in specific impulse to result in more time spent in-atmosphere, with the third stage spending 14.9s at dynamic pressures greater than 5kPa at 105\% $I_{SP}$, compared to 12.1s at 95\% $I_{SP}$. 






\section{Comparison of Sensitivities with Fly-Back}
%0.4/(50*0.01)/(1.5/(4957*0.01)) calculated like this

\begin{figure}[th]
	\centering
	\includegraphics[width=0.99\linewidth]{figures/6_FlyBack/BarChart}
	\caption{The sensitivity of the key design parameters of the launch system, including SPARTAN fly-back. Red and green coloured areas indicate decreases or increases in the magnitude of sensitivity respectively, compared to the sensitivity study without SPARTAN fly-back in Section \ref{sec:comparisonNoReturn}.}
	\label{fig:BarChartreturn}
\end{figure}

The sensitivities of the performance of the launch system, including the fly-back of the SPARTAN, to a variety of design parameters have been presented in the preceding sections. Figure \ref{fig:BarChartreturn} shows a relative comparison of the payload-to-orbit sensitivity for each design parameter, by percentage. The magnitude of each sensitivity is also compared with the sensitivity of the launch system performance without fly-back, detailed in Section \ref{sec:comparisonNoReturn}.

The sensitivity of the launch system to the maximum dynamic pressure is unchanged when fly-back is included. However, the slight decrease in the sensitivity of the launch system to the structural mass of the SPARTAN, to -1.4$\frac{\Delta kg}{\Delta\%m_{SPARTAN}}$, means that the potential beneficial effects of reducing the maximum dynamic pressure of the SPARTAN are reduced slightly. So long as the mass of the SPARTAN reduces by 28.3kg for each 1kPa reduction in the maximum dynamic pressure, the performance of the launch system will improve.
 The sensitivity of the launch system to the specific impulse of the SPARTAN is decreased significantly when the fly-back of the SPARTAN is included, to 1.7$\frac{\Delta kg}{\Delta\%I_{SP,SPARTAN}}$, a decrease of -0.5$\frac{\Delta kg}{\Delta\%I_{SP,SPARTAN}}$ (-22.7\%) compared to the sensitivity without fly-back. The sensitivity of the launch system to the SPARTAN's structural mass is also decreased, to -1.4$\frac{\Delta kg}{\Delta\%m_{SPARTAN}}$. Comparing these sensitivities, it is apparent that if the specific impulse of the SPARTAN can be increased by 1\% with less than 1.21\% (60.0kg) increase in the total mass of the SPARTAN, then the overall performance of the launch system will be improved.  
Similarly, the sensitivity of the launch system to variation in the drag of the SPARTAN is reduced, to -1.5$\frac{\Delta kg}{\Delta\%C_{d,SPARTAN}}$. Comparing this sensitivity with the sensitivity to the structural mass of the SPARTAN, the specific impulse of the C-REST engines must be improved by 1\% while increasing the drag of the SPARTAN by less than 1.13\% due to shape variation, in order for the overall performance change to be beneficial. 

The decreased sensitivity of the launch system performance to the structural mass of the SPARTAN, along with the unchanged fuel mass sensitivity, means that so long as 1kg of fuel mass can be added with less than 1.59kg of structural mass added, the performance of the launch system will improve. Additionally, the decreased sensitivity of the launch system to the drag of the SPARTAN means that so long as 1kg of fuel can be added to the SPARTAN, with a drag increase of less than 0.030\% due to increased size, then the maximum payload-to-orbit will increase. 
Lastly, comparing the increased third stage mass sensitivity, of 1$\frac{\Delta kg}{\Delta\%m_{3}}$, with the decreased SPARTAN drag sensitivity, shows that if the size of the third stage can be increased so that the third stage mass increases by 1kg, while the size of the SPARTAN's fuselage is varied so that the increase in SPARTAN drag is less than 0.020\%, the maximum payload-to-orbit will be improved. 





\section{Summary}

In this chapter, the maximum payload-to-orbit trajectory for a rocket-scramjet-rocket system has been calculated, with the inclusion of the fly-back of the SPARTAN scramjet-powered stage. It was found that this launch system is able to deliver \PayloadToOrbitStandard kg of payload to sun synchronous orbit, while successfully returning the scramjet-powered stage to the initial launch site. 
This return flight decreases the payload-to-orbit by -19.0kg (-10\%), but removes the need for the costly and time consuming transportation of the SPARTAN after launch, which would be necessary if landing at a downrange location.
During the return flight, the scramjet engines are powered on three times, in total using \returnFuelStandard kg of fuel for the return flight, 17.2\% of the SPARTAN's total fuel.

It was found that when the fly-back of the SPARTAN is included in the optimal trajectory calculation, the first stage of the launch system pitches in an easterly direction. 
The launch system exhibits a first stage-SPARTAN separation point of \firstsecondSeparationAltStandard km, an increase of 3.0km when compared to the maximum payload-to-orbit trajectory with no fly-back, and a trajectory angle of \firstsecondSeparationgammaStandard $^\circ$, an increase of 2.5$^\circ$. 
This higher separation point allows the first stage to efficiently use 17943kg of fuel, as well as increasing its exergy efficiency to \firstExergyEffStandard \%$\eta$, increases of +758kg (+4.4\%) and +0.308\%$\eta$ (+4.9\%) respectively when compared to the maximum payload-to-orbit trajectory with no fly-back.
In addition to increasing the fuel and exergy efficiency of the first stage, the higher first stage-SPARTAN separation serves to increase the altitude of the SPARTAN at the beginning of its trajectory. This allows the SPARTAN time to increase its bank angle, so that when the SPARTAN descends it is able to change its heading angle rapidly. The SPARTAN maintains a high bank angle throughout its trajectory, executing a banking manoeuvre, and staying close to its maximum dynamic pressure. 
This banking manoeuvre requires higher angles of attack, increasing the drag of the SPARTAN, but also reduces the ground distance necessary for the return of the SPARTAN, decreasing the amount of fuel necessary for fly-back and increasing the overall efficiency of the SPARTAN. 
At the end of its acceleration, the SPARTAN was found to exhibit a pull-up manoeuvre before the separation of the third stage, in a similar fashion to the maximum payload-to-orbit trajectory with no fly-back. 

The fly-back of the SPARTAN is found to be separated into three stages; an initial turn, a boost phase, and an approach. 
The initial turn takes place immediately after separation, and consists of the SPARTAN banking heavily in order to manoeuvre the heading angle back towards the initial launch site. 
During the boost-skip phase the SPARTAN exhibits multiple `skipping' manoeuvres. These skipping manoeuvres have been shown in previous literature to extend the flight range of hypersonic vehicles\cite{Moshman2014,Darby2011,Toso2015,Tetlow1992,Eggers1957,Kanda2007,Chai2015}, and serve to reduce the amount of fuel used during the fly-back.
In addition, the skipping manoeuvres allow the scramjet engines to be powered on at the points where the specific impulse of the C-REST engines are highest, maximising the performance of the SPARTAN, and minimising the fuel necessary for return. 
During the approach phase, the trajectory of the SPARTAN is smoothed, and the SPARTAN glides to the landing point. 
 The optimal trajectory terminates when SPARTAN reaches 1km altitude at a velocity of 120m/s. After this point, it is assumed that the SPARTAN lands on a traditional runway.  
This result indicates that it is feasible to return a hypersonic launch vehicle separated at a high Mach number to its initial launch site, and that a cost efficient mission profile for a rocket-scramjet-rocket launch system is attainable.  

The sensitivity of the launch system to various design parameters has been investigated. 
The payload-to-orbit sensitivity of the launch system to variations in the specific impulse, drag and structural mass of the SPARTAN was found to decrease when fly-back is included, compared to the sensitivity study with no fly-back. This decreased sensitivity indicates that the fly-back of the SPARTAN offsets some of the payload-to-orbit variation due to changes in these parameters. 
It was found that the first stage-SPARTAN separation conditions do not exhibit clear trends with SPARTAN performance when fly-back is included, in contrast to the trade-offs observed in Section \ref{sec:sensitivityNoReturn}. The disappearance of these trends indicates that when the fly-back of the SPARTAN is included, the first stage-SPARTAN separation point is determined by a more complex trade-off. This trade-off involves the banking and manoeuvrability of the SPARTAN at the start of its acceleration, which affects the efficiency of the return flight. 
