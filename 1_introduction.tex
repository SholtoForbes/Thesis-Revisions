% 1_introduction.tex

\cleardoublepage
\chapter{Introduction}


  
  	%\textcolor{red}{ XXX check velocity vs speed while im reading}
  	
  	
  	\textcolor{red}{ XXX check airbreathing vs partially-airbreathing}
  	\textcolor{red}{ XXX sort out hyphenation in single stage-to-orbit and two-stage-to-orbit}
  	
  	In recent years, the space sector has seen a significant shift in the paradigm of space launch system design. 
  	The sector has moved towards privatisation, with new and innovative launch systems competing to offer the most cost-efficient and reliable launches. 
  	The sector has also seen a split between those who produce large satellite launchers and those who produce small satellite launchers.
  	For large payload launchers, reusability is a major focus in the design of new launch systems, with the purpose of making a launch system cost efficient over multiple launches\cite{Faa2018}. 
  	For small payload launchers, reusability is more complex than for large launchers, as the additional systems necessary for reusability add a larger fraction of system mass, and require a proportionally larger fuel mass. 
  	Consequently, the focus of small launch system design is currently on producing expendable launch systems as cheaply and efficiently as possible, using state of the art technologies such as 3D printing to expedite the process and minimise cost\cite{Niederstrasser2015}.
  	However, if reusability is able to be successfully integrated into small launch system design, it has the potential to increase the cost efficiency and launch flexibility, potentially opening up the small satellite market significantly. 
  	
  	
  	
  	A potential candidate for integrating reusability into small satellite launch systems is the use of airbreathing engines\cite{Smart2009a,Ketsdever2010}.
Airbreathing engines produce higher specific impulse than rockets, and do not require oxidiser to be carried on-board a launch vehicle\cite{Smart2010}.  	 
  	The higher efficiency and reduced propellant mass of airbreathing vehicles allows the additional mass of the systems necessary for reusability to be mitigated\cite{Curran2003}. An airbreathing vehicle can be designed in a similar fashion to a conventional aircraft, with wings, stabilisers and ailerons\cite{Shaughnessy1990,Preller2017b}. A vehicle designed in this fashion has a high lift-to-drag ratio, and good manoeuvrability, allowing for a return flight and landing on a conventional landing strip\cite{Preller2017b}. This style of return removes the need for transport, enabling a fast turn-around and cost-efficient re-use. 
  	
  	The primary airbreathing engines in consideration for launch vehicles are ramjet and scramjet engines, \textcolor{red}{and rocket-based combined cycle engines that combine multiple engine cycles for operation over a wider range of flight conditions}\cite{HeiserWilliamPratt1994,Kors1988}. These engines offer good efficiency and have operational regimes that allow them to effectively accelerate a launch vehicle over a range of Mach numbers. 
  	Ramjets and scramjets rely on the high speed of the aircraft to compress the flow of air entering the engine before combustion.  Ramjets slow the air to subsonic speeds before combustion and are limited to operation at low Mach numbers, whereas scramjets keep the flow supersonic throughout, and operate within the hypersonic regime, above Mach 5. 
  	These engines have limited operational regimes, and require atmospheric flight in order to take oxidiser from the air. These operational constraints mean that a launch system cannot be solely powered by airbreathing engines. Rocket power is necessary for at least the exoatmospheric portion of the trajectory. As a result, the designs of airbreathing launch systems require rocket stages at reach orbit\cite{Smart2009a}, \textcolor{red}{and if high-speed airbreathing engines are used in the launch system}, rocket power is also desirable for accelerating scramjet accelerator to minimum operational speed, as the alternative is using \textcolor{red}{different types of lower-speed airbreathing engines sequentially}\cite{Smart2009a}, which is weight and cost intensive. 
  	
  	\textcolor{red}{These various propulsion systems may all be integrated into a single stage-to-orbit spaceplane that is capable of launching, placing payload in orbit, reentering, and returning to a suitable landing site\cite{Argus,Powell1991,Trefny1999,Roche2000,Pescetelli2012,Young2006,Bradford2000,Hyperion}. Alternatively, they may be separated into multiple stages, similarly to typical, fully rocket-based launch systems\cite{Wilhite1991,Fujikawa2017,Mehta2001,Takahashi1997,Aberleen,Germain2001,Eklund2012,Bradford2002,Kimura1999,Preller2018a}. Single stage-to-orbit launch vehicles are fully-reusable, and have the potential to be extremely cost-efficient, if they are able to be reused for many missions\cite{Mcclinton2008}. However, their development cost is likely to be very high, and they are generally suited for launching large amounts of payload-to-orbit\cite{Argus,Powell1991,Trefny1999,Roche2000,Pescetelli2012,Young2006,Bradford2000,Hyperion}, a market which is now extremely competitive thanks to the relatively recent advent of large, partially reusable, rocket-based launch systems. Multi-stage airbreathing launch systems, however, have the potential to bring cost-efficient reusability to small payload launchers\cite{Preller2017b}, particularly if they are able to stage at high speeds\cite{Mcclinton2008}. Airbreathing systems scale more efficiently than rockets, meaning that the systems needed for reusability in a small launcher do not require the launcher system to be dramatically increased in size, one of the primary limitations on reusability in small rocket-based launch systems. For this reason, partially-reusable two and three-stage airbreathing launch systems are being investigated for small satellite launch\cite{Preller2017b}. To date, two-stage launch systems have primarily been studied over three-stage launch systems, however, three stage launchers have recently been espoused as potentially offering advantages in reusable mass and mass-to-orbit efficiency\cite{Preller2017b}.}
  	
  	\textcolor{red}{There are as of yet no airbreathing launch systems that have progressed past the research and very early design stages\cite{Argus,Powell1991,Trefny1999,Roche2000,Pescetelli2012,Young2006,Bradford2000,Hyperion,Wilhite1991,Fujikawa2017,Mehta2001,Takahashi1997,Aberleen,Germain2001,Eklund2012,Bradford2002,Kimura1999,Preller2018a}. The development and analysis of trajectories is a crucial part of this early design process, and is a complex task for an airbreathing launch system if the efficiency of the launch system is to be maximised.}
  	 A trajectory must be calculated that allows the launch system to achieve its objective of placing the maximum payload into orbit, while recovering any reusable stages and adhering to the structural, heating and propulsive limitations of the vehicle\cite{Bulirsch1995}.  
  	\textcolor{red}{ In order to maximise the efficiency of the launch system, and thus the payload-to-orbit, there are complex trade-offs in the performance of the launch system that must be taken into account:} the airbreathing engines of a ramjet or scramjet-powered stage require high dynamic pressure to operate effectively, and airbreathing stages are generally designed for high lift-to-drag. Conversely, rocket-powered stages operate more efficiently at higher altitude, and are generally designed for weight and cost efficiency. For airbreathing launch systems, the various \textcolor{red}{distinct operation modes and phases involved} during launch require trade-offs in engine efficiency and thrust generation, stage mass, and vehicle aerodynamics. These factors require the launch trajectory of the system to be thoroughly simulated and optimised, to ensure that the launch vehicle is operating effectively. 
  	 
  
  	 \textcolor{red}{
  	   For single-stage-to-orbit airbreathing launch systems, trajectories have been developed and studied in detail\cite{Argus,Powell1991,Trefny1999,Roche2000,Pescetelli2012,Young2006,Bradford2000,Hyperion}, in particular due to a multitude of studies in the 1980s and 90s, the most prominent of which was the National Aerospace Plane (NASP), which led, in part, to the Hyper-X program and the X-43 flight experiment\cite{Mcclinton2008}. These studies show complex trade-offs that occur in a maximum efficiency airbreathing vehicle trajectory, including trade-offs between the operational efficiencies of the various engine modes. These trajectories often show flight at maximum dynamic pressure or maximum thermal loading, to maximise the efficiency of there airbreathing engines, before a pull-up to orbit, sometimes initially under airbreathing power. These trajectory analyses generally assume that return is possible, due to the high speed and manoeuvrability of the vehicles allowing flexible reentry, and their on-board low-speed engines allowing propulsion and cruise in-atmosphere. 
  	}
  	
  	   \textcolor{red}{
  	  Compared to single-stage airbreathing systems, multi-stage airbreathing launch system trajectories are far more complex, with even more complex trade-offs between the operational modes due to these modes being separated into mechanically distinct stages\cite{Bulirsch1995}. The trajectories of two-stage-to-orbit airbreathing launch systems that have been studied generally exhibit constant dynamic pressure flight to maximise the operation of the airbreathing engines, and a pull-up at the end of the airbreathing trajectory. The return trajectories of two-stage-to-orbit launch systems are generally performed under cruise power, utilising turbojet, or sometimes ramjet engines\cite{Wilhite1991,Mehta2001,Eklund2012,Bradford2002}, or land at a point downrange\cite{Takahashi1997}. There is considerable disparity between studies over whether constant dynamic pressure flight is optimal, and if a pull-up under airbreathing power is optimal\cite{Wilhite1991,Fujikawa2017,Mehta2001,Takahashi1997,Aberleen,Germain2001,Eklund2012,Bradford2002,Kimura1999,Preller2018a}. No studies have attempted a detailed investigation into the trade-offs between the stages, and the optimal trajectory is not well understood. 
  	   For three-stage-to-orbit launch systems the launch trajectory is more complex again, due to there being two separation points at which to consider trade-offs, and each stage generally only having a single engine type. Three-stage-to-orbit systems are by far the least studied of the airbreathing launcher configurations, and have to this point only been designed around prescribed trajectory shapes that constrain the hypersonic airbreathing stages to constant dynamic pressure\cite{Kimura1999,Preller2018a}. It is currently unknown what the optimal trajectory shape for a three-stage launcher is, and if it is possible to return the second stage of a launch system of this type to its initial launch location. The presence of a first-second stage separation, high second-third stage separation speeds, lack of low speed engines on the second stage, and small third stage rocket make the calculation of an optimal trajectory shape for a three-stage airbreathing launcher a unique problem, with extremely complex trade-offs. The study of two-stage-to-orbit trajectories is somewhat useful in this regard, however, the disparities between the various studies that have been conducted, and the lack of clear studies on the trade-offs between the performance of the launch system stages makes the development of an maximum efficiency trajectory solely from previous studies untenable. 
  	}
  	   
  	     	  	\begin{figure}[ht]
  	     	  		\centering
  	     	  		\includegraphics[width=0.7\linewidth]{figures/1_introduction/project-spartan}
  	     	  		\caption{The SPARTAN scramjet-powered accelerator\cite{BBC}.}
  	     	  		\label{fig:project-spartan}
  	     	  	\end{figure}
  	   \textcolor{red}{
  	   This work aims to expand our knowledge on the operation of airbreathing launch systems, by developing a trajectory for a partially-reusable rocket-scramjet-rocket small satellite launch system, of a design based on the SPARTAN launch system under development by The University of Queensland and Hypersonix\cite{Preller2017b}.
  	   This trajectory is developed using modern and robust optimal control theory techniques, that are able to calculate the optimised trajectory profile for a launch vehicle in a robust and computationally efficient manner, allowing a trajectory to be calculated in which the flight path of each individual stage is considered simultaneously to produce a maximum-payload trajectory\cite{Betts1998}. 
  	   Optimal control is able to produce an optimised trajectory that satisfies the specific structural and flight constraints of the vehicle being simulated, allowing the physical limitations of the vehicle, such as heating and structural loading limits, to be imposed\cite{Betts1998}.
  	   This optimal trajectory profile is investigated to give insights into the trade-offs between the rocket and airbreathing stages that are unique to this type of launch system. This manner of optimal trajectory analysis allows for generalities to be made about the nature of the trajectory shape, and for the understanding of the performance of this type of launch system to be improved.
  	   This trajectory analysis is intended to both aid in the ongoing design process of multi-stage launchers, by characterising the performance needs of a rocket-scramjet-rocket launch system throughout its trajectory, as well as to stand on its own merit by indicating the best possible trajectory shape for this type of launch system. 
  	}
  	   
  	   

  	  	
  	  	

  \section{Research aims}

    The aim of this work is to design the trajectory of a rocket-scramjet-rocket small satellite launch system. The purpose of this optimised trajectory is to maximise the payload-to-orbit capabilities of the launch system, thereby also maximising the cost efficiency of the system. The optimal trajectory will be utilised to assess the feasibility of return flight, as well as to determine the impact of key vehicle design parameters on the performance of the launch system. 
 
    
\vspace*{10pt}
    \noindent These aims will be achieved by addressing the following objectives:
    \begin{enumerate}
    	 \item \emph{Development of a detailed design and aerodynamic simulation for a rocket-scramjet-rocket launch system.}
    	 
    	   A detailed launch system design and robust dynamic simulation are required in order for optimal control to be applied to a launch system. The design must be representative of a standard rocket-scramjet-rocket launch system for the optimal trajectory results to be generally applicable. The dynamic simulation must be accurate and robust in order for the optimised trajectory to be meaningful. \\

\item \emph{Calculation of the maximum payload-to-orbit trajectory for a rocket-scramjet-rocket launch system using optimal control, with and without fly-back.}

The optimal trajectory shape of a multi-stage rocket-scramjet-rocket system is sensitive to the design and aerodynamic characteristics of each stage, and cannot be easily assumed. The use of optimal control techniques allows a maximum-payload trajectory to be calculated with few assumptions as to the general shape of the trajectory. The inclusion of the fly-back of the scramjet stage in the trajectory optimisation allows the impact of the fly-back to be minimised.\\

      \item \emph{Analysis of the sensitivity of the maximum payload-to-orbit trajectory to variations in key design parameters of the launch system} 

	The optimal trajectory shape and maximum payload-to-orbit are dependent on the design of the launch system. 
	Assessing the sensitivity of the optimised trajectory shape and payload-to-orbit to key aerodynamic and propulsive properties allows the relative impacts of various design parameters to be calculated and contrasted, and for the optimal trajectory shape to be investigated. \\

    

    \end{enumerate}

  \clearpage
  \section{Thesis Outline and Contributions}

    \textcolor{red}{XXX Update}

    \subsubsection*{Chapter 2 - Literature Review}

      A review of literature related to the various aspects of this study is presented. The theory behind scramjet propulsion is outlined, followed by a background of reusable and small satellite launch systems. A review of the trajectories of partially-airbreathing launch systems is presented, comparing the optimised trajectories of various conceptual vehicles. An overview of optimal control techniques is presented, with particular emphasis on the pseudospectral method of solving optimal control problems, which is employed within this study. Lastly, an overview of the optimal control and aerodynamic solvers that are used in this study is presented.
      

    \subsubsection*{Chapter 3 - Launch Vehicle Design and Simulation}

      The design, aerodynamics and engine models of all three stages are detailed. the SPARTAN scramjet-powered stage is presented first, followed by the first and third stages. The design of each stage is shown, along with sizing and mass breakdowns. The propulsion model used for each stage is detailed, along with the modelling and interpolation schemes used. The aerodynamic characteristics and simulation methodology of each stage is presented, and the process for trimming each vehicle is specified. 
      
      
      \subsubsection*{Chapter 4 - LODESTAR}
      
      The method used for the simulation and optimisation of the trajectory is presented, including the details of the trajectory analysis program, LODESTAR, which has been developed for this study. The specifics of the optimal control methodology are presented. The simulation methodology is detailed, along with the construction of the optimal control simulation for the mission used in this study. The specific set-up of the optimal control program is detailed for each trajectory stage, specifying the costs and constraints which drive the optimal control solver. Finally, the methods for validating the final solutions are specified.
      
      \subsubsection*{Chapter 5 - Optimised Ascent Trajectory}
      
Optimised trajectories, designed using LODESTAR, are presented. A trajectory is designed in which the SPARTAN flies at a constant dynamic pressure, for comparison purposes. 
 A maximum payload-to-orbit trajectory is created and it is found that an increase in altitude at the stage separation points significantly improves payload-to-orbit.
 This trajectory is compared and contrasted to the constant dynamic pressure trajectory to determine the sources of the performance increase.  
 Key vehicle design parameters are varied. The trends in maximised payload-to-orbit and trajectory shape are analysed to study the relative impact of the design parameters on the performance of the launch system. 
 
 
      
      \subsubsection*{Chapter 6 - Optimised Trajectory Including Fly-Back}
      
      The trajectory of the launch system is optimised for maximum payload-to-orbit, including the fly-back of the SPARTAN to its initial launch location. 
      It is found to be necessary to reignite the scramjet engines during the return flight of the SPARTAN to achieve fly-back.
      the SPARTAN is found to bank during acceleration to lessen the fuel consumed during the return flight.
      The trajectories with, and without, fly-back are compared to determine the impact of SPARTAN fly-back on the performance of the launch system
      In a similar fashion to Chapter 5, the effects of key vehicle parameters on the optimised trajectory are studied. The sensitivity of the optimised trajectory and payload-to-orbit are analysed, with emphasis on how the fly-back trajectory is affected by the varied vehicle parameters.
      
     
      

    \subsubsection*{Conclusions and Recommendations}

      The body of this thesis concludes by summarising the most significant findings from this work. Recommendations for future work are made. 
      
      
      
      \chapter{Background}
    
    
    
    
    
    
  
    
    \begin{figure}[ht]
    	\centering
    	\includegraphics[width=0.9\linewidth]{figures/2_literature-review/LaunchVehicles}
    	\caption{Comparison of Blue Origin and SpaceX partially-reusable launch systems with existing and historic launch systems\cite{BlueOrigin}.}
    	\label{fig:LaunchVehicles}
    \end{figure}
    
    Launch system technologies have progressed rapidly over the last 60 years. From the early vehicles based on intercontinental ballistic missile technology such as the Thor based launch systems, capable of launching 40-400kg to LEO in the 1960s, to the more modern Atlas V based systems of the 2000s capable of launching 9750-18500kg to LEO\cite{Kendall2010}. 
    The materials, propulsion technology, aerodynamics and guidance algorithms have all improved significantly, enabling  rockets to become more efficient, cheaper to produce, and more reliable. As the demand for satellite launches grows, and the cost of development of launchers becomes cheaper, the potential for profiting from space launches increases. 
    This has driven a large portion of the space flight industry to move towards privatisation, with a heavy focus on reusable technology. 
    \begin{figure}[ht]
    	\centering
    	\includegraphics[width=1\linewidth]{"figures/2_literature-review/FalconTrajectory"}
    	\caption{The trajectory of the Falcon Heavy\cite{FalconHeavy}.}
    	\label{fig:FalconTrajectory}
    \end{figure}
    
    Reusable launch technology has been possible for many years, in the form of the space shuttle. However, the space shuttle was weighed down by large launch costs and operational complexity, and was not a commercial success\cite{Launius2006}. Recently, reusable launchers have become the focus of many of the largest private launch companies, as reusability becomes more achievable due to technological advances\cite{Foust2018,Mosher2018}. The SpaceX Falcon 9 and Falcon Heavy have been demonstrated on multiple occasions, landing booster stages successfully, and re-flying reused boosters multiple times\cite{Foust2018}. In the near future the Blue Origin New Glenn is planned\cite{Foust2018}, with potentially the Airbus Adeline to follow (to be used on the Ariane 6)\cite{Adeline}. The Falcon and New Glenn launchers are large rockets, and are shown in Figure \ref{fig:LaunchVehicles} in a size comparison with other, expendable rocket systems, including the Ariane 5, which is projected to be similarly sized to the Ariane 6. 
     
    
    
    
    
    
    
    \begin{figure}[ht]
    	\centering
    	\includegraphics[width=0.7\linewidth]{figures/2_literature-review/visuel_adeline1}
    	\caption{The trajectory of the Ariane featuring Adeline\cite{Adelineb}.}
    	\label{fig:visuel_adeline1}
    \end{figure}
    
    The purpose of reusing launch vehicles is to reduce the cost-over-time of the reused components drastically, which subsequently allows the cost of individual launches to be reduced\cite{RichardWebb2016}. Reducing costs lowers the barrier of entry for space launches, potentially improving the diversity of space-based enterprises. 
    Reusing launch system components also allows faster turnaround times for launches, as refurbishment of stages is much faster than manufacturing stages from scratch. Reduced turnaround times are key for improving mission scheduling, allowing satellites to be launched more rapidly, on a more flexible time frame. 
    For a launch vehicle to be reusable, it must necessarily have the ability to come back to Earth safely, without damage to major system components. In addition, it is preferable for the vehicle to return to its initial launch site, to reduce the cost and time necessary for transport. The trajectories of the Falcon Heavy and Adeline are shown in Figures \ref{fig:FalconTrajectory} and \ref{fig:visuel_adeline1}, \textcolor{red}{ illustrating two radically different approaches to reusability; the exoatmospheric manoeuvre and booster-deceleration of the Falcon; and the in-atmospheric fly-back of the Adeline, which only recovers the engine component of the system. While these reusability approaches use different technologies, they are aiming to achieve the same thing: the recovery of a significant portion of the launch system, with as much of the launch system as possible returned to the initial launch location.} These return flights require the addition of system components which allow the reusable stage to fly to a specified landing point. Control surfaces\cite{Clark2018}, structural components\cite{Sarigul2003}, additional fuel\cite{Sarigul2003}, and in the case of the Adeline, additional engines\cite{Adeline}, must be incorporated within the reusable launch vehicle designs. The additional weight that these components contribute further increases the fuel and structural mass necessary to initially accelerate the reusable stage. The impact of reusability on the mass and cost of the vehicle is minimised when the velocity at the initiation of the return trajectory is decreased. Because of this mass increase on any stage which is to be designed to be reusable, most current reusable launch vehicle designs include only reusable first stages, with later stages being expendable. 
    
    
    
    
    \section{Small Satellite Launchers}\label{subsec:smallsats}
    
    
    
    \afterpage{
    	\begin{landscape}% Landscape page
    		\begin{table}
    			
    			\begin{tabularx}{\linewidth}{|X|X|X|X|X|X|X|X|}
    				
    				\hline  \textbf{Launcher} & \textbf{Company} & \textbf{Country} & \textbf{Payload Capacity} & \textbf{Cost/Kg} (USD) & \textbf{Availability }& \textbf{Stages \& Propulsion} & \textbf{Reusability}  \\ 
    				\hline  Electron\cite{Electron} & RocketLab & NZ/USA & 150Kg to SSO & \$32,600 & Available & Rocket-rocket & No \\ 
    				
    				\hline Bloostar\cite{Bloostar} & Zero2Infinity & Spain & 100kg to SSO & \$40,000 &  - & Balloon-rocket-rocket-rocket & No \\ 
    				
    				\hline Eris\cite{Gilmour} & Gilmour Space Technologies & Aus/SG& 380kg to LEO  &\$23,000-38,000   & Q4 2020 && No\\ 
    				\hline Intrepid-1\cite{Intrepid} & Rocket Crafters  & USA & 376kg to SSO & \$23,936 & Q1 2019 & Rocket-rocket & No \\ 
    				\hline KZ-1A\cite{FT1} & CASIC  &  China & 250kg to SSO  & - & - & Rocket-rocket & No \\ 
    				\hline  Vector-H\cite{Vector}& Vector Space Systems  & USA & 160kg to LEO  & \$21,875 & 2018 & Rocket-Rocket-(Third rocket optional)& No\\ 
    				\hline SMILE\cite{Kuhn2017} & NLR &  EU & 50kg & $<$\$50,000 & -  & -& -\\ 
    				\hline 
    				Firefly $\alpha$\cite{firefly}& Firefly Aerospace &  USA & 630kg to SSO & - & 2019  & Rocket-Rocket& No\\ 
    				\hline 
    				LauncherOne\cite{Virgin} & Virgin Orbit & UK & 300kg to SSO & \$33,000 & - & Aircraft-rocket-rocket & Aircraft\\ 
    				\hline XS-1\cite{DARPA2017} & Boeing & USA & - & - &  - & - & First Stage\\ 
    				\hline
    				500R\cite{500R} & Orbital Access  & UK & 500kg to SSO & - & - & Aircraft-Rocket & Fully Reusable \\ 
    				\hline
    			\end{tabularx} 
    			\caption{A selection of the small satellite launchers which are operational or in development.}
    			\label{tab:smallsats}
    		\end{table}
    	\end{landscape}
    }
    
    
    The vast improvements in computational technologies in recent years have allowed satellites to decrease in size and cost to a large degree. These factors have lowered the barrier of entry into small satellite manufacturing significantly, driving a surge in the demand for small satellite launches. Many private and public companies are currently developing small satellite launchers, that will allow small satellites to be launched into bespoke orbits on schedules determined by the customer\cite{Faa2018}. The details of a selection of the most promising or innovative of the small satellite launchers currently in active development is shown in Table \ref{tab:smallsats}. Many of these launchers are projected to be available within the next few years, and will offer cost-per-kg comparable to piggybacking on larger launches.
    These small satellite launch systems are mostly expendable, aiming to reduce costs by creating a launch system which is as cost-efficient as possible to produce and launch\cite{Niederstrasser2015}. This often entails making use of conventional, well-tested designs, combined with state of the art manufacturing techniques, such as 3-D printing\cite{Niederstrasser2015,Gilmour}. This method allows for rapid development, although it has an intrinsic cost limit due to the requirement of manufacturing a new launch system for each launch. 
    
    \textcolor{red}{If small satellite systems are able to be designed to be reusable, the cost of launch would be able to be brought down further. However, designing a small satellite for reusability is more complex than for a larger rocket system. The systems used to manoeuvre exoatmospherically and decelerate like the Falcon or New Glenn or fly-back like the Adeline take a relatively higher mass margin as the size of the launcher decreases, in turn increasing the mass, and decreasing the cost efficiency of the launcher. If a cost efficient, reusable design for a small satellite launcher is to be produced, it is likely that a technology must be developed that is specifically geared towards small satellite launch. 
    One of the most promising technologies currently under development, that may enable reusable small satellite launch systems, is high speed airbreathing engines\cite{Smart2009}, in particular the hypersonic scramjet engine.}
    
 
    
    \section{Scramjets}
    
    -XXX have something about how sensitive scramjet engines are
    
    A Scramjet, or supersonic combustion ramjet, is an airbreathing engine design which combusts air at supersonic speeds and is capable of high Mach number operation. 
    Across their operating range, scramjets offer much higher specific impulse than rockets, the only other propulsion system capable of operating efficiently at hypersonic speeds\cite{Billig1993,Cook2003}.
    Scramjets compress air without moving parts, using geometry changes within the engine\cite{Curran2001a}, as well as on the forebody of the vehicle to create inlet shocks which provide the compression required for combustion\cite{Smart2012}. After combustion, the combustion products are expanded through a thrust nozzle, schematically shown in Figure \ref{fig:scramjet}. This is similar in operation to a ramjet engine, though a scramjet does not generate a normal shock, allowing supersonic air to enter the combustor. Maintaining supersonic speeds throughout the engine allows scramjets to operate efficiently at Mach numbers of 5 and greater. 
    Scramjets were proposed in the 1940's\cite{Curran2001} and found to be capable of positive net thrust in 1993\cite{Paull1993}, but have yet to be developed to a level which would allow for commercial application. 
    Though scramjets are still in-development, the potential advantages which they offer over rockets for certain applications are driving their development in a number of institutions\cite{Curran2001b}.
    \begin{figure}[ht]
    	\centering
    	\includegraphics[width=0.7\linewidth]{figures/2_literature-review/scramjet}
    	\caption{A simple schematic of a scramjet engine\cite{Heiser1994}.}
    	\label{fig:scramjet}
    \end{figure}
    
    Scramjet engines are suitable for a number of applications where sustained flight or acceleration is desired at high Mach numbers. 
    The high efficiency of scramjet engines means that significantly less propellant (fuel + oxidiser) is used during flight compared to rocket engines, and consequently, that a much smaller fraction of a scramjet-powered vehicle consists of propellant mass\cite{Curran2003}. 
    The smaller fuel mass fraction of a vehicle powered by scramjet engines mitigates the mass of the vehicle systems, allowing features such as wings, control surfaces, landing gear, and passenger transport capabilities to be included in the vehicle design\cite{Curran2003}. 
    
    \begin{figure}[ht]
    	\centering
    	\includegraphics[width=0.7\linewidth]{figures/2_literature-review/Scramjet-Efficiency}
    	\caption{Characteristic performance for airbreathing and rocket engines with Mach number\cite{Fry2004}.}
    	\label{fig:Scramjet-Efficiency}
    \end{figure}
    
    Theoretically, the operable range of scramjets is wide\cite{Smart2007a}; the specific impulse of a scramjet decreases with velocity, until it is equal to rockets around Mach 19\cite{Fry2004}, as shown in Figure \ref{fig:Scramjet-Efficiency}. 
    However, in practical designs, the operating range for a scramjet engine is far more limited. 
    For a fixed geometry scramjet, the operable region is constrained by the geometries of the forebody of the vehicle, the inlet, and the combustor of the scramjet engine\cite{Smart2010}.  
    The Mach number range of a scramjet engine varies by design, but Mach number ranges of 5-10\cite{Preller2017b}, 7-11\cite{Dalle2014} and 6-10\cite{Bradford2000} have been suggested as appropriate operable regimes for scramjet-powered launch vehicles.
    The operable range of scramjet engines can be improved with mechanisms to vary the geometry of the inlet during flight\cite{Dalle2011}. However, the systems necessary for variable geometry inlets add weight and complexity to the scramjet engine, and can be detrimental to overall system performance\cite{Smart2010}. 
    
    \textcolor{red}{The inclusion of scramjet engines for high efficiency operation in the hypersonic regime may be considerably advantageous for small satellite launch systems. However, the incorporation of these engines within a launch vehicle requires significant additional design complexity over tradition all-rocket systems, due to their unique operational constraint: the requirement to fly in-atmosphere for long periods, at hypersonic speeds. This means that a launch system incorporating scramjet engines must be designed to fly aerodynamically, balancing lift and drag with thrust and aerodynamic control. 
    This is a significantly different design philosophy when compared to traditional all-rocket launch systems, that generally want to exit the atmosphere as quickly as possible, and minimise drag losses while doing so. Because of this significant difference in operation, the designs of airbreathing launch vehicles are likely to be radically different from traditional rocket systems. Novel designs must be developed for airbreathing launch systems, that take advantage of the unique advantages offered by airbreathing engines, while overcoming the challenges that they introduce. 
    	}
    
    
    
    
    
    \textcolor{red}{
    	\section{The Launch System Design Process}
    }
    %ingo really specfically distinuish airbreatrhing from rocket, ie design is much more complex
    
    The design of any launch system is a complex, multi-faceted process, that involves many areas of engineering and scientific expertise. Developing a trajectory for a launch system is simply a part of this process, although it is an integral step in the early launch vehicle design. The analysis of the trajectory of a launch system enables the designers to understand the performance characteristics of the launch system, and drives design decisions that are made throughout the development process. 
    
     Figure \ref{fig:DesignFlow} illustrates the general design process flow for a launch system as defined by NASA\cite{Blair2001}, in which the design stages involve a repeated iterative process between the overall launch system design and compartmentalised design tasks. This compartmentalisation takes place firstly by separating the
    launch system into its hardware and software subsystems, and then into general areas that define the specifications of the subsystem design (design functions). These areas include;
    \begin{figure}[ht]
    	\centering
    	\includegraphics[width=0.7\linewidth]{figures/2_literature-review/DesignFlow}
    	\caption{The general design process flow for a launch vehicle, adapted from Blair et al\cite{Blair2001}}.
    	\label{fig:DesignFlow}
    \end{figure}
    \begin{itemize}
    	\setlength\itemsep{.2em}
    	\item aerodynamics;
    	\item trajectory, guidance and navigation;
    	\item control;
    	\item structures;
    	\item thermal;
    	\item propulsion;
    	\item avionics;
    	\item materials;
    	\item and manufacturing,
    \end{itemize}
    among others\cite{Blair2001}. The design is then further compartmentalised into pertinent areas of specific speciality and expertise (discipline functions) that defines the problems that must be solved, and skills that are necessary for low level design. This process of compartmentalisation allows the effort associated with launch system design to be broken down into manageable amounts, and enables the efficient utilisation of discipline and industrial specialities. The design process is a continuous effort to compartmentalise the work that most be undertaken into manageable tasks, and then to reintegrate and synthesise the results in order to converge on a design that is able to satisfy the overarching mission requirements\cite{Blair2001}. The trajectory design is simply one part of this design process, that can have significantly different levels of impact on the early design stages depending on the type of launcher being designed. 
    
 
    The design of a traditional all-rocket launch system may use quite simplified trajectory models in the early stages, because the general shape of rocket launch system trajectories are well-known and relatively easily modelled.  The primary goals of the trajectory of a rocket-powered launch system are to minimise drag while not exceeding structural and thermal limits, and in general this results in a vertically-launched pitching trajectory, that exits the atmosphere as rapidly as possible while managing velocity at points of maximum structural and thermal stress. 
    
    The trajectory of an airbreathing launch system is much more complex, because there is heavy dependencies and trade-offs between the way in which an airbeathing launch system is flown, and its performance characteristics. Relying on atmospheric air for oxider means that an airbreathing launch system requires sustained atmospheric flight, at relatively high densities to allow for efficient operation of the airbreathing engines. This atmospheric flight requires aircraft-style designs for control and stability, and results in complex trade-offs between the performance of the airbreathing engines, and the structural and thermal limitations of each stage. In addition, there is trade-offs between the performance of the various engines that are utilised by airbreathing launch systems, which generally mix airbreathing and rocket engines that have significantly different performance regimes.
    These trade-offs are specific to the types of engines and number of stages being employed by the launch system, and are closely dependent on the trajectory being flown. The determination of the ideal launch trajectory of an airbreathing launch system is an integral step in the early design process, allowing for the design of the launcher and its subsystems to be tailored to capitalise on the ideal flight profile, and for significant performance drivers to be identified early on. 
  This is particularly true because many of the subsystems and technologies necessary to achieve airbreathing space access are in development at an academic level, and advances are necessary in multiple fields before the flight of an airbreathing launch vehicle becomes a reality. The investigation of the trajectory of an airbreathing launch system gives key insights into the design requirements of the launcher and it subsystems, that can drive academic study on the subsystems of airbreathing launchers, such as their propulsion and thermal protection systems, as well as their overall designs. 
    
    
    \textcolor{red}{XXX here I need to include some more specifics - from ingos overall comments 1) that can be linked to the three stage launch system section of lit review}
  