\newglossaryentry{Space}{name={Space},description={The expansive region that exists outside of Earth's atmosphere. The Karman line at 100km altitude above the Earth's surface is commonly represented as the border to space. }}
\newglossaryentry{Launch System}{name={Launch System},description={A system of propulsion systems and vehicles designed for launching to space. }}
\newglossaryentry{Launch}{name={Launch},description={The process of flight to space.}}
\newglossaryentry{Space Sector}{name={Space Sector},description={The companies, government entities, and private individuals involved in space launch, satellite manufacturing, and space and ground operations. }}
\newglossaryentry{Satellite}{name={Satellite},description={An man-made object that has been placed into orbit.}}
\newglossaryentry{Payload}{name={Payload},description={The object or entity that is carried by a launch system for delivery to a particular location or orbit. }}
\newglossaryentry{Airbreathing Engines}{name={Airbreathing Engines},description={An engine that uses atmospheric air as oxidiser.}}
\newglossaryentry{Rocket Engine}{name={Rocket Engine},description={A reaction engine that that uses stored propellant.}}
\newglossaryentry{Rocket Propellant}{name={Rocket Propellant},description={The reaction mass used by a rocket engine for propulsion, usually a combination of fuel and oxidiser.}}
\newglossaryentry{Ramjet}{name={Ramjet},description={A supersonic airbreathing engine that uses forward motion to compress air before combustion at subsonic speeds. }}
\newglossaryentry{Scramjet}{name={Scramjet},description={A supersonic airbreathing engine that uses forward motion to compress air before combustion at supersonic speeds. }}
\newglossaryentry{Turbojet}{name={Turbojet},description={An airbreathing engine that uses a compressor to compress air before combustion. }}
\newglossaryentry{Mach Number}{name={Mach Number},description={A dimensionless value representing air speed as a multiple of the local speed of sound. }}
\newglossaryentry{Trajectory}{name={Trajectory},description={The flight path of a launch vehicle. }}
\newglossaryentry{Optimal Control Theory}{name={Optimal Control Theory},description={A seto of mathematical techniques that find a control law over a period of time in order to maximise a given objective function.}}
\newglossaryentry{Stage}{name={Stage},description={A distinct seciton of a launch system, containing its own engine and propellant. }}
\newglossaryentry{Dynamic Pressure}{name={Dynamic Pressure},description={An expression of the increase in a fluids pressure due to velocity. }}
\newglossaryentry{Nozzle}{name={Nozzle},description={The section of an engine that is used to expand and accelerate the gases from the combustion chamber to produce propulsive force.}}
\newglossaryentry{Combustion Chamber}{name={Combustion chamber},description={The section of an engine where fuel and oxidiser are combined and burned.}}
\newglossaryentry{Inlet}{name={Inlet},description={The section of an airbreathing engine where air is drawn in.}}
\newglossaryentry{Shock Wave}{name={Shock Wave},description={A sharp pressure, temperaure and density gradient caused by a vehicle moving faster than the speed of sound. }}
\newglossaryentry{Subsonic}{name={Subsonic},description={Less than the speed of sound.}}
\newglossaryentry{Supersonic}{name={Supersonic},description={Greater than the speed of sound.}}
\newglossaryentry{Hypersonic}{name={Hypersonic},description={Much greater than the speed of sound, usually denoted as speeds greater than Mach 5. }}
\newglossaryentry{Transonic}{name={Transonic},description={Speeds transitioning between subsonic and supersonic, close to Mach 1.}}
\newglossaryentry{Specific Impulse}{name={Specific Impulse},description={A measure of the fuel usage efficiency of an engine, thrust produced divided by flow rate.}}
\newglossaryentry{Thrust}{name={Thrust},description={The propulsive force generated by an engine.}}
\newglossaryentry{Reusable Launch System}{name={Reusable Launch System},description={A launch system capable of being used for launch more than once.}}
\newglossaryentry{Performance}{name={Performance},description={The measured degree to which a system accomplishes its objectives. }}
\newglossaryentry{Fly-back}{name={Fly-back},description={The process of returning part of a launch system to its original launch location. }}
\newglossaryentry{Shooting Method}{name={Shooting Method},description={A type of optimal control determination that involves forward time simulations using a varied set of controls. Includes the single and multiple shooting methods.}}
\newglossaryentry{Collocation Method}{name={Collocation Method},description={A type of optimal control determination that approximates states and controls as polynomials, and constrains the dynamics of the system at distinct collocation points. Includes the psudospectral methods of collocation.}}
\newglossaryentry{State}{name={State},description={The time varying parameters of a system that are not directly controlled.}}
\newglossaryentry{Control }{name={Control },description={The time varying parameters of a system that are directly controlled. }}
\newglossaryentry{Objective Function}{name={Objective Function},description={The function that describes the metric to which a given system is to be optimised. }}
\newglossaryentry{Constraints}{name={Constraints},description={The limitations on the states or controls of a system. May be functions of states and controls.}}
\newglossaryentry{Aileron}{name={Aileron},description={A mechanically deflected surface used to control a vehicle aerodynamically.}}
\newglossaryentry{Empirical}{name={Empirical},description={Values or trends derived from experiment and experience rather than, or in addition to, theoretical calculations. }}
\newglossaryentry{Fidelity}{name={Fidelity},description={The degree of accuracy of reproduction or representation.}}
\newglossaryentry{Operational Regime}{name={Operational Regime},description={The range of Mach numbers over which an engine or vehicle is capable of functioning.}}
\newglossaryentry{Aerodynamic Loads}{name={Aerodynamic Loads},description={The forces on an aircraft due to the pressure of the incoming air.}}
\newglossaryentry{Orbit}{name={Orbit},description={A repeating, curved trajectory path that a satellite takes around the Earth.}}
\newglossaryentry{Computational Fluid Dynamics}{name={Computational Fluid Dynamics},description={Numerical models used to calculate the flow of a fluid to produce quantitative analysis. Involves discretising a mesh of points or elements and time-marching the numerical solution.}}
\newglossaryentry{Mesh}{name={Mesh},description={The discrete set of points or elements used for computational fluid dynamics. }}
\newglossaryentry{Angle of Attack}{name={Angle of Attack},description={The incidence angle of an aircraft relative to oncoming air.}}
\newglossaryentry{Bank Angle}{name={Roll Angle},description={Also known as roll angle. The angle at which an aircraft has been turned about its centre axis (nose-to-tail) relative to the horizontal.}}
\newglossaryentry{Downrange}{name={Downrange},description={A horizontal distance from the launch site. }}
\newglossaryentry{Viscous Effects}{name={Viscous Effects},description={The aerodynamic effects of the viscosity (resistance to deformation or flow) of the air, generally significant near the body of the vehicle where a boundary layer results.}}
\newglossaryentry{Circularisation}{name={Circularisation},description={A burn that raises the perigee of an orbit so that the orbit is circular. }}
\newglossaryentry{Hohmann Transfer}{name={Hohmann Transfer},description={Generally the most fuel efficient manoeuvre to get from one circular orbit to another. Two burns are utilised, the first into an elliptical orbit with apogee equal to the final desired orbital radius, and the second to circularise at apogee. }}
\newglossaryentry{Nonlinear Programming}{name={Nonlinear Programming},description={The process of solving an optimisation problem subject to a nonlinear objective value and/or nonlinear constraints. Computing an optimal trajectory is generally a nonlinear optimisation problem. }}
\newglossaryentry{Cruise}{name={Cruise},description={Flight at relatively constant speed and altitude.}}
\newglossaryentry{Glide}{name={Glide},description={Flight without propulsion.}}
\newglossaryentry{Pull-up}{name={Pull-Up},description={The raising of altitude using aerodynamic lift. }}
\newglossaryentry{Reentry}{name={Reentry},description={The entry of a spacecraft into the atmosphere from space.}}
\newglossaryentry{Aerothermal Loading}{name={Aerothermal Loading},description={The heating of an aircraft due to the compression and friction of incoming air. }}
\newglossaryentry{Boost-Skip}{name={Boost-Skip},description={A supersonic or hypersonic trajectory involving repeated periodic altitude raising manoeuvres (‘skips’), with periods of interspersed thrust from an airbreathing engine (‘boosts’). Generally used to maximise range for minimum fuel.}}
\newglossaryentry{On-Orbit}{name={On-Orbit},description={Manoeuvres or services performed while orbiting. }}
\newglossaryentry{Robustness}{name={Robustness},description={The degree to which a solver is able to converge from various initial guesses.}}
\newglossaryentry{Wingspan}{name={Wingspan},description={The maximum distance between the two wing-tips of a vehicle. }}
\newglossaryentry{Space-Plane}{name={Space-Plane},description={An aerospace vehicle with aeroplane-like aerodynamic surfaces that is also designed to be operated exoatmospherically. }}
\newglossaryentry{Thermal Protection System}{name={Thermal Protection System},description={The system that protects a vehicle from the aerothermal heating it generates during flight. }}
\newglossaryentry{Trim}{name={Trim},description={The deflection of control ailerons in order to counteract the moment generated by an aircraft, to maintain attitude. }}
\newglossaryentry{Centre of Gravity}{name={Centre of Gravity},description={The location within a body at which the force due to gravity appears to act. Generally equated (with the assumption of a uniform gravity field) to the centre of mass, the averaged mass location within a body at which all uniform forces may be approximated to act. }}
\newglossaryentry{Carbon-Carbon}{name={Carbon-Carbon},description={Carbon fibre in a matric of graphite, used for high temperature structural applications and heat shielding. }}
\newglossaryentry{C-REST Engine}{name={C-REST Engine},description={Rectangular-to-elliptical shape transition scramjet engines, configured to a conical forebody. }}
\newglossaryentry{Exergy}{name={Exergy},description={The available work that may be extracted from a system. }}
\newglossaryentry{Thrust Vectoring}{name={Thrust Vectoring},description={The manipulation of the direction of thrust of a vehicle, generally through gymballing of a nozzle, flow injector manipulation, or the manipulation of the shape of a nozzle or flow path.}}
\newglossaryentry{Sun Synchronous Orbit}{name={Sun Synchronous Orbit},description={A slightly retrograde, near-polar orbit with a precession rate that matches the orbit of the Earth around the Sun (ie 360$^\circ$ per year).}}
\newglossaryentry{Throttle}{name={Throttle},description={The system that controls the flow of fuel to an engine. }}
\newglossaryentry{Apogee}{name={Apogee},description={The point of an orbit that is farthest from the centre of the Earth.}}
\newglossaryentry{Perigee}{name={Perigee},description={The point of an orbit that is closest to the centre of the Earth.}}



%\newglossaryentry{}{name={},description={}}

% symbols

\glsxtrnewsymbol[description={velocity}]{v}{\ensuremath{v}}
