% frontmatter.tex





\thispagestyle{empty}
\begin{center}
	\centering
  \includegraphics{figures/uq_logo}

  \vspace{50pt}

  \textbf{\Large Trajectory Optimisation of a Partially-Reusable Rocket-Scramjet-Rocket Small Satellite Launch System}

  \vspace{25pt}

  {\large Sholto O. Forbes-Spyratos }
    
    \vspace{5pt}

  {\large B.Eng. (Mechanical and Aerospace) (Hons. I) \& B.Sc. (Physics)}


  \vspace{60pt}
                                                                                                                                                                                                                                                                                                                                                                                                                                                                                                                                                                                                                                                                                                                                                                                                                                                                                                                                                                                                                                                 
  \vfill

  {\large A thesis submitted for the degree of Doctor of Philosophy at\\ The University of Queensland in 2018}

  \vspace{20pt}

  School of Mechanical Engineering
  
  \textit{Centre for Hypersonics}

  \vfill
\end{center}

\cleardoublepage
\textcolor{red}{
\section*{Abstract}
}
\noindent
 The small satellite industry is expanding rapidly, driving a need for dedicated and cost effective small satellite launchers.  
These small satellite launchers are currently designed to be single-use and low-cost, however, significant improvements in cost effectiveness and responsiveness may be achieved via the introduction of reusability. Unfortunately, the inclusion of reusability into small launch system designs is significantly more challenging than for larger launchers, and novel design methods may be necessary to achieve reliable and efficient re-use. One possibility for accomplishing reusability in a small launch system is the inclusion of airbreathing propulsion systems.

Airbreathing propulsion systems are extremely efficient, and do not require oxidiser to be carried on-board a launch system, bringing considerable operational and design advantages. 
However, airbreathing engines also add complexity to a launch system, with intricate subsystems and limited flight regimes. Because of this complexity, and the relative immaturity of airbreathing engine development, airbreathing launch systems are still in the early stages of design, with various configurations being investigated. Of these configurations, multi-stage systems are likely to be the most applicable to small launch systems, due to the favourable effects of stage separation with the restricted mass margins of small launchers. However, a multi-stage design compounds on the complexity of airbreathing launch systems, and the efficiency trade-offs between the stages of multi-stage airbreathing launch systems, particularly for three-stage systems with a fly-back component, are not well understood. This work aims to aid in the development of multi-stage airbreathing launch systems by developing and studying optimal payload-to-orbit trajectories for a three-stage airbreathing launcher, in order to understand the capabilities of the system and analyse the performance trade-offs between the stages during flight of a maximum payload-to-orbit trajectory with a fly-back of the airbreathing stage.



 A representative launch system is developed, consisting of two rocket stages and scramjet-powered accelerator stage.
This launch system is based on the SPARTAN scramjet-powered accelerator, being developed by the University of Queensland and Hypersonix, with a first stage based on the Falcon-1e, and a bespoke third stage. These stages are designed to the detail required for preliminary analysis, along with aerodynamic, mass and propulsive models.
During the launch of this system, there are complex trade-offs between the performance of each stage that must be accounted for, requiring a robust and accurate trajectory optimisation methodology. 
 A package is developed to calculate the maximum payload-to-orbit trajectory of the representative launch system, designated LODESTAR, which uses optimal control theory to design flight paths. LODESTAR utilises GPOPS-2, a pseudospectral method optimal control software, configured to calculate maximum payload-to-orbit trajectory profiles. Along with the configuration of GPOPS-2, LODESTAR provides a 3 degree-of-freedom simulation of each stage, and tools to verify and examine the optimised solutions produced by GPOPS-2.


 
Launch trajectories are initially simulated assuming that the scramjet accelerator lands at some position downrange. A launch trajectory is simulated in which the scramjet accelerator flies at maximum dynamic pressure as a reference and verification case, as flight at this limit during airbreathing engine operation is often assumed in previous studies to produce a maximum efficiency trajectory. This trajectory achieves a payload-to-orbit of \PayloadToOrbitConstqNoReturn kg, launching to sun synchronous orbit.  
The maximum payload-to-orbit trajectory of the launch system is calculated, and is found to differ significantly from the trajectory in which the scramjet accelerator is constrained to constant dynamic pressure. 
 The scramjet accelerator is found to deviate from its maximum dynamic pressure at both stage separation points, and for a segment in the middle of its trajectory.
 The higher separation points result in the efficiency of the scramjet accelerator reducing, but increase the efficiency of the rocket stages, improving the overall efficiency of the system. 
  Additionally, an altitude raising manoeuvre is performed in a region where the specific impulse of the scramjet engines is relatively homogeneous with varied flight conditions, resulting in a further small performance increase. Overall, flying an optimal trajectory increases the payload-to-orbit of the system launching to sun synchronous orbit to \PayloadToOrbitStandardNoReturn kg, an increase of 59.1\% compared to a trajectory in which the scramjet accelerator flies at maximum dynamic pressure.  
 
 
 Next, the fly-back of the scramjet accelerator is included within the trajectory optimisation, and simulated to produce a non-intuitive maximum payload-to-orbit flight profile. 
 It is found that the scramjet accelerator is able to achieve fly-back, but that it must ignite its scramjet engines during its return flight, causing the fly-back to become an important consideration in the optimal trajectory design. When the fly-back is included, the first stage pitches easterly, rather than northerly. The scramjet accelerator banks heavily throughout its acceleration to manoeuvre to polar inclination, decreasing the performance of the scramjet accelerator, but also reducing the amount of fuel used during fly-back, for a net performance gain. 
The fly-back is found to exhibit multiple `skipping' manoeuvres, which serve to increase the range of the scramjet accelerator, minimising the fuel necessary during the return flight. In addition, the scramjet engines are powered on at the troughs of the first three skips, corresponding to the points of highest possible specific impulse. This optimal trajectory for the launch system is able to deliver \PayloadToOrbitStandard kg of payload to sun synchronous orbit while successfully returning the scramjet accelerator to its initial launch site.  

Finally, a study is conducted to quantify the sensitivity of the launch system to variations in key design parameters. 
The sensitivities of coupled design parameters are compared, to quantify their relative impacts on the performance of the launch system. The magnitudes of these relative impacts are assessed, and the behaviour of the maximum payload-to-orbit trajectory is investigated as the physical characteristics of the launch system are modified, to provide insights into both the optimised trajectory and the launch vehicle design. 

 

\clearpage
\section*{Declaration by author}

  This thesis is composed of my original work, and contains no material previously published or written by another person except where due reference has been made in the text. I have clearly stated the contribution by others to jointly-authored works that I have included in my thesis.

  I have clearly stated the contribution of others to my thesis as a whole, including statistical assistance, survey design, data analysis, significant technical procedures, professional editorial advice, and any other original research work used or reported in my thesis. The content of my thesis is the result of work I have carried out since the commencement of my research higher degree candidature and does not include a substantial part of work that has been submitted to qualify for the award of any other degree or diploma in any university or other tertiary institution. I have clearly stated which parts of my thesis, if any, have been submitted to qualify for another award.

  I acknowledge that an electronic copy of my thesis must be lodged with the University Library and, subject to the policy and procedures of The University of Queensland, the thesis be made available for research and study in accordance with the Copyright Act 1968 unless a period of embargo has been approved by the Dean of the Graduate School.

  I acknowledge that copyright of all material contained in my thesis resides with the copyright holder(s) of that material. Where appropriate I have obtained copyright permission from the copyright holder to reproduce material in this thesis.

\clearpage
\section*{Publications During Candidature}

\subsection*{Journal papers}

\noindent\AtNextCite{\defcounter{maxnames}{99}}\fullcite{ForbesSpyratos2018}\\


\subsection*{Conference papers}

\noindent\AtNextCite{\defcounter{maxnames}{99}}\fullcite{ForbesSpyratos2017}\\

\noindent\AtNextCite{\defcounter{maxnames}{99}}\fullcite{Forbes2018a}\\

\noindent\AtNextCite{\defcounter{maxnames}{99}}\fullcite{chai2017}\\

\section*{Publications Included in This Thesis}

This thesis comprises partly of publications, as allowed by University of Queensland Policy PPL 4.60.07. The papers that have been included have all been published in peer reviewed journals at the time of submission. 

\vspace{\baselineskip}
\AtNextCite{\defcounter{maxnames}{99}}\fullcite{ForbesSpyratos2018}

\begin{center}
  \begin{tabular}{ll}
    \toprule
    Contributor   & Contribution \\
    \midrule
    Sholto O. Forbes-Spyratos             
                                  & Conception and Design (85\%)\\
                                  & Performed simulations (100\%)\\
                                  & Analysis of results (90\%)\\
                                  & Wrote and edited paper (85\%)\\
    \midrule
    Ingo H. Jahn            
                                  & Conception and Design (5\%)\\
                                  & Analysis of results (5\%)\\
                                  & Wrote and edited paper (7.5\%)\\
                                  
        \midrule
        Michael P. Kearney            
								      & Conception and Design (5\%)\\
								      & Wrote and edited paper (7.5\%)\\                       

    \midrule
    Michael K. Smart              
                                  & Conception and Design (5\%)\\
                                  & Analysis of results (5\%)\\
                                  & Wrote and edited paper (5\%)\\
    \bottomrule
  \end{tabular}
\end{center}



\section*{Contributions by Others to the Thesis}

The model of the Baseline SPARTAN was provided for this work by Dr. Dawid Preller and Mr. Joseph Chai, including mass properties, dimensions, and CAD models. The CRESTM10 scramjet engine database was provided for this study by Prof. Michael Smart, consisting of tabulated performance data over a range of inlet conditions. The viscous correction incorporated into the SPARTAN's aerodynamic calculations was performed by Mr. Alexander Ward, and provided for this study in the form of a tabulated aerodynamic database. 




\section*{Manuscripts included in this thesis}
No manuscripts included.

\section*{Research involving Human or Animal Subjects}
No animal or human subjects were involved in this research.

\section*{Statement of Parts of the Thesis Submitted to Qualify for the Award of Another Degree}

None.

\clearpage

\section*{Financial Support}
This research was supported by an Australian Government Research Training Program Scholarship and a University of Queensland Candidate Development Award.
\clearpage


\section*{Acknowledgements} 
  For Kaitlin, with love to my family and friends, and utmost gratitude to my advisors; Ingo Jahn, Michael Kearney, and Michael Smart. 

      \vspace*{\fill}


\clearpage
\subsection*{Keywords}
  airbreathing propulsion, scramjets, hypersonics, access-to-space, small satellite launchers, airbreathing launch systems

\subsection*{Australian and New Zealand Standard Research Classification (ANZSRC)}

  ANZSRC code: 090107 Hypersonic Propulsion and Hypersonic Aerodynamics, 20\% \newline
 ANZSRC code: 090106 Flight Dynamics, 10\% \newline
 ANZSRC code: 090108, Satellite, Space Vehicle and Missile Design and Testing, 25\% \newline
ANZSRC code: 090104 Aircraft Performance and Flight Control Systems, 15\% \newline
ANZSRC code: 010303 Optimisation, 30\% \newline

\subsection*{Fields of Research (FoR) Classification}

  FoR code: 0901, Aerospace Engineering, 100\%

\tableofcontents

\listoffigures
\addcontentsline{toc}{chapter}{List of figures}
\listoftables
\addcontentsline{toc}{chapter}{List of tables}

\newpage
\section*{Nomenclature}

\addcontentsline{toc}{chapter}{Nomenclature}
\markboth{NOMENCLATURE}{NOMENCLATURE}

\textcolor{red}{XXXX Format, put in alphabetical order, add space before subscripts}
\noindent
\begin{tabular}{p{0.8cm}p{0.8cm}p{5.6cm}}
	$t$ & $=$ & Time (s)\\
\end{tabular} 
\begin{tabular}{p{0.8cm}p{0.8cm}p{5.6cm}}
	$\textbf{x}$& $=$ & State Variables\\
\end{tabular} 
\begin{tabular}{p{0.8cm}p{0.8cm}p{5.6cm}}
	$\textbf{u}$& $=$ & Control Variables\\
\end{tabular} 
\begin{tabular}{p{0.8cm}p{0.8cm}p{5.6cm}}
	$q$ & $=$ & Dynamic Pressure (Pa)\\
\end{tabular} 
\begin{tabular}{p{0.8cm}p{0.8cm}p{5.6cm}}
	$H$ & $=$ & Hamiltonian\\
\end{tabular}
\begin{tabular}{p{0.8cm}p{0.8cm}p{5.6cm}}
	$\lambda$ & $=$ & Costate Variables\\
\end{tabular}
\begin{tabular}{p{0.8cm}p{0.8cm}p{5.6cm}}
	$\psi$ & $=$ & Equality Constraint Function\\
\end{tabular}
\begin{tabular}{p{0.8cm}p{0.8cm}p{5.6cm}}
	$b$ & $=$ & Bound Constraint Values\\
\end{tabular}
\begin{tabular}{p{0.8cm}p{0.8cm}p{5.6cm}}
	$\lambda$ & $=$ & Inequality Constraint Function\\
\end{tabular}
\begin{tabular}{p{0.8cm}p{0.8cm}p{5.6cm}}
	$J$ & $=$ & Total Cost Functional\\
\end{tabular}
\begin{tabular}{p{0.8cm}p{0.8cm}p{5.6cm}}
	$I_{sp}$ & $=$ & Specific Impulse (s)\\ 
\end{tabular} 
\begin{tabular}{p{0.8cm}p{0.8cm}p{5.6cm}}
	$M$ & $=$ & Mach Number \& Endpoint Cost Function\\
\end{tabular}
\begin{tabular}{p{0.8cm}p{0.8cm}p{5.6cm}}
	$P$ & $=$ & Pressure (Pa)  \\
\end{tabular}
\begin{tabular}{p{0.8cm}p{0.8cm}p{5.6cm}}
	$F$ & $=$ & Force (N)\\
\end{tabular} 
\begin{tabular}{p{0.8cm}p{0.8cm}p{5.6cm}}
	$\rho$ & $=$ & Density (kg/m$^2$)\\
\end{tabular} 
\begin{tabular}{p{0.8cm}p{0.8cm}p{5.6cm}}
	$R$ & $=$ & Gas Constant (8.314 $J K^{-1} mol^{-1}$)\\
\end{tabular} 
\begin{tabular}{p{0.8cm}p{0.8cm}p{5.6cm}}
	$D$ & $=$ & Drag (N)\\
\end{tabular} 
\begin{tabular}{p{0.8cm}p{0.8cm}p{5.6cm}}
	$L$ & $=$ & Lift (N) \& Integrated Cost Function\\
\end{tabular} 
\begin{tabular}{p{0.8cm}p{0.8cm}p{5.6cm}}
	$C_L,C_D$ & $=$ & Aerodynamic Coefficients\\
\end{tabular} 
\begin{tabular}{p{0.8cm}p{0.8cm}p{5.6cm}}
	$\xi$ & $=$ & Longitude (rad)\\
\end{tabular} 
\begin{tabular}{p{0.8cm}p{0.8cm}p{5.6cm}}
	$\phi$ & $=$ & Latitude (rad) \& Equivalence Ratio\\
\end{tabular} 
\begin{tabular}{p{0.8cm}p{0.8cm}p{5.6cm}}
	$\zeta$ & $=$ & Heading Angle (rad)\\
\end{tabular} 
\begin{tabular}{p{0.8cm}p{0.8cm}p{5.6cm}}
	$v$ & $=$ & Speed (m/s)\\
\end{tabular} 
\begin{tabular}{p{0.8cm}p{0.8cm}p{5.6cm}}
	$A$ & $=$ & Reference Area (m$^2$)\\
\end{tabular} 
\begin{tabular}{p{0.8cm}p{0.8cm}p{5.6cm}}
	$g$ & $=$ & Gravitational Acceleration at Earth's Surface (m/s$^2$)\\
\end{tabular} 
\begin{tabular}{p{0.8cm}p{0.8cm}p{5.6cm}}
	
	
	$r$ & $=$ & Radius from Earth Centre (m)\\
\end{tabular} 
\begin{tabular}{p{0.8cm}p{0.8cm}p{5.6cm}}
	$\gamma$ & $=$ & flight path angle (rad)\\
\end{tabular} 
\begin{tabular}{p{0.8cm}p{0.8cm}p{5.6cm}}
	$\omega$ & $=$ & Angular Velocity (rad/s)\\
\end{tabular} 
\begin{tabular}{p{0.8cm}p{0.8cm}p{5.6cm}}
	$a$ & $=$ & Acceleration (m/s$^2$)\\
\end{tabular} 
\begin{tabular}{p{0.8cm}p{0.8cm}p{5.6cm}}
	$m$ & $=$ & Mass (kg)\\
\end{tabular} 
\begin{tabular}{p{0.8cm}p{0.8cm}p{5.6cm}}
	$T$ & $=$ & Thrust (N) \& Temperature (K)\\
\end{tabular} 
\begin{tabular}{p{0.8cm}p{0.8cm}p{5.6cm}}
	$w_{cap}$ & $=$ & Capture Width\\
\end{tabular} 
\begin{tabular}{p{0.8cm}p{0.8cm}p{5.6cm}}
	$\alpha$ & $=$ & Angle of Attack (rad)\\
\end{tabular} 	
\begin{tabular}{p{0.8cm}p{0.8cm}p{5.6cm}}
	$\eta$ & $=$ & Efficiency\\
\end{tabular} 	
\begin{tabular}{p{0.8cm}p{0.8cm}p{5.6cm}}
	$KE$ & $=$ & Kinetic Energy (J)\\
\end{tabular} 
\begin{tabular}{p{0.8cm}p{0.8cm}p{5.6cm}}
	$PE$ & $=$ & Potential Energy (J)\\
\end{tabular} 
\newline  	
\begin{tabular}{p{5.2cm}p{1cm}p{5cm}}

	\textit{Subscripts} \\
\end{tabular} 
\newline
\begin{tabular}{p{0.8cm}p{0.8cm}p{5.6cm}}
	$E$ & $=$ & Earth\\
	
\end{tabular} 
\begin{tabular}{p{0.8cm}p{0.8cm}p{5.6cm}}
	$1$ & $=$ & $1^{st}$ Stage Rocket\\
\end{tabular} 
\begin{tabular}{p{0.8cm}p{0.8cm}p{5.6cm}}
	$2$ & $=$ & $2^{nd}$ Stage Scramjet Vehicle\\
\end{tabular} 
\begin{tabular}{p{0.8cm}p{0.8cm}p{5.6cm}}
	$3$ & $=$ & $3^{rd}$ Stage Rocket\\
\end{tabular} 
\begin{tabular}{p{0.8cm}p{0.8cm}p{5.6cm}}
	$\rightarrow$ & = & Stage Transition\\
\end{tabular} 
\begin{tabular}{p{0.8cm}p{0.8cm}p{5.6cm}}
	$0$ & $=$ & Initial\\
\end{tabular} 
\begin{tabular}{p{0.8cm}p{0.8cm}p{5.6cm}}
	$f$ & $=$ & Final\\
\end{tabular} 
\begin{tabular}{p{0.8cm}p{0.8cm}p{5.6cm}}
	$a,b$ & $=$ & Nominal Stage Numbers\\
\end{tabular} 
\begin{tabular}{p{0.8cm}p{0.8cm}p{5.6cm}}
	LOX & $=$ & Liquid Oxygen\\
\end{tabular} 
\begin{tabular}{p{0.8cm}p{0.8cm}p{5.6cm}}
	LH2 & $=$ & Liquid Hydrogen\\
	
\end{tabular} 
\begin{tabular}{p{0.8cm}p{0.8cm}p{5.6cm}}
	AoA & $=$ & Angle of Attack\\
	
\end{tabular} 
\begin{tabular}{p{0.8cm}p{0.8cm}p{5.6cm}}
	e & $=$ & Exit\\
\end{tabular} 
\begin{tabular}{p{0.8cm}p{0.8cm}p{5.6cm}}
	$SL$ & $=$ & Sea Level\\
\end{tabular} 
\begin{tabular}{p{0.8cm}p{0.8cm}p{5.6cm}}
	$atm$ & $=$ & Atmospheric\\
	
\end{tabular} 
\begin{tabular}{p{0.8cm}p{0.8cm}p{5.6cm}}
	$vac$ & $=$ & Vacuum\\
	
\end{tabular} 
\begin{tabular}{p{0.8cm}p{0.8cm}p{5.6cm}}
	$LEO$ & $=$ & Low Earth Orbit\\
	
\end{tabular} 
\begin{tabular}{p{0.8cm}p{0.8cm}p{5.6cm}}
	$r$ & $=$ & Radial\\
	
\end{tabular} 
\begin{tabular}{p{0.8cm}p{0.8cm}p{5.6cm}}
	$t$ & $=$ & Tangential\\
	
\end{tabular} 
\begin{tabular}{p{0.8cm}p{0.8cm}p{5.6cm}}
	$p$ & $=$ & Propulsive\\
	
\end{tabular} 
\begin{tabular}{p{0.9cm}p{0.7cm}p{5.6cm}}
	\rom{1}-\rom{7} & $=$ & Trajectory Phases\\
	
\end{tabular} 